% Options for packages loaded elsewhere
\PassOptionsToPackage{unicode}{hyperref}
\PassOptionsToPackage{hyphens}{url}
\PassOptionsToPackage{dvipsnames,svgnames,x11names}{xcolor}
%
\documentclass[
  12pt,
  letterpaper]{article}

\usepackage{amsmath,amssymb}
\usepackage{lmodern}
\usepackage{iftex}
\ifPDFTeX
  \usepackage[T1]{fontenc}
  \usepackage[utf8]{inputenc}
  \usepackage{textcomp} % provide euro and other symbols
\else % if luatex or xetex
  \usepackage{unicode-math}
  \defaultfontfeatures{Scale=MatchLowercase}
  \defaultfontfeatures[\rmfamily]{Ligatures=TeX,Scale=1}
  \setmainfont[]{Charis SIL}
  \setmathfont[]{Monaco}
\fi
% Use upquote if available, for straight quotes in verbatim environments
\IfFileExists{upquote.sty}{\usepackage{upquote}}{}
\IfFileExists{microtype.sty}{% use microtype if available
  \usepackage[]{microtype}
  \UseMicrotypeSet[protrusion]{basicmath} % disable protrusion for tt fonts
}{}
\makeatletter
\@ifundefined{KOMAClassName}{% if non-KOMA class
  \IfFileExists{parskip.sty}{%
    \usepackage{parskip}
  }{% else
    \setlength{\parindent}{0pt}
    \setlength{\parskip}{6pt plus 2pt minus 1pt}}
}{% if KOMA class
  \KOMAoptions{parskip=half}}
\makeatother
\usepackage{xcolor}
\usepackage[margin = 1in]{geometry}
\setlength{\emergencystretch}{3em} % prevent overfull lines
\setcounter{secnumdepth}{-\maxdimen} % remove section numbering
% Make \paragraph and \subparagraph free-standing
\ifx\paragraph\undefined\else
  \let\oldparagraph\paragraph
  \renewcommand{\paragraph}[1]{\oldparagraph{#1}\mbox{}}
\fi
\ifx\subparagraph\undefined\else
  \let\oldsubparagraph\subparagraph
  \renewcommand{\subparagraph}[1]{\oldsubparagraph{#1}\mbox{}}
\fi

\usepackage{color}
\usepackage{fancyvrb}
\newcommand{\VerbBar}{|}
\newcommand{\VERB}{\Verb[commandchars=\\\{\}]}
\DefineVerbatimEnvironment{Highlighting}{Verbatim}{commandchars=\\\{\}}
% Add ',fontsize=\small' for more characters per line
\usepackage{framed}
\definecolor{shadecolor}{RGB}{241,243,245}
\newenvironment{Shaded}{\begin{snugshade}}{\end{snugshade}}
\newcommand{\AlertTok}[1]{\textcolor[rgb]{0.68,0.00,0.00}{#1}}
\newcommand{\AnnotationTok}[1]{\textcolor[rgb]{0.37,0.37,0.37}{#1}}
\newcommand{\AttributeTok}[1]{\textcolor[rgb]{0.40,0.45,0.13}{#1}}
\newcommand{\BaseNTok}[1]{\textcolor[rgb]{0.68,0.00,0.00}{#1}}
\newcommand{\BuiltInTok}[1]{\textcolor[rgb]{0.00,0.23,0.31}{#1}}
\newcommand{\CharTok}[1]{\textcolor[rgb]{0.13,0.47,0.30}{#1}}
\newcommand{\CommentTok}[1]{\textcolor[rgb]{0.37,0.37,0.37}{#1}}
\newcommand{\CommentVarTok}[1]{\textcolor[rgb]{0.37,0.37,0.37}{\textit{#1}}}
\newcommand{\ConstantTok}[1]{\textcolor[rgb]{0.56,0.35,0.01}{#1}}
\newcommand{\ControlFlowTok}[1]{\textcolor[rgb]{0.00,0.23,0.31}{#1}}
\newcommand{\DataTypeTok}[1]{\textcolor[rgb]{0.68,0.00,0.00}{#1}}
\newcommand{\DecValTok}[1]{\textcolor[rgb]{0.68,0.00,0.00}{#1}}
\newcommand{\DocumentationTok}[1]{\textcolor[rgb]{0.37,0.37,0.37}{\textit{#1}}}
\newcommand{\ErrorTok}[1]{\textcolor[rgb]{0.68,0.00,0.00}{#1}}
\newcommand{\ExtensionTok}[1]{\textcolor[rgb]{0.00,0.23,0.31}{#1}}
\newcommand{\FloatTok}[1]{\textcolor[rgb]{0.68,0.00,0.00}{#1}}
\newcommand{\FunctionTok}[1]{\textcolor[rgb]{0.28,0.35,0.67}{#1}}
\newcommand{\ImportTok}[1]{\textcolor[rgb]{0.00,0.46,0.62}{#1}}
\newcommand{\InformationTok}[1]{\textcolor[rgb]{0.37,0.37,0.37}{#1}}
\newcommand{\KeywordTok}[1]{\textcolor[rgb]{0.00,0.23,0.31}{#1}}
\newcommand{\NormalTok}[1]{\textcolor[rgb]{0.00,0.23,0.31}{#1}}
\newcommand{\OperatorTok}[1]{\textcolor[rgb]{0.37,0.37,0.37}{#1}}
\newcommand{\OtherTok}[1]{\textcolor[rgb]{0.00,0.23,0.31}{#1}}
\newcommand{\PreprocessorTok}[1]{\textcolor[rgb]{0.68,0.00,0.00}{#1}}
\newcommand{\RegionMarkerTok}[1]{\textcolor[rgb]{0.00,0.23,0.31}{#1}}
\newcommand{\SpecialCharTok}[1]{\textcolor[rgb]{0.37,0.37,0.37}{#1}}
\newcommand{\SpecialStringTok}[1]{\textcolor[rgb]{0.13,0.47,0.30}{#1}}
\newcommand{\StringTok}[1]{\textcolor[rgb]{0.13,0.47,0.30}{#1}}
\newcommand{\VariableTok}[1]{\textcolor[rgb]{0.07,0.07,0.07}{#1}}
\newcommand{\VerbatimStringTok}[1]{\textcolor[rgb]{0.13,0.47,0.30}{#1}}
\newcommand{\WarningTok}[1]{\textcolor[rgb]{0.37,0.37,0.37}{\textit{#1}}}

\providecommand{\tightlist}{%
  \setlength{\itemsep}{0pt}\setlength{\parskip}{0pt}}\usepackage{longtable,booktabs,array}
\usepackage{calc} % for calculating minipage widths
% Correct order of tables after \paragraph or \subparagraph
\usepackage{etoolbox}
\makeatletter
\patchcmd\longtable{\par}{\if@noskipsec\mbox{}\fi\par}{}{}
\makeatother
% Allow footnotes in longtable head/foot
\IfFileExists{footnotehyper.sty}{\usepackage{footnotehyper}}{\usepackage{footnote}}
\makesavenoteenv{longtable}
\usepackage{graphicx}
\makeatletter
\def\maxwidth{\ifdim\Gin@nat@width>\linewidth\linewidth\else\Gin@nat@width\fi}
\def\maxheight{\ifdim\Gin@nat@height>\textheight\textheight\else\Gin@nat@height\fi}
\makeatother
% Scale images if necessary, so that they will not overflow the page
% margins by default, and it is still possible to overwrite the defaults
% using explicit options in \includegraphics[width, height, ...]{}
\setkeys{Gin}{width=\maxwidth,height=\maxheight,keepaspectratio}
% Set default figure placement to htbp
\makeatletter
\def\fps@figure{htbp}
\makeatother

\usepackage{tabularx}
\usepackage{threeparttable}
\usepackage{booktabs}
\usepackage{tipa}
\let\Oldtexttt\texttt
\renewcommand\texttt[1]{{\ttfamily\color{BrickRed}#1}}
\usepackage{authoraftertitle}
\usepackage{fancyhdr}
\pagestyle{fancy}
\rfoot{\copyright Matt Hunt Gardner}
\cfoot{\thepage}
\lhead{Doing LVC with \textit{R}: \MyTitle}
\rhead{}
\makeatletter
\@ifpackageloaded{tcolorbox}{}{\usepackage[many]{tcolorbox}}
\@ifpackageloaded{fontawesome5}{}{\usepackage{fontawesome5}}
\definecolor{quarto-callout-color}{HTML}{909090}
\definecolor{quarto-callout-note-color}{HTML}{0758E5}
\definecolor{quarto-callout-important-color}{HTML}{CC1914}
\definecolor{quarto-callout-warning-color}{HTML}{EB9113}
\definecolor{quarto-callout-tip-color}{HTML}{00A047}
\definecolor{quarto-callout-caution-color}{HTML}{FC5300}
\definecolor{quarto-callout-color-frame}{HTML}{acacac}
\definecolor{quarto-callout-note-color-frame}{HTML}{4582ec}
\definecolor{quarto-callout-important-color-frame}{HTML}{d9534f}
\definecolor{quarto-callout-warning-color-frame}{HTML}{f0ad4e}
\definecolor{quarto-callout-tip-color-frame}{HTML}{02b875}
\definecolor{quarto-callout-caution-color-frame}{HTML}{fd7e14}
\makeatother
\makeatletter
\makeatother
\makeatletter
\makeatother
\makeatletter
\@ifpackageloaded{caption}{}{\usepackage{caption}}
\AtBeginDocument{%
\ifdefined\contentsname
  \renewcommand*\contentsname{Table of contents}
\else
  \newcommand\contentsname{Table of contents}
\fi
\ifdefined\listfigurename
  \renewcommand*\listfigurename{List of Figures}
\else
  \newcommand\listfigurename{List of Figures}
\fi
\ifdefined\listtablename
  \renewcommand*\listtablename{List of Tables}
\else
  \newcommand\listtablename{List of Tables}
\fi
\ifdefined\figurename
  \renewcommand*\figurename{Figure}
\else
  \newcommand\figurename{Figure}
\fi
\ifdefined\tablename
  \renewcommand*\tablename{Table}
\else
  \newcommand\tablename{Table}
\fi
}
\@ifpackageloaded{float}{}{\usepackage{float}}
\floatstyle{ruled}
\@ifundefined{c@chapter}{\newfloat{codelisting}{h}{lop}}{\newfloat{codelisting}{h}{lop}[chapter]}
\floatname{codelisting}{Listing}
\newcommand*\listoflistings{\listof{codelisting}{List of Listings}}
\makeatother
\makeatletter
\@ifpackageloaded{caption}{}{\usepackage{caption}}
\@ifpackageloaded{subcaption}{}{\usepackage{subcaption}}
\makeatother
\makeatletter
\@ifpackageloaded{tcolorbox}{}{\usepackage[many]{tcolorbox}}
\makeatother
\makeatletter
\@ifundefined{shadecolor}{\definecolor{shadecolor}{rgb}{.97, .97, .97}}
\makeatother
\makeatletter
\makeatother
\ifLuaTeX
  \usepackage{selnolig}  % disable illegal ligatures
\fi
\IfFileExists{bookmark.sty}{\usepackage{bookmark}}{\usepackage{hyperref}}
\IfFileExists{xurl.sty}{\usepackage{xurl}}{} % add URL line breaks if available
\urlstyle{same} % disable monospaced font for URLs
% Make links footnotes instead of hotlinks:
\DeclareRobustCommand{\href}[2]{#2\footnote{\url{#1}}}
\hypersetup{
  pdftitle={Crosstabs: Counts, Proportions, and More},
  pdfauthor={Matt Hunt Gardner},
  colorlinks=true,
  linkcolor={blue},
  filecolor={Maroon},
  citecolor={Blue},
  urlcolor={Blue},
  pdfcreator={LaTeX via pandoc}}

\title{Crosstabs: Counts, Proportions, and More}
\usepackage{etoolbox}
\makeatletter
\providecommand{\subtitle}[1]{% add subtitle to \maketitle
  \apptocmd{\@title}{\par {\large #1 \par}}{}{}
}
\makeatother
\subtitle{from
\href{https://lingmethodshub.github.io/content/R/lvc_r/}{Doing LVC with
\emph{R}}}
\author{Matt Hunt Gardner}
\date{3/10/23}

\begin{document}
\maketitle
\ifdefined\Shaded\renewenvironment{Shaded}{\begin{tcolorbox}[breakable, sharp corners, frame hidden, interior hidden, borderline west={3pt}{0pt}{shadecolor}, enhanced, boxrule=0pt]}{\end{tcolorbox}}\fi

\renewcommand*\contentsname{Table of contents}
{
\hypersetup{linkcolor=}
\setcounter{tocdepth}{3}
\tableofcontents
}
It took me two years to figure out how to do cross-tabs in \emph{R} the
way that \emph{Goldvarb} does cross-tabs. Below I show you how to build
cross-tabs from scratch.

\hypertarget{token-counts}{%
\subsection{Token Counts}\label{token-counts}}

A good starting point is the function \texttt{table()}. This function
returns token numbers.

\begin{tcolorbox}[enhanced jigsaw, rightrule=.15mm, leftrule=.75mm, title=\textcolor{quarto-callout-tip-color}{\faLightbulb}\hspace{0.5em}{Get the data first}, coltitle=black, colframe=quarto-callout-tip-color-frame, toprule=.15mm, breakable, left=2mm, bottomrule=.15mm, colback=white, bottomtitle=1mm, opacitybacktitle=0.6, titlerule=0mm, arc=.35mm, opacityback=0, toptitle=1mm, colbacktitle=quarto-callout-tip-color!10!white]

If you don't have the \texttt{td} data loaded in \emph{R}, go back to
\href{https://lingmethodshub.github.io/content/R/lvc_r/050_lvcr.html}{Doing
it all again, but \texttt{tidy}} and run the code.

\end{tcolorbox}

\begin{Shaded}
\begin{Highlighting}[]
\CommentTok{\# Get the number of tokens by level of Dep.Var}
\FunctionTok{table}\NormalTok{(td}\SpecialCharTok{$}\NormalTok{Dep.Var)}
\end{Highlighting}
\end{Shaded}

\begin{verbatim}

Deletion Realized 
     386      803 
\end{verbatim}

This tells you that there are 386 \texttt{Deletion} tokens and 803 not
deleted, or \texttt{Realized} tokens. If you add another factor group
like \texttt{Age.Group}, you get the number of tokens for each level of
\texttt{Dep.Var} for each level of that additional factor group. These
two factor groups are returned as the rows and then columns in the
table.

\begin{Shaded}
\begin{Highlighting}[]
\CommentTok{\# Get the number of tokens by level of Dep.Var}
\CommentTok{\# and Age.Group}
\FunctionTok{table}\NormalTok{(td}\SpecialCharTok{$}\NormalTok{Dep.Var, td}\SpecialCharTok{$}\NormalTok{Age.Group)}
\end{Highlighting}
\end{Shaded}

\begin{verbatim}
          
           Old Middle Young
  Deletion  67    125   194
  Realized 134    235   434
\end{verbatim}

If you add one more factor group, \texttt{Sex}, it divides the data in
what \emph{R} calls ``pages''. The first page is the number of tokens
for each level of \texttt{Dep.Var} by each level of \texttt{Age.Group}
for female data (\texttt{Sex\ =\ F}), and then the same for the male
data (\texttt{Sex\ =\ M}).

\begin{Shaded}
\begin{Highlighting}[]
\CommentTok{\# Get the number of tokens by Dep.Var, Sex, and}
\CommentTok{\# Age.Group}
\FunctionTok{table}\NormalTok{(td}\SpecialCharTok{$}\NormalTok{Dep.Var, td}\SpecialCharTok{$}\NormalTok{Age.Group, td}\SpecialCharTok{$}\NormalTok{Sex)}
\end{Highlighting}
\end{Shaded}

\begin{verbatim}
, ,  = F

          
           Old Middle Young
  Deletion  43     73    72
  Realized 107    165   199

, ,  = M

          
           Old Middle Young
  Deletion  24     52   122
  Realized  27     70   235
\end{verbatim}

You can add the option \texttt{deparse.level\ =\ 2} to include the names
of the columns in the table.

\begin{Shaded}
\begin{Highlighting}[]
\CommentTok{\# Get the number of tokens by Dep.Var, Sex, and}
\CommentTok{\# Age.Group}
\FunctionTok{table}\NormalTok{(td}\SpecialCharTok{$}\NormalTok{Dep.Var, td}\SpecialCharTok{$}\NormalTok{Age.Group, td}\SpecialCharTok{$}\NormalTok{Sex, }\AttributeTok{deparse.level =} \DecValTok{2}\NormalTok{)}
\end{Highlighting}
\end{Shaded}

\begin{verbatim}
, , td$Sex = F

          td$Age.Group
td$Dep.Var Old Middle Young
  Deletion  43     73    72
  Realized 107    165   199

, , td$Sex = M

          td$Age.Group
td$Dep.Var Old Middle Young
  Deletion  24     52   122
  Realized  27     70   235
\end{verbatim}

If you wrap the \texttt{table()} function in the \texttt{addmargins()}
function you get the sums of each row and column, and another page for
both the male and the female data together.

\begin{Shaded}
\begin{Highlighting}[]
\CommentTok{\# Get the number of tokens by Dep.Var, Sex, and}
\CommentTok{\# Age.Group, with column, row and page totals}
\FunctionTok{addmargins}\NormalTok{(}\FunctionTok{table}\NormalTok{(td}\SpecialCharTok{$}\NormalTok{Dep.Var, td}\SpecialCharTok{$}\NormalTok{Age.Group, td}\SpecialCharTok{$}\NormalTok{Sex,}
    \AttributeTok{deparse.level =} \DecValTok{2}\NormalTok{))}
\end{Highlighting}
\end{Shaded}

\begin{verbatim}
, , td$Sex = F

          td$Age.Group
td$Dep.Var  Old Middle Young  Sum
  Deletion   43     73    72  188
  Realized  107    165   199  471
  Sum       150    238   271  659

, , td$Sex = M

          td$Age.Group
td$Dep.Var  Old Middle Young  Sum
  Deletion   24     52   122  198
  Realized   27     70   235  332
  Sum        51    122   357  530

, , td$Sex = Sum

          td$Age.Group
td$Dep.Var  Old Middle Young  Sum
  Deletion   67    125   194  386
  Realized  134    235   434  803
  Sum       201    360   628 1189
\end{verbatim}

If you change the order of factor groups you include in the
\texttt{table()} function you can change which factors are rows, which
are columns, and which are pages. You can also keep adding factors as
additional pages. The order is always: rows, columns, page 1, page 2,
etc.

\begin{Shaded}
\begin{Highlighting}[]
\CommentTok{\# Get the number of tokens by Age.Group,}
\CommentTok{\# Education, Sex, and Dep.Var, with row, column,}
\CommentTok{\# and page totals}
\FunctionTok{addmargins}\NormalTok{(}\FunctionTok{table}\NormalTok{(td}\SpecialCharTok{$}\NormalTok{Age.Group, td}\SpecialCharTok{$}\NormalTok{Education, td}\SpecialCharTok{$}\NormalTok{Sex,}
\NormalTok{    td}\SpecialCharTok{$}\NormalTok{Dep.Var, }\AttributeTok{deparse.level =} \DecValTok{2}\NormalTok{))}
\end{Highlighting}
\end{Shaded}

\begin{verbatim}
, , td$Sex = F, td$Dep.Var = Deletion

            td$Education
td$Age.Group Educated Not Educated Student  Sum
      Old           2           41       0   43
      Middle       68            5       0   73
      Young        20            0      52   72
      Sum          90           46      52  188

, , td$Sex = M, td$Dep.Var = Deletion

            td$Education
td$Age.Group Educated Not Educated Student  Sum
      Old           0           24       0   24
      Middle       16           36       0   52
      Young        48           24      50  122
      Sum          64           84      50  198

, , td$Sex = Sum, td$Dep.Var = Deletion

            td$Education
td$Age.Group Educated Not Educated Student  Sum
      Old           2           65       0   67
      Middle       84           41       0  125
      Young        68           24     102  194
      Sum         154          130     102  386

, , td$Sex = F, td$Dep.Var = Realized

            td$Education
td$Age.Group Educated Not Educated Student  Sum
      Old          30           77       0  107
      Middle      153           12       0  165
      Young        52            0     147  199
      Sum         235           89     147  471

, , td$Sex = M, td$Dep.Var = Realized

            td$Education
td$Age.Group Educated Not Educated Student  Sum
      Old           0           27       0   27
      Middle       30           40       0   70
      Young        77           31     127  235
      Sum         107           98     127  332

, , td$Sex = Sum, td$Dep.Var = Realized

            td$Education
td$Age.Group Educated Not Educated Student  Sum
      Old          30          104       0  134
      Middle      183           52       0  235
      Young       129           31     274  434
      Sum         342          187     274  803

, , td$Sex = F, td$Dep.Var = Sum

            td$Education
td$Age.Group Educated Not Educated Student  Sum
      Old          32          118       0  150
      Middle      221           17       0  238
      Young        72            0     199  271
      Sum         325          135     199  659

, , td$Sex = M, td$Dep.Var = Sum

            td$Education
td$Age.Group Educated Not Educated Student  Sum
      Old           0           51       0   51
      Middle       46           76       0  122
      Young       125           55     177  357
      Sum         171          182     177  530

, , td$Sex = Sum, td$Dep.Var = Sum

            td$Education
td$Age.Group Educated Not Educated Student  Sum
      Old          32          169       0  201
      Middle      267           93       0  360
      Young       197           55     376  628
      Sum         496          317     376 1189
\end{verbatim}

The above function produces 9 ``pages'', one for each combination of
\texttt{Sex} (two levels) and \texttt{Dep.Var} (two levels), plus the
sum of each (one additional level each), and the sum for both. With more
than three factor groups like this it is very useful to have the column
names included in the output. Scroll to the sixth page, for example (the
one that begins
\texttt{,\ ,\ td\$Sex\ =\ Sum,\ td\$Dep.Var\ =\ Realized}). It shows the
number of tokens by \texttt{Age.Group} and \texttt{Education} (the first
two factor groups in the function), when \texttt{Sex} equals
\texttt{Sum} (e.g., \texttt{M} and \texttt{F} combined) and
\texttt{Dep.Var} equals \texttt{Realized}.

One advantage of doing cross-tabs in \emph{R}, rather than
\emph{Goldvarb}, is that you can simultaneously cross more than two
factor groups at once. But, the presentation of these factors in pages
may not be the most useful. The function \texttt{ftable()} in the
package \texttt{vcd} presents the cross-tab in a more condensed format.
The last factor group in the \texttt{table()} function will be the
variable for the columns in \texttt{ftable()}, so you always want to
make that the dependent variable. Below is the \texttt{ftable()} for the
cross-tab of \texttt{Age.Group}, \texttt{Education}, \texttt{Sex}, and
\texttt{Dep.Var}. You can see, for example, that there are 52
\texttt{Deletion} tokens from young, student, female speakers and that
there are no tokens from old, educated men.

\begin{Shaded}
\begin{Highlighting}[]
\CommentTok{\# Get the number of tokens by Age.Group,}
\CommentTok{\# Education, Sex, and Dep.Var, with row, column}
\CommentTok{\# and page totals, presented in a flattened table}
\FunctionTok{library}\NormalTok{(vcd)}
\FunctionTok{ftable}\NormalTok{(}\FunctionTok{table}\NormalTok{(td}\SpecialCharTok{$}\NormalTok{Age.Group, td}\SpecialCharTok{$}\NormalTok{Education, td}\SpecialCharTok{$}\NormalTok{Sex, td}\SpecialCharTok{$}\NormalTok{Dep.Var))}
\end{Highlighting}
\end{Shaded}

\begin{verbatim}
                       Deletion Realized
                                        
Old    Educated     F         2       30
                    M         0        0
       Not Educated F        41       77
                    M        24       27
       Student      F         0        0
                    M         0        0
Middle Educated     F        68      153
                    M        16       30
       Not Educated F         5       12
                    M        36       40
       Student      F         0        0
                    M         0        0
Young  Educated     F        20       52
                    M        48       77
       Not Educated F         0        0
                    M        24       31
       Student      F        52      147
                    M        50      127
\end{verbatim}

\begin{Shaded}
\begin{Highlighting}[]
\CommentTok{\# Do the same but include the margin values}
\FunctionTok{ftable}\NormalTok{(}\FunctionTok{addmargins}\NormalTok{(}\FunctionTok{table}\NormalTok{(td}\SpecialCharTok{$}\NormalTok{Age.Group, td}\SpecialCharTok{$}\NormalTok{Education,}
\NormalTok{    td}\SpecialCharTok{$}\NormalTok{Sex, td}\SpecialCharTok{$}\NormalTok{Dep.Var)))}
\end{Highlighting}
\end{Shaded}

\begin{verbatim}
                         Deletion Realized  Sum
                                               
Old    Educated     F           2       30   32
                    M           0        0    0
                    Sum         2       30   32
       Not Educated F          41       77  118
                    M          24       27   51
                    Sum        65      104  169
       Student      F           0        0    0
                    M           0        0    0
                    Sum         0        0    0
       Sum          F          43      107  150
                    M          24       27   51
                    Sum        67      134  201
Middle Educated     F          68      153  221
                    M          16       30   46
                    Sum        84      183  267
       Not Educated F           5       12   17
                    M          36       40   76
                    Sum        41       52   93
       Student      F           0        0    0
                    M           0        0    0
                    Sum         0        0    0
       Sum          F          73      165  238
                    M          52       70  122
                    Sum       125      235  360
Young  Educated     F          20       52   72
                    M          48       77  125
                    Sum        68      129  197
       Not Educated F           0        0    0
                    M          24       31   55
                    Sum        24       31   55
       Student      F          52      147  199
                    M          50      127  177
                    Sum       102      274  376
       Sum          F          72      199  271
                    M         122      235  357
                    Sum       194      434  628
Sum    Educated     F          90      235  325
                    M          64      107  171
                    Sum       154      342  496
       Not Educated F          46       89  135
                    M          84       98  182
                    Sum       130      187  317
       Student      F          52      147  199
                    M          50      127  177
                    Sum       102      274  376
       Sum          F         188      471  659
                    M         198      332  530
                    Sum       386      803 1189
\end{verbatim}

Of course we can use the pipe \texttt{\%\textgreater{}\%} to make things
a bit easier

\begin{Shaded}
\begin{Highlighting}[]
\CommentTok{\# Get the number of tokens by Age.Group,}
\CommentTok{\# Education, Sex, and Dep.Var, with row, column}
\CommentTok{\# and page totals, presented in a flattened table}
\FunctionTok{table}\NormalTok{(td}\SpecialCharTok{$}\NormalTok{Age.Group, td}\SpecialCharTok{$}\NormalTok{Education, td}\SpecialCharTok{$}\NormalTok{Sex, td}\SpecialCharTok{$}\NormalTok{Dep.Var) }\SpecialCharTok{\%\textgreater{}\%}
    \FunctionTok{addmargins}\NormalTok{() }\SpecialCharTok{\%\textgreater{}\%}
    \FunctionTok{ftable}\NormalTok{()}
\end{Highlighting}
\end{Shaded}

\begin{verbatim}
                         Deletion Realized  Sum
                                               
Old    Educated     F           2       30   32
                    M           0        0    0
                    Sum         2       30   32
       Not Educated F          41       77  118
                    M          24       27   51
                    Sum        65      104  169
       Student      F           0        0    0
                    M           0        0    0
                    Sum         0        0    0
       Sum          F          43      107  150
                    M          24       27   51
                    Sum        67      134  201
Middle Educated     F          68      153  221
                    M          16       30   46
                    Sum        84      183  267
       Not Educated F           5       12   17
                    M          36       40   76
                    Sum        41       52   93
       Student      F           0        0    0
                    M           0        0    0
                    Sum         0        0    0
       Sum          F          73      165  238
                    M          52       70  122
                    Sum       125      235  360
Young  Educated     F          20       52   72
                    M          48       77  125
                    Sum        68      129  197
       Not Educated F           0        0    0
                    M          24       31   55
                    Sum        24       31   55
       Student      F          52      147  199
                    M          50      127  177
                    Sum       102      274  376
       Sum          F          72      199  271
                    M         122      235  357
                    Sum       194      434  628
Sum    Educated     F          90      235  325
                    M          64      107  171
                    Sum       154      342  496
       Not Educated F          46       89  135
                    M          84       98  182
                    Sum       130      187  317
       Student      F          52      147  199
                    M          50      127  177
                    Sum       102      274  376
       Sum          F         188      471  659
                    M         198      332  530
                    Sum       386      803 1189
\end{verbatim}

Another \texttt{tidy} way to find out the number of tokens by the
different levels of a factor group is using the \texttt{group\_by()} and
\texttt{tally()} functions. First, we specify how to group the data,
i.e., what combination of factors we want to investigate. In this case,
we want the number of tokens for every combination of
\texttt{Age.Group}, \texttt{Education}, \texttt{Sex} and
\texttt{Dep.Var}. Next we use the \texttt{tally()} function to provide
the token counts for each of those combinations. The results are very
similar to those produced by \texttt{ftable(table())}.

\begin{Shaded}
\begin{Highlighting}[]
\CommentTok{\# Group data by Age, Education, and Sex then}
\CommentTok{\# tally each group}
\NormalTok{td }\SpecialCharTok{\%\textgreater{}\%}
    \FunctionTok{group\_by}\NormalTok{(Age.Group, Education, Sex, Dep.Var) }\SpecialCharTok{\%\textgreater{}\%}
    \FunctionTok{tally}\NormalTok{()}
\end{Highlighting}
\end{Shaded}

\begin{verbatim}
# A tibble: 24 x 5
# Groups:   Age.Group, Education, Sex [12]
   Age.Group Education    Sex   Dep.Var      n
   <fct>     <fct>        <fct> <fct>    <int>
 1 Old       Educated     F     Deletion     2
 2 Old       Educated     F     Realized    30
 3 Old       Not Educated F     Deletion    41
 4 Old       Not Educated F     Realized    77
 5 Old       Not Educated M     Deletion    24
 6 Old       Not Educated M     Realized    27
 7 Middle    Educated     F     Deletion    68
 8 Middle    Educated     F     Realized   153
 9 Middle    Educated     M     Deletion    16
10 Middle    Educated     M     Realized    30
# ... with 14 more rows
\end{verbatim}

As the results of \texttt{tally()} is a \emph{tibble}, only the first 10
rows will be printed. To print all the rows add \texttt{print(n=Inf)} at
the end.

\begin{Shaded}
\begin{Highlighting}[]
\CommentTok{\# Group data by Age, Education, and Sex, tally}
\CommentTok{\# each group, then print all rows}
\NormalTok{td }\SpecialCharTok{\%\textgreater{}\%}
    \FunctionTok{group\_by}\NormalTok{(Age.Group, Education, Sex, Dep.Var) }\SpecialCharTok{\%\textgreater{}\%}
    \FunctionTok{tally}\NormalTok{() }\SpecialCharTok{\%\textgreater{}\%}
    \FunctionTok{print}\NormalTok{(}\AttributeTok{n =} \ConstantTok{Inf}\NormalTok{)}
\end{Highlighting}
\end{Shaded}

\begin{verbatim}
# A tibble: 24 x 5
# Groups:   Age.Group, Education, Sex [12]
   Age.Group Education    Sex   Dep.Var      n
   <fct>     <fct>        <fct> <fct>    <int>
 1 Old       Educated     F     Deletion     2
 2 Old       Educated     F     Realized    30
 3 Old       Not Educated F     Deletion    41
 4 Old       Not Educated F     Realized    77
 5 Old       Not Educated M     Deletion    24
 6 Old       Not Educated M     Realized    27
 7 Middle    Educated     F     Deletion    68
 8 Middle    Educated     F     Realized   153
 9 Middle    Educated     M     Deletion    16
10 Middle    Educated     M     Realized    30
11 Middle    Not Educated F     Deletion     5
12 Middle    Not Educated F     Realized    12
13 Middle    Not Educated M     Deletion    36
14 Middle    Not Educated M     Realized    40
15 Young     Educated     F     Deletion    20
16 Young     Educated     F     Realized    52
17 Young     Educated     M     Deletion    48
18 Young     Educated     M     Realized    77
19 Young     Not Educated M     Deletion    24
20 Young     Not Educated M     Realized    31
21 Young     Student      F     Deletion    52
22 Young     Student      F     Realized   147
23 Young     Student      M     Deletion    50
24 Young     Student      M     Realized   127
\end{verbatim}

The above code gives us the number of \texttt{Realized} and
\texttt{Deletion} tokens for each combination of \texttt{Age.Group},
\texttt{Education}, and \texttt{Sex}. What if we want the total number
of tokens for each combination, rather than the number of each level of
\texttt{Dep.Var}. In this case, you can just drop \texttt{Dep.Var} from
the \texttt{group\_by()} function.

\begin{Shaded}
\begin{Highlighting}[]
\CommentTok{\# Get total number of tokens per group by}
\CommentTok{\# removing Dep.Var}
\NormalTok{td }\SpecialCharTok{\%\textgreater{}\%}
    \FunctionTok{group\_by}\NormalTok{(Age.Group, Education, Sex) }\SpecialCharTok{\%\textgreater{}\%}
    \FunctionTok{tally}\NormalTok{() }\SpecialCharTok{\%\textgreater{}\%}
    \FunctionTok{print}\NormalTok{(}\AttributeTok{n =} \ConstantTok{Inf}\NormalTok{)}
\end{Highlighting}
\end{Shaded}

\begin{verbatim}
# A tibble: 12 x 4
# Groups:   Age.Group, Education [7]
   Age.Group Education    Sex       n
   <fct>     <fct>        <fct> <int>
 1 Old       Educated     F        32
 2 Old       Not Educated F       118
 3 Old       Not Educated M        51
 4 Middle    Educated     F       221
 5 Middle    Educated     M        46
 6 Middle    Not Educated F        17
 7 Middle    Not Educated M        76
 8 Young     Educated     F        72
 9 Young     Educated     M       125
10 Young     Not Educated M        55
11 Young     Student      F       199
12 Young     Student      M       177
\end{verbatim}

We know now that there are 32 tokens from \texttt{Old},
\texttt{Educated}, \texttt{F} (female) speakers. The previous
\texttt{tally()} shows us that 2 of the tokens are \texttt{Deletion} and
30 are of \texttt{Realized}.

An alternative to \texttt{tally()} is the much more flexible
\texttt{summarize()} function.\footnote{\texttt{summarise()} and
  \texttt{summarize()} are synonyms.} With this function you can apply a
summary statistic function to each combination of the grouping
variables. If no summary statistic function is created, the a tibble of
the combination of the groups is produced.

\begin{Shaded}
\begin{Highlighting}[]
\CommentTok{\# Create a tibble of all combinations of}
\CommentTok{\# Age.Group, Education, and Sex (for which there}
\CommentTok{\# are rows of data)}
\NormalTok{td }\SpecialCharTok{\%\textgreater{}\%}
    \FunctionTok{group\_by}\NormalTok{(Age.Group, Education, Sex) }\SpecialCharTok{\%\textgreater{}\%}
    \FunctionTok{summarize}\NormalTok{()}
\end{Highlighting}
\end{Shaded}

\begin{verbatim}
# A tibble: 12 x 3
# Groups:   Age.Group, Education [7]
   Age.Group Education    Sex  
   <fct>     <fct>        <fct>
 1 Old       Educated     F    
 2 Old       Not Educated F    
 3 Old       Not Educated M    
 4 Middle    Educated     F    
 5 Middle    Educated     M    
 6 Middle    Not Educated F    
 7 Middle    Not Educated M    
 8 Young     Educated     F    
 9 Young     Educated     M    
10 Young     Not Educated M    
11 Young     Student      F    
12 Young     Student      M    
\end{verbatim}

To get the count, or number of rows, of each combination, we create a
new column in the tibble that is the output of \texttt{summarize()} and
assign to it the value of the count function \texttt{n()}

\begin{Shaded}
\begin{Highlighting}[]
\CommentTok{\# Create a tibble of grouping variables, then add}
\CommentTok{\# a new column \textquotesingle{}Tokens\textquotesingle{} with the value of the}
\CommentTok{\# count function}
\NormalTok{td }\SpecialCharTok{\%\textgreater{}\%}
    \FunctionTok{group\_by}\NormalTok{(Age.Group, Education, Sex, Dep.Var) }\SpecialCharTok{\%\textgreater{}\%}
    \FunctionTok{summarize}\NormalTok{(}\AttributeTok{Tokens =} \FunctionTok{n}\NormalTok{()) }\SpecialCharTok{\%\textgreater{}\%}
    \FunctionTok{print}\NormalTok{(}\AttributeTok{n =} \ConstantTok{Inf}\NormalTok{)}
\end{Highlighting}
\end{Shaded}

\begin{verbatim}
# A tibble: 24 x 5
# Groups:   Age.Group, Education, Sex [12]
   Age.Group Education    Sex   Dep.Var  Tokens
   <fct>     <fct>        <fct> <fct>     <int>
 1 Old       Educated     F     Deletion      2
 2 Old       Educated     F     Realized     30
 3 Old       Not Educated F     Deletion     41
 4 Old       Not Educated F     Realized     77
 5 Old       Not Educated M     Deletion     24
 6 Old       Not Educated M     Realized     27
 7 Middle    Educated     F     Deletion     68
 8 Middle    Educated     F     Realized    153
 9 Middle    Educated     M     Deletion     16
10 Middle    Educated     M     Realized     30
11 Middle    Not Educated F     Deletion      5
12 Middle    Not Educated F     Realized     12
13 Middle    Not Educated M     Deletion     36
14 Middle    Not Educated M     Realized     40
15 Young     Educated     F     Deletion     20
16 Young     Educated     F     Realized     52
17 Young     Educated     M     Deletion     48
18 Young     Educated     M     Realized     77
19 Young     Not Educated M     Deletion     24
20 Young     Not Educated M     Realized     31
21 Young     Student      F     Deletion     52
22 Young     Student      F     Realized    147
23 Young     Student      M     Deletion     50
24 Young     Student      M     Realized    127
\end{verbatim}

The \texttt{summarize()} function can be used with a number of summary
statistic functions, including, but not limited to, the following:

\begin{longtable}[]{@{}ll@{}}
\toprule()
Type & Some Useful Functions \\
\midrule()
\endhead
Center & \texttt{mean()}, \texttt{median()} \\
Spread & \texttt{sd()}, \texttt{IQR()} \\
Range & \texttt{min()}, \texttt{max()} \\
Position & \texttt{first()}, \texttt{last()}, \texttt{nth()} \\
Count & \texttt{n()}, \texttt{n\_distinct()} \\
Logical & \texttt{any()}, \texttt{all()} \\
\bottomrule()
\end{longtable}

\hypertarget{summary-statistics-for-continous-variables}{%
\subsection{Summary Statistics for Continous
Variables}\label{summary-statistics-for-continous-variables}}

This seems like an appropriate place to describe how to summarize values
that are continous, like \texttt{YOB}. Normally in variationist
sociolinguistics we are very concerned with frequency and proportion of
usage, and we will explore how to generate those statistics in the
following section. Here, however, let's explore the functions available
to use inside \texttt{summarize()}. These functions can be used on their
own, also. For example, the first two, \texttt{mean()} and
\texttt{median()} provide the arithmetic mean (basically the average) of
a set of numbers while the \texttt{median()} provides the exact middle
number of a set of values organized from smallest to largest (if there
are an even number of values, \texttt{median()} returns the halfway
point between the two middle numbers).

\begin{Shaded}
\begin{Highlighting}[]
\CommentTok{\# Get mean year of birth}
\FunctionTok{mean}\NormalTok{(td}\SpecialCharTok{$}\NormalTok{YOB)}
\end{Highlighting}
\end{Shaded}

\begin{verbatim}
[1] 1969.447
\end{verbatim}

\begin{Shaded}
\begin{Highlighting}[]
\CommentTok{\# Get median year of birth}
\FunctionTok{median}\NormalTok{(td}\SpecialCharTok{$}\NormalTok{YOB)}
\end{Highlighting}
\end{Shaded}

\begin{verbatim}
[1] 1984
\end{verbatim}

We already know that the mean year of birth for the \texttt{td} data set
is 1969.447. You can also see that the middle number of all years of
birth organized from oldest to youngest is 1984. If we wanted to find
the mean or median year of birth for either just male or just female
speakers, we have two options. We can use the base filter technique, or
we can use the \texttt{tidy} method to group the data and summarize it.

\begin{Shaded}
\begin{Highlighting}[]
\CommentTok{\# Get mean year of birth of just female speakers}
\FunctionTok{mean}\NormalTok{(td}\SpecialCharTok{$}\NormalTok{YOB[td}\SpecialCharTok{$}\NormalTok{Sex }\SpecialCharTok{==} \StringTok{"F"}\NormalTok{])}
\end{Highlighting}
\end{Shaded}

\begin{verbatim}
[1] 1963.487
\end{verbatim}

\begin{Shaded}
\begin{Highlighting}[]
\CommentTok{\# Get mean year of birth of just male speaker}
\FunctionTok{mean}\NormalTok{(td}\SpecialCharTok{$}\NormalTok{YOB[td}\SpecialCharTok{$}\NormalTok{Sex }\SpecialCharTok{==} \StringTok{"M"}\NormalTok{])}
\end{Highlighting}
\end{Shaded}

\begin{verbatim}
[1] 1976.857
\end{verbatim}

\begin{Shaded}
\begin{Highlighting}[]
\CommentTok{\# Get mean year of birth for each level of Sex}
\NormalTok{td }\SpecialCharTok{\%\textgreater{}\%}
    \FunctionTok{group\_by}\NormalTok{(Sex) }\SpecialCharTok{\%\textgreater{}\%}
    \FunctionTok{summarize}\NormalTok{(}\AttributeTok{Mean.YOB =} \FunctionTok{mean}\NormalTok{(YOB))}
\end{Highlighting}
\end{Shaded}

\begin{verbatim}
# A tibble: 2 x 2
  Sex   Mean.YOB
  <fct>    <dbl>
1 F        1963.
2 M        1977.
\end{verbatim}

\hypertarget{dealing-with-decimals}{%
\subsubsection{Dealing with Decimals}\label{dealing-with-decimals}}

\emph{Tibbles} are intended to be succinct and concise, so they provide
very few values after the decimal place by default. If you require more
decimal values, the easiest (trust me) thing to do is to convert the
tibble into a \emph{data frame}.

\begin{Shaded}
\begin{Highlighting}[]
\CommentTok{\# Get mean year of birth by Sex, converted to}
\CommentTok{\# data frame}
\NormalTok{td }\SpecialCharTok{\%\textgreater{}\%}
    \FunctionTok{group\_by}\NormalTok{(Sex) }\SpecialCharTok{\%\textgreater{}\%}
    \FunctionTok{summarize}\NormalTok{(}\AttributeTok{Mean.YOB =} \FunctionTok{mean}\NormalTok{(YOB)) }\SpecialCharTok{\%\textgreater{}\%}
    \FunctionTok{as.data.frame}\NormalTok{()}
\end{Highlighting}
\end{Shaded}

\begin{verbatim}
  Sex Mean.YOB
1   F 1963.487
2   M 1976.857
\end{verbatim}

\emph{data frames} will display whole numbers, and numbers with decimals
up to the total number of digits set by \texttt{options()} function.
Keep in mind, though, that changing this value changes the global
options for \emph{R}. An alternative is to use the \texttt{format()}
function.

\begin{Shaded}
\begin{Highlighting}[]
\CommentTok{\# Change number of significant digits displayed}
\CommentTok{\# to 6}
\FunctionTok{options}\NormalTok{(}\AttributeTok{digits =} \DecValTok{6}\NormalTok{)}
\CommentTok{\# Get mean year of birth by sex, converted to}
\CommentTok{\# data frame}
\NormalTok{td }\SpecialCharTok{\%\textgreater{}\%}
    \FunctionTok{group\_by}\NormalTok{(Sex) }\SpecialCharTok{\%\textgreater{}\%}
    \FunctionTok{summarize}\NormalTok{(}\AttributeTok{Mean.YOB =} \FunctionTok{mean}\NormalTok{(YOB)) }\SpecialCharTok{\%\textgreater{}\%}
    \FunctionTok{as.data.frame}\NormalTok{()}
\end{Highlighting}
\end{Shaded}

\begin{verbatim}
  Sex Mean.YOB
1   F  1963.49
2   M  1976.86
\end{verbatim}

\begin{Shaded}
\begin{Highlighting}[]
\CommentTok{\# Change number of significant digits displayed}
\CommentTok{\# to 10}
\FunctionTok{options}\NormalTok{(}\AttributeTok{digits =} \DecValTok{10}\NormalTok{)}
\CommentTok{\# Get mean year of birth by sex, converted to}
\CommentTok{\# data frame}
\NormalTok{td }\SpecialCharTok{\%\textgreater{}\%}
    \FunctionTok{group\_by}\NormalTok{(Sex) }\SpecialCharTok{\%\textgreater{}\%}
    \FunctionTok{summarize}\NormalTok{(}\AttributeTok{Mean.YOB =} \FunctionTok{mean}\NormalTok{(YOB)) }\SpecialCharTok{\%\textgreater{}\%}
    \FunctionTok{as.data.frame}\NormalTok{()}
\end{Highlighting}
\end{Shaded}

\begin{verbatim}
  Sex    Mean.YOB
1   F 1963.487102
2   M 1976.856604
\end{verbatim}

\begin{Shaded}
\begin{Highlighting}[]
\CommentTok{\# Change number of significant digits displayed}
\CommentTok{\# to 3}
\FunctionTok{options}\NormalTok{(}\AttributeTok{digits =} \DecValTok{3}\NormalTok{)}
\CommentTok{\# Get mean year of birth by sex, converted to}
\CommentTok{\# data frame}
\NormalTok{td }\SpecialCharTok{\%\textgreater{}\%}
    \FunctionTok{group\_by}\NormalTok{(Sex) }\SpecialCharTok{\%\textgreater{}\%}
    \FunctionTok{summarize}\NormalTok{(}\AttributeTok{Mean.YOB =} \FunctionTok{mean}\NormalTok{(YOB)) }\SpecialCharTok{\%\textgreater{}\%}
    \FunctionTok{as.data.frame}\NormalTok{()}
\end{Highlighting}
\end{Shaded}

\begin{verbatim}
  Sex Mean.YOB
1   F     1963
2   M     1977
\end{verbatim}

\begin{Shaded}
\begin{Highlighting}[]
\CommentTok{\# Change number of significant digits displayed}
\CommentTok{\# to 3}
\FunctionTok{options}\NormalTok{(}\AttributeTok{digits =} \DecValTok{3}\NormalTok{)}
\CommentTok{\# Get mean year of birth by sex, converted to}
\CommentTok{\# data frame but showing 10 significant digits}
\NormalTok{td }\SpecialCharTok{\%\textgreater{}\%}
    \FunctionTok{group\_by}\NormalTok{(Sex) }\SpecialCharTok{\%\textgreater{}\%}
    \FunctionTok{summarize}\NormalTok{(}\AttributeTok{Mean.YOB =} \FunctionTok{mean}\NormalTok{(YOB)) }\SpecialCharTok{\%\textgreater{}\%}
    \FunctionTok{as.data.frame}\NormalTok{() }\SpecialCharTok{\%\textgreater{}\%}
    \FunctionTok{format}\NormalTok{(}\AttributeTok{digits =} \DecValTok{10}\NormalTok{)}
\end{Highlighting}
\end{Shaded}

\begin{verbatim}
  Sex    Mean.YOB
1   F 1963.487102
2   M 1976.856604
\end{verbatim}

For very large numbers \emph{R} will often display values in exponential
notation. We can alter this by setting the value of \texttt{scipen}
inside the \texttt{option()} function. Again, though, remember that this
is a global change for your whole \emph{R} session. For \texttt{scipen}
positive values increase the likelihood of using real numbers, negative
values increase the likelihood of using exponential notation. To ensure
printouts are always real numbers, set \texttt{scipen} to 9999 (this is
the default). To ensure printouts are always exponential notation, set
\texttt{scipen} to -9999. To demonstrate, below we multiply mean
\texttt{YOB} by 10000.

\begin{Shaded}
\begin{Highlighting}[]
\CommentTok{\# Change number of significant digits displayed}
\CommentTok{\# to 6, alter the likelihood of use of real}
\CommentTok{\# number rather than scientific notation by 0}
\FunctionTok{options}\NormalTok{(}\AttributeTok{digits =} \DecValTok{6}\NormalTok{, }\AttributeTok{scipen =} \DecValTok{0}\NormalTok{)}
\CommentTok{\# Get mean year of birth by sex multiplied by}
\CommentTok{\# 100000, converted to data frame}
\NormalTok{td }\SpecialCharTok{\%\textgreater{}\%}
    \FunctionTok{group\_by}\NormalTok{(Sex) }\SpecialCharTok{\%\textgreater{}\%}
    \FunctionTok{summarize}\NormalTok{(}\AttributeTok{Mean.YOB =} \FunctionTok{mean}\NormalTok{(YOB) }\SpecialCharTok{*} \FloatTok{1e+05}\NormalTok{) }\SpecialCharTok{\%\textgreater{}\%}
    \FunctionTok{as.data.frame}\NormalTok{()}
\end{Highlighting}
\end{Shaded}

\begin{verbatim}
  Sex  Mean.YOB
1   F 196348710
2   M 197685660
\end{verbatim}

With \texttt{scipen} set to 0, we still get real numbers as the values
\texttt{Mean.YOB} are not too big. To ensure we have real numbers,
though, we change the \texttt{scipen} value.

\begin{Shaded}
\begin{Highlighting}[]
\CommentTok{\# Change number of significant digits displayed}
\CommentTok{\# to 6, alter the likelihood of use of real}
\CommentTok{\# number rather than scientific notation by 9999}
\FunctionTok{options}\NormalTok{(}\AttributeTok{digits =} \DecValTok{6}\NormalTok{, }\AttributeTok{scipen =} \DecValTok{9999}\NormalTok{)}
\CommentTok{\# Get mean year of birth by sex multiplied by}
\CommentTok{\# 100000, converted to data frame}
\NormalTok{td }\SpecialCharTok{\%\textgreater{}\%}
    \FunctionTok{group\_by}\NormalTok{(Sex) }\SpecialCharTok{\%\textgreater{}\%}
    \FunctionTok{summarize}\NormalTok{(}\AttributeTok{Mean.YOB =} \FunctionTok{mean}\NormalTok{(YOB) }\SpecialCharTok{*} \DecValTok{10000}\NormalTok{) }\SpecialCharTok{\%\textgreater{}\%}
    \FunctionTok{as.data.frame}\NormalTok{()}
\end{Highlighting}
\end{Shaded}

\begin{verbatim}
  Sex Mean.YOB
1   F 19634871
2   M 19768566
\end{verbatim}

If, instead we prefer exponential notation, we use the maximum negative
\texttt{scipen} value, -9999/

\begin{Shaded}
\begin{Highlighting}[]
\CommentTok{\# Change number of significant digits displayed}
\CommentTok{\# to 6, alter the likelihood of use of real}
\CommentTok{\# number rather than scientific notation by {-}9999}
\FunctionTok{options}\NormalTok{(}\AttributeTok{digits =} \DecValTok{6}\NormalTok{, }\AttributeTok{scipen =} \SpecialCharTok{{-}}\DecValTok{9999}\NormalTok{)}
\CommentTok{\# Get mean year of birth by sex multiplied by}
\CommentTok{\# 100000, converted to data frame}
\NormalTok{td }\SpecialCharTok{\%\textgreater{}\%}
    \FunctionTok{group\_by}\NormalTok{(Sex) }\SpecialCharTok{\%\textgreater{}\%}
    \FunctionTok{summarize}\NormalTok{(}\AttributeTok{Mean.YOB =} \FunctionTok{mean}\NormalTok{(YOB) }\SpecialCharTok{*} \DecValTok{10000}\NormalTok{) }\SpecialCharTok{\%\textgreater{}\%}
    \FunctionTok{as.data.frame}\NormalTok{()}
\end{Highlighting}
\end{Shaded}

\begin{verbatim}
  Sex    Mean.YOB
1   F 1.96349e+07
2   M 1.97686e+07
\end{verbatim}

Above, the value \texttt{1.96349e+07} means \(1.96349 \times 10^7\). The
easiest way to calculate this is to simply move the decimal places 7
spaces to the right (as the exponent is positive), which gives
\texttt{19634900}. Notice some precision is lost because our number of
\texttt{digits} is only 6.

\begin{Shaded}
\begin{Highlighting}[]
\CommentTok{\# Change number of significant digits displayed}
\CommentTok{\# to 10, alter the likelihood of use of real}
\CommentTok{\# number rather than scientific notation by {-}9999}
\FunctionTok{options}\NormalTok{(}\AttributeTok{digits =} \FloatTok{1e+01}\NormalTok{, }\AttributeTok{scipen =} \SpecialCharTok{{-}}\FloatTok{9.999e+03}\NormalTok{)}
\CommentTok{\# Get mean year of birth by sex multiplied by}
\CommentTok{\# 100000, converted to data frame}
\NormalTok{td }\SpecialCharTok{\%\textgreater{}\%}
    \FunctionTok{group\_by}\NormalTok{(Sex) }\SpecialCharTok{\%\textgreater{}\%}
    \FunctionTok{summarize}\NormalTok{(}\AttributeTok{Mean.YOB =} \FunctionTok{mean}\NormalTok{(YOB) }\SpecialCharTok{*} \FloatTok{1e+04}\NormalTok{) }\SpecialCharTok{\%\textgreater{}\%}
    \FunctionTok{as.data.frame}\NormalTok{()}
\end{Highlighting}
\end{Shaded}

\begin{verbatim}
  Sex        Mean.YOB
1   F 1.963487102e+07
2   M 1.976856604e+07
\end{verbatim}

Now, with more \texttt{digits} we have more precision;
\(1.963487102 \times 10^7 = 19634671.02\). If the exponential values are
negative, move the decimal place to the left. For example,
\(1.963487102 \times 10^-7 = 0.0000001963467102\).

Similarly, we can set whether or not we want scientific notation using
the \texttt{format()} function. The \texttt{scientific} option can be
either \texttt{TRUE} or \texttt{FALSE}, or a value like \texttt{scipen}.

\begin{Shaded}
\begin{Highlighting}[]
\CommentTok{\# Change number of significant digits displayed}
\CommentTok{\# to 3, alter the likelihood of use of real}
\CommentTok{\# number rather than scientific notation by 9999}
\FunctionTok{options}\NormalTok{(}\AttributeTok{digits =} \FloatTok{3e+00}\NormalTok{, }\AttributeTok{scipen =} \FloatTok{9.999e+03}\NormalTok{)}
\CommentTok{\# Get mean year of birth by sex multiplied by}
\CommentTok{\# 100000, converted to data frame, digits}
\CommentTok{\# formatted to 10 significant digits, and}
\CommentTok{\# exponential notation}
\NormalTok{td }\SpecialCharTok{\%\textgreater{}\%}
    \FunctionTok{group\_by}\NormalTok{(Sex) }\SpecialCharTok{\%\textgreater{}\%}
    \FunctionTok{summarize}\NormalTok{(}\AttributeTok{Mean.YOB =} \FunctionTok{mean}\NormalTok{(YOB) }\SpecialCharTok{*} \FloatTok{1e+04}\NormalTok{) }\SpecialCharTok{\%\textgreater{}\%}
    \FunctionTok{as.data.frame}\NormalTok{() }\SpecialCharTok{\%\textgreater{}\%}
    \FunctionTok{format}\NormalTok{(}\AttributeTok{digits =} \FloatTok{1e+01}\NormalTok{, }\AttributeTok{scientific =} \ConstantTok{TRUE}\NormalTok{)}
\end{Highlighting}
\end{Shaded}

\begin{verbatim}
  Sex        Mean.YOB
1   F 1.963487102e+07
2   M 1.976856604e+07
\end{verbatim}

\hypertarget{more-summary-statistics-for-continous-variables}{%
\subsection{More Summary Statistics for Continous
Variables}\label{more-summary-statistics-for-continous-variables}}

The other summary statistics for continuous variables include spread
functions and the range functions. Some spread functions are
\texttt{sd()}, which returns the standard deviation; and \texttt{IQR()}
which returns the interquartile range.\footnote{If we order the data
  from lowest to highest values, 50\% of the data will be less than the
  mean, and 50\% of the data will be higher than the mean. The mean is
  also called the 2nd quartile. The first quartile is halfway between
  the mean and the lowest value in the data. The third quartile is
  halfway betwen the mean and the highest value in the data. The
  interquartile range is the difference between the 3rd quartile and the
  1st quartile and represents the spread of the middle 50\% of the data.}
Some range functions include: \texttt{min()}, which returns the lowest
value; \texttt{max()}, which returns the highest value. To find the
maximum spread (from highest to lowest), we can either subtract the
\texttt{min()} value from the \texttt{max()} value, or employ the
\texttt{diff()} function plus the \texttt{range()} function (which
produces a vector containing the minimum and maximum values).

We can include these functions inside the same \texttt{summarize()}
function as we used above.

\begin{Shaded}
\begin{Highlighting}[]
\CommentTok{\# Get mean, standard deviation, interquartile}
\CommentTok{\# range, minimum value, maximum value, and range}
\CommentTok{\# of values (twice) for year of birth}
\NormalTok{td }\SpecialCharTok{\%\textgreater{}\%}
    \FunctionTok{group\_by}\NormalTok{(Sex) }\SpecialCharTok{\%\textgreater{}\%}
    \FunctionTok{summarize}\NormalTok{(}\AttributeTok{Mean.YOB =} \FunctionTok{mean}\NormalTok{(YOB), }\AttributeTok{SD.YOB =} \FunctionTok{sd}\NormalTok{(YOB),}
        \AttributeTok{IQR.YOB =} \FunctionTok{IQR}\NormalTok{(YOB), }\AttributeTok{Min.YOB =} \FunctionTok{min}\NormalTok{(YOB), }\AttributeTok{Max.YOB =} \FunctionTok{max}\NormalTok{(YOB),}
        \AttributeTok{Range =} \FunctionTok{max}\NormalTok{(YOB) }\SpecialCharTok{{-}} \FunctionTok{min}\NormalTok{(YOB), }\AttributeTok{Range2 =} \FunctionTok{diff}\NormalTok{(}\FunctionTok{range}\NormalTok{(YOB)))}
\end{Highlighting}
\end{Shaded}

\begin{verbatim}
# A tibble: 2 x 8
  Sex   Mean.YOB SD.YOB IQR.YOB Min.YOB Max.YOB Range Range2
  <fct>    <dbl>  <dbl>   <dbl>   <int>   <int> <int>  <int>
1 F        1963.   26.5      45    1915    1999    84     84
2 M        1977.   19.6      33    1921    1994    73     73
\end{verbatim}

Based on these values, we can make the following statements:

\begin{itemize}
\item
  Among females in the (t, d) data, the average or mean year of birth is
  1963 \(\pm\) 26.5 years.
\item
  The oldest female speakers was born in 1915, and the youngest female
  speaker was born in 1999.
\item
  Fifty-percent of women were born in the 45 years centered around 1963.
\item
  The female data represents 84 years of
  \href{https://en.wikipedia.org/wiki/Apparent-time_hypothesis}{apparent
  time}.
\end{itemize}

\hypertarget{position-functions-with-summarize}{%
\subsection{\texorpdfstring{Position functions with
\texttt{summarize()}}{Position functions with summarize()}}\label{position-functions-with-summarize}}

The position functions \texttt{first()}, \texttt{last()}, and
\texttt{nth()} also work on the data created by \texttt{group\_by()} and
\texttt{summarize()}. \texttt{first()} returns the first value,
\texttt{last()} returns the last value, and \texttt{nth()} returns the
value after a specific number of rows.

\begin{Shaded}
\begin{Highlighting}[]
\CommentTok{\# Get first six rows of just Sex and Dep.Var}
\CommentTok{\# columns of td}
\NormalTok{td }\SpecialCharTok{\%\textgreater{}\%}
    \FunctionTok{select}\NormalTok{(Sex, Dep.Var) }\SpecialCharTok{\%\textgreater{}\%}
    \FunctionTok{head}\NormalTok{()}
\end{Highlighting}
\end{Shaded}

\begin{verbatim}
  Sex  Dep.Var
1   F Realized
2   F Deletion
3   F Deletion
4   F Deletion
5   M Realized
6   M Deletion
\end{verbatim}

\begin{Shaded}
\begin{Highlighting}[]
\CommentTok{\# Get last six rows of just Sex and Dep.Var}
\CommentTok{\# columns of td}
\NormalTok{td }\SpecialCharTok{\%\textgreater{}\%}
    \FunctionTok{select}\NormalTok{(Sex, Dep.Var) }\SpecialCharTok{\%\textgreater{}\%}
    \FunctionTok{tail}\NormalTok{()}
\end{Highlighting}
\end{Shaded}

\begin{verbatim}
     Sex  Dep.Var
1184   F Realized
1185   F Realized
1186   F Realized
1187   M Realized
1188   M Deletion
1189   M Realized
\end{verbatim}

Above we use the \texttt{select()} function to choose just the
\texttt{Sex} and \texttt{Dep.Var} columns and run the \texttt{head()}
and \texttt{tail()} functions in order to see the first and last six
values for both in the data. We do this just for comparisons sake. Now,
lets use the position functions an compare them to our results.

\begin{Shaded}
\begin{Highlighting}[]
\CommentTok{\# Get first, last, second, and second to last}
\CommentTok{\# value of Dep.Var by Sex}
\NormalTok{td }\SpecialCharTok{\%\textgreater{}\%}
    \FunctionTok{group\_by}\NormalTok{(Sex) }\SpecialCharTok{\%\textgreater{}\%}
    \FunctionTok{summarize}\NormalTok{(}\AttributeTok{First =} \FunctionTok{first}\NormalTok{(Dep.Var), }\AttributeTok{Last =} \FunctionTok{last}\NormalTok{(Dep.Var),}
        \AttributeTok{Second =} \FunctionTok{nth}\NormalTok{(Dep.Var, }\DecValTok{2}\NormalTok{), }\AttributeTok{Second.Last =} \FunctionTok{nth}\NormalTok{(Dep.Var,}
            \SpecialCharTok{{-}}\DecValTok{2}\NormalTok{))}
\end{Highlighting}
\end{Shaded}

\begin{verbatim}
# A tibble: 2 x 5
  Sex   First    Last     Second   Second.Last
  <fct> <fct>    <fct>    <fct>    <fct>      
1 F     Realized Realized Deletion Realized   
2 M     Realized Realized Deletion Deletion   
\end{verbatim}

Compare the male values with those from the \texttt{head()} and
\texttt{tail()} functions above. The first (row 5) is \texttt{Realized},
the last (row 1198) is \texttt{Realized}. The second (row 6) is
\texttt{Deletion}, and the second to last (row 1188) is also
\texttt{Deletion}.

\hypertarget{count-functions-with-summarize}{%
\subsubsection{\texorpdfstring{Count functions with
\texttt{summarize()}}{Count functions with summarize()}}\label{count-functions-with-summarize}}

We've already looked at \texttt{n()} above, but there is also the
\texttt{n\_distinct()} function, which reports the number of distinct
values. We can use this, for example, to find the number of speakers in
each social category. To do this using base \emph{R} filtering is a lot
more complicated to code (so much so its not even worth doing). One
example is shown below. It would need to be repeated for every
combination of sex, education, and age group.

\begin{Shaded}
\begin{Highlighting}[]
\CommentTok{\# Example using base R filtering, finding the}
\CommentTok{\# number of unique speakers who are female,}
\CommentTok{\# educated, and middle aged}

\FunctionTok{n\_distinct}\NormalTok{(td}\SpecialCharTok{$}\NormalTok{Speaker[td}\SpecialCharTok{$}\NormalTok{Sex }\SpecialCharTok{==} \StringTok{"F"} \SpecialCharTok{\&}\NormalTok{ td}\SpecialCharTok{$}\NormalTok{Education }\SpecialCharTok{==}
    \StringTok{"Educated"} \SpecialCharTok{\&}\NormalTok{ td}\SpecialCharTok{$}\NormalTok{Age.Group }\SpecialCharTok{==} \StringTok{"Middle"}\NormalTok{])}
\end{Highlighting}
\end{Shaded}

\begin{verbatim}
[1] 12
\end{verbatim}

\begin{Shaded}
\begin{Highlighting}[]
\CommentTok{\# Much easier way to find number of unique}
\CommentTok{\# speakers for every combination of Sex,}
\CommentTok{\# Education, and Age. Group}

\NormalTok{td }\SpecialCharTok{\%\textgreater{}\%}
    \FunctionTok{group\_by}\NormalTok{(Sex, Education, Age.Group) }\SpecialCharTok{\%\textgreater{}\%}
    \FunctionTok{summarize}\NormalTok{(}\AttributeTok{Speaker.Count =} \FunctionTok{n\_distinct}\NormalTok{(Speaker)) }\SpecialCharTok{\%\textgreater{}\%}
    \FunctionTok{print}\NormalTok{(}\AttributeTok{n =} \ConstantTok{Inf}\NormalTok{)}
\end{Highlighting}
\end{Shaded}

\begin{verbatim}
# A tibble: 12 x 4
# Groups:   Sex, Education [6]
   Sex   Education    Age.Group Speaker.Count
   <fct> <fct>        <fct>             <int>
 1 F     Educated     Old                   1
 2 F     Educated     Middle               12
 3 F     Educated     Young                 3
 4 F     Not Educated Old                   6
 5 F     Not Educated Middle                1
 6 F     Student      Young                11
 7 M     Educated     Middle                3
 8 M     Educated     Young                 6
 9 M     Not Educated Old                   5
10 M     Not Educated Middle                7
11 M     Not Educated Young                 3
12 M     Student      Young                 8
\end{verbatim}

You'll notice that there are is no value for older educated males. This
is because there are no speakers in the data from this group.

\hypertarget{logical-functions}{%
\subsubsection{Logical functions}\label{logical-functions}}

The two logical functions only work on data that is logical (i.e., is
\texttt{TRUE} or \texttt{FALSE}). \texttt{any()} returns the answer to
the question ``Are any values \texttt{TRUE}?'' and \texttt{all()}
returns the answer to the question ``Are all values \texttt{TRUE}?''.
There are no logical values in the \texttt{td} data set, so lets make
some as an example.

\begin{Shaded}
\begin{Highlighting}[]
\CommentTok{\# Create a new column in which all values are}
\CommentTok{\# FALSE}
\NormalTok{td}\SpecialCharTok{$}\NormalTok{Logical.Test }\OtherTok{\textless{}{-}} \ConstantTok{FALSE}
\CommentTok{\# Modify the new column so for any tokens from}
\CommentTok{\# young female speakers are coded as TRUE instead}
\CommentTok{\# of FALSE}
\NormalTok{td}\SpecialCharTok{$}\NormalTok{Logical.Test[td}\SpecialCharTok{$}\NormalTok{Sex }\SpecialCharTok{==} \StringTok{"F"} \SpecialCharTok{\&}\NormalTok{ td}\SpecialCharTok{$}\NormalTok{Age.Group }\SpecialCharTok{==} \StringTok{"Young"}\NormalTok{] }\OtherTok{\textless{}{-}} \ConstantTok{TRUE}

\CommentTok{\# Get logical value (TRUE or FALSE) of whether}
\CommentTok{\# any tokens and all tokens of Logical.Test are}
\CommentTok{\# TRUE, by Sex}
\NormalTok{td }\SpecialCharTok{\%\textgreater{}\%}
    \FunctionTok{group\_by}\NormalTok{(Sex) }\SpecialCharTok{\%\textgreater{}\%}
    \FunctionTok{summarize}\NormalTok{(}\AttributeTok{Any.True =} \FunctionTok{any}\NormalTok{(Logical.Test), }\AttributeTok{All.True =} \FunctionTok{all}\NormalTok{(Logical.Test))}
\end{Highlighting}
\end{Shaded}

\begin{verbatim}
# A tibble: 2 x 3
  Sex   Any.True All.True
  <fct> <lgl>    <lgl>   
1 F     TRUE     FALSE   
2 M     FALSE    FALSE   
\end{verbatim}

Above we created a logical column in which only tokens from young
females are set to \texttt{TRUE}. The \texttt{any()} function returns
\texttt{TRUE} for \texttt{F} but not for \texttt{M} because there is at
least one \texttt{TRUE} value in the female data. Conversely, the
\texttt{all()} function returns \texttt{FALSE} for \texttt{F} because
not all of the female values are \texttt{TRUE}.

\hypertarget{proportions}{%
\subsection{Proportions}\label{proportions}}

Finding out the proportion of a variant is just like finding out the
number of tokens. Using the base \emph{R} methods, you simply wrap the
\texttt{table()} function in a \texttt{prop.table()} function.

\begin{Shaded}
\begin{Highlighting}[]
\CommentTok{\# Proportion of each level of Dep.Var}
\FunctionTok{prop.table}\NormalTok{(}\FunctionTok{table}\NormalTok{(td}\SpecialCharTok{$}\NormalTok{Dep.Var))}
\end{Highlighting}
\end{Shaded}

\begin{verbatim}

Deletion Realized 
   0.325    0.675 
\end{verbatim}

Usually proportions are expressed as hundredths. To force \emph{R} to
express numbers in hundredths, you can use the \texttt{options()}
function to set the number of significant digits displayed to two.

\begin{Shaded}
\begin{Highlighting}[]
\CommentTok{\# Display values rounded to nearest hundredth.}
\FunctionTok{options}\NormalTok{(}\AttributeTok{digits =} \DecValTok{2}\NormalTok{)}

\CommentTok{\# Proportion of each level of Dep.Var}
\FunctionTok{prop.table}\NormalTok{(}\FunctionTok{table}\NormalTok{(td}\SpecialCharTok{$}\NormalTok{Dep.Var))}
\end{Highlighting}
\end{Shaded}

\begin{verbatim}

Deletion Realized 
    0.32     0.68 
\end{verbatim}

In the example above there is only one dimension: \texttt{Dep.Var}. The
\texttt{prop.table()} outer function takes the \texttt{table()} inner
function and divides the number of tokens in each cell by some total
(e.g.~denominator). The default denominator is the total number of
tokens in the whole table. Because, in the example above, the total
number of tokens in the one dimension table is the same as the total
number of \texttt{Dep.Var} tokens, you don't need to specify anything
further. In the example below, however, there are two dimensions:
\texttt{Dep.Var} and \texttt{Age.Group}. If you do not specify which
total to use as a denominator, the proportions expressed use the total
number of tokens in the table as the denominator.\footnote{You'll notice
  that the values in this table are expressed in thousandths instead of
  hundredths. This is because the proportion for \texttt{Deletion} and
  \texttt{Old} tokens requires three decimal places to have two
  meaningful digits.} If you want to know the percentage of deletion
tokens that come from \texttt{Young}, \texttt{Middle} and \texttt{Old}
speakers, you set \texttt{margin\ =\ 1}, meaning that you want the total
(e.g., denominator) to be the sum of the tokens for the first variable
in the function, (e.g., rows total). If instead you want to know the
percentage of \texttt{Young} tokens (or \texttt{Middle} tokens, or
\texttt{Old} tokens) that are \texttt{Deletion}, and the percentage that
are \texttt{Realized}, you set \texttt{margin\ =\ 2}, or rather set the
denominator to the sum of the second factor group in the function (e.g.,
column total). This follows \emph{R}'s global pattern of rows, columns,
page 1, page 2, etc. You can verify this by adding up the proportions in
each table below. In the first table all of the proportions add up to 1.
In the second table, on the other hand, the proportions add up to 1
going across the rows. In the third table they add up to 1 going down
the columns.

\begin{Shaded}
\begin{Highlighting}[]
\CommentTok{\# Proportion of each level of Dep.Var and}
\CommentTok{\# Age.Group (all values sum to 1)}
\FunctionTok{prop.table}\NormalTok{(}\FunctionTok{table}\NormalTok{(td}\SpecialCharTok{$}\NormalTok{Dep.Var, td}\SpecialCharTok{$}\NormalTok{Age.Group))}
\end{Highlighting}
\end{Shaded}

\begin{verbatim}
          
             Old Middle Young
  Deletion 0.056  0.105 0.163
  Realized 0.113  0.198 0.365
\end{verbatim}

\begin{Shaded}
\begin{Highlighting}[]
\CommentTok{\# Proportion of each level of Age.Group for each}
\CommentTok{\# level of Dep.Var (each row sums to 1)}
\FunctionTok{prop.table}\NormalTok{(}\FunctionTok{table}\NormalTok{(td}\SpecialCharTok{$}\NormalTok{Dep.Var, td}\SpecialCharTok{$}\NormalTok{Age.Group), }\AttributeTok{margin =} \DecValTok{1}\NormalTok{)}
\end{Highlighting}
\end{Shaded}

\begin{verbatim}
          
            Old Middle Young
  Deletion 0.17   0.32  0.50
  Realized 0.17   0.29  0.54
\end{verbatim}

\begin{Shaded}
\begin{Highlighting}[]
\CommentTok{\# Proportion of each level of Dep.Var for each}
\CommentTok{\# level of Age.Group (each column sums to 1)}
\FunctionTok{prop.table}\NormalTok{(}\FunctionTok{table}\NormalTok{(td}\SpecialCharTok{$}\NormalTok{Dep.Var, td}\SpecialCharTok{$}\NormalTok{Age.Group), }\AttributeTok{margin =} \DecValTok{2}\NormalTok{)}
\end{Highlighting}
\end{Shaded}

\begin{verbatim}
          
            Old Middle Young
  Deletion 0.33   0.35  0.31
  Realized 0.67   0.65  0.69
\end{verbatim}

In order to achieve the three-dimension cross-tabs you get from
\emph{Goldvarb}, with one dependent variable and two independent
variables, you must set up the \texttt{prop.table(table())} function
with your variables in the following order: \emph{independent variable
1}, \emph{independent variable 2}, \emph{dependent variable}. You must
also specify a particular \texttt{margin}, e.g., denominator. In a
\emph{Goldvarb}-style cross-tab each cell is the number of tokens for
one level of the dependent variable (e.g., the application or
non-application value) divided by the total number of tokens for that
cell. In an \emph{R} proportion table the total number of tokens per
cell is the number of tokens for the value of the row and the column at
the same time --- not the row total, or the column total. To specify
that you want the denominator to be the cell total you set \emph{margin
= c(1,2)}, where the \texttt{c()} concatenating function specifies both
row (1) and column (2). The result is a separate page for proportions of
each level of \texttt{Dep.Var}. The proportions for the corresponding
cells in each page add up to 1.

\begin{Shaded}
\begin{Highlighting}[]
\CommentTok{\# Proportion of each level of Dep.Var for each}
\CommentTok{\# level of Age.Group and Sex (all corresponding}
\CommentTok{\# cells sum to 1)}
\FunctionTok{prop.table}\NormalTok{(}\FunctionTok{table}\NormalTok{(td}\SpecialCharTok{$}\NormalTok{Age.Group, td}\SpecialCharTok{$}\NormalTok{Sex, td}\SpecialCharTok{$}\NormalTok{Dep.Var),}
    \AttributeTok{margin =} \FunctionTok{c}\NormalTok{(}\DecValTok{1}\NormalTok{, }\DecValTok{2}\NormalTok{))}
\end{Highlighting}
\end{Shaded}

\begin{verbatim}
, ,  = Deletion

        
            F    M
  Old    0.29 0.47
  Middle 0.31 0.43
  Young  0.27 0.34

, ,  = Realized

        
            F    M
  Old    0.71 0.53
  Middle 0.69 0.57
  Young  0.73 0.66
\end{verbatim}

You can keep adding factor groups to your proportion table, but you must
do two things. You must keep the dependent variable, \texttt{Dep.Var},
as the rightmost variable in the function, and you must include all the
other variables in the margin specification. For example, below you add
\texttt{Education} as the third variable, and add 3 to the margin
specification. There will be a separate page for each combination of the
levels of \texttt{Education} and \texttt{Dep.Var}.

\begin{Shaded}
\begin{Highlighting}[]
\CommentTok{\# Proportion of each level of Dep.Var for each}
\CommentTok{\# level of Age.Group, Sex and Education}
\FunctionTok{prop.table}\NormalTok{(}\FunctionTok{table}\NormalTok{(td}\SpecialCharTok{$}\NormalTok{Age.Group, td}\SpecialCharTok{$}\NormalTok{Sex, td}\SpecialCharTok{$}\NormalTok{Education,}
\NormalTok{    td}\SpecialCharTok{$}\NormalTok{Dep.Var), }\AttributeTok{margin =} \FunctionTok{c}\NormalTok{(}\DecValTok{1}\NormalTok{, }\DecValTok{2}\NormalTok{, }\DecValTok{3}\NormalTok{))}
\end{Highlighting}
\end{Shaded}

\begin{verbatim}
, ,  = Educated,  = Deletion

        
             F     M
  Old    0.062      
  Middle 0.308 0.348
  Young  0.278 0.384

, ,  = Not Educated,  = Deletion

        
             F     M
  Old    0.347 0.471
  Middle 0.294 0.474
  Young        0.436

, ,  = Student,  = Deletion

        
             F     M
  Old               
  Middle            
  Young  0.261 0.282

, ,  = Educated,  = Realized

        
             F     M
  Old    0.938      
  Middle 0.692 0.652
  Young  0.722 0.616

, ,  = Not Educated,  = Realized

        
             F     M
  Old    0.653 0.529
  Middle 0.706 0.526
  Young        0.564

, ,  = Student,  = Realized

        
             F     M
  Old               
  Middle            
  Young  0.739 0.718
\end{verbatim}

Again, you can make these larger tables easier to read by flattening the
pages using \texttt{ftable()}. Here the \texttt{NaN} means there is no
data in the cell.

\begin{Shaded}
\begin{Highlighting}[]
\CommentTok{\# Proportion of each level of Dep.Var for each}
\CommentTok{\# level of Age.Group, Sex and Education,}
\CommentTok{\# presented as a flattened table. Here the \textasciigrave{}NaN\textquotesingle{}}
\CommentTok{\# just means there is no data in the cell.}
\FunctionTok{library}\NormalTok{(vcd)}
\FunctionTok{ftable}\NormalTok{(}\FunctionTok{prop.table}\NormalTok{(}\FunctionTok{table}\NormalTok{(td}\SpecialCharTok{$}\NormalTok{Age.Group, td}\SpecialCharTok{$}\NormalTok{Sex, td}\SpecialCharTok{$}\NormalTok{Education,}
\NormalTok{    td}\SpecialCharTok{$}\NormalTok{Dep.Var), }\AttributeTok{margin =} \FunctionTok{c}\NormalTok{(}\DecValTok{1}\NormalTok{, }\DecValTok{2}\NormalTok{, }\DecValTok{3}\NormalTok{)))}
\end{Highlighting}
\end{Shaded}

\begin{verbatim}
                       Deletion Realized
                                        
Old    F Educated         0.062    0.938
         Not Educated     0.347    0.653
         Student            NaN      NaN
       M Educated           NaN      NaN
         Not Educated     0.471    0.529
         Student            NaN      NaN
Middle F Educated         0.308    0.692
         Not Educated     0.294    0.706
         Student            NaN      NaN
       M Educated         0.348    0.652
         Not Educated     0.474    0.526
         Student            NaN      NaN
Young  F Educated         0.278    0.722
         Not Educated       NaN      NaN
         Student          0.261    0.739
       M Educated         0.384    0.616
         Not Educated     0.436    0.564
         Student          0.282    0.718
\end{verbatim}

There are a number of functions specifically designed to create
cross-tables that are somewhat easier to use, but can be somewhat less
flexible. Generally, they are most useful for one independent variable
and one dependent variable. I tend to use the \texttt{CrossTable()}
function from the \texttt{gmodels} package frequently.

\begin{Shaded}
\begin{Highlighting}[]
\CommentTok{\# Load gmodels}
\FunctionTok{library}\NormalTok{(gmodels)}

\CommentTok{\# Generate cross tab of Sex and Dep.Var in which}
\CommentTok{\# the row proportions are displayed, but table}
\CommentTok{\# proportions, column proportions, and}
\CommentTok{\# contribution to chi{-}square are suppressed, with}
\CommentTok{\# 0 decimal values displayed, and missing}
\CommentTok{\# combinations included.}
\FunctionTok{CrossTable}\NormalTok{(td}\SpecialCharTok{$}\NormalTok{Sex, td}\SpecialCharTok{$}\NormalTok{Dep.Var, }\AttributeTok{prop.r =} \ConstantTok{TRUE}\NormalTok{, }\AttributeTok{prop.c =} \ConstantTok{FALSE}\NormalTok{,}
    \AttributeTok{prop.t =} \ConstantTok{FALSE}\NormalTok{, }\AttributeTok{prop.chisq =} \ConstantTok{FALSE}\NormalTok{, }\AttributeTok{format =} \StringTok{"SPSS"}\NormalTok{,}
    \AttributeTok{digits =} \DecValTok{0}\NormalTok{, }\AttributeTok{missing.include =} \ConstantTok{TRUE}\NormalTok{)}
\end{Highlighting}
\end{Shaded}

\begin{verbatim}

   Cell Contents
|-------------------------|
|                   Count |
|             Row Percent |
|-------------------------|

Total Observations in Table:  1189 

             | td$Dep.Var 
      td$Sex | Deletion  | Realized  | Row Total | 
-------------|-----------|-----------|-----------|
           F |      188  |      471  |      659  | 
             |       29% |       71% |       55% | 
-------------|-----------|-----------|-----------|
           M |      198  |      332  |      530  | 
             |       37% |       63% |       45% | 
-------------|-----------|-----------|-----------|
Column Total |      386  |      803  |     1189  | 
-------------|-----------|-----------|-----------|

 
\end{verbatim}

For the \texttt{CrossTable()} function you can set the denominator to
row total with the option \texttt{prop.r=TRUE}. If instead you wanted to
the proportion by column, you set \texttt{prop.c\ =\ TRUE}, and if you
want the proportion across the entire table you can set
\texttt{prop.t\ =\ TRUE}. You can actually set all of these to
\texttt{TRUE} to get all three. There are other values that can be
generated, including values for calculating chi-square (see the
\texttt{CrossTable()} documentation
\href{https://www.rdocumentation.org/packages/gmodels/versions/2.18.1.1/topics/CrossTable}{here}).
The above code includes the minimal number of options needed to generate
the type of cross-table we generally want.

To produce proportions using the \texttt{tidy} method, we combine the
\texttt{group\_by()} and \texttt{summarize()} functions with the
\texttt{mutate()} discussed in an
\href{https://lingmethodshub.github.io/content/R/lvc_r/040_lvcr.html}{earlier
section}.

\begin{Shaded}
\begin{Highlighting}[]
\CommentTok{\# Generate tibble of combination of Sex and}
\CommentTok{\# Dep.Var with token counts and proportion of}
\CommentTok{\# each level of Dep.Var by Sex}
\NormalTok{td }\SpecialCharTok{\%\textgreater{}\%}
    \FunctionTok{group\_by}\NormalTok{(Sex, Dep.Var) }\SpecialCharTok{\%\textgreater{}\%}
    \FunctionTok{summarize}\NormalTok{(}\AttributeTok{Count =} \FunctionTok{n}\NormalTok{()) }\SpecialCharTok{\%\textgreater{}\%}
    \FunctionTok{mutate}\NormalTok{(}\AttributeTok{Prop =}\NormalTok{ Count}\SpecialCharTok{/}\FunctionTok{sum}\NormalTok{(Count))}
\end{Highlighting}
\end{Shaded}

\begin{verbatim}
# A tibble: 4 x 4
# Groups:   Sex [2]
  Sex   Dep.Var  Count  Prop
  <fct> <fct>    <int> <dbl>
1 F     Deletion   188 0.285
2 F     Realized   471 0.715
3 M     Deletion   198 0.374
4 M     Realized   332 0.626
\end{verbatim}

After grouping the data by \texttt{Sex} and \texttt{Dep.Var}, we create
a new column \texttt{Count} with values equal to the number of tokens
for the particular combination, then we create a new column using
\texttt{mutate()} and a math equation to generate proportions. It is
important here that your dependent variable \texttt{Dep.Var} is the last
grouping variable. If we change the order, instead of generating the
proportion of \texttt{Realized} and \texttt{Deletion} tokens, it will
instead return the percentage of \texttt{Realized} tokens that are
\texttt{M} and the percentage that are \texttt{F}, which is the
incorrect denominator for our purposes.

\begin{Shaded}
\begin{Highlighting}[]
\CommentTok{\# Generate tibble of combination of Dep.Var and}
\CommentTok{\# Sex with token counts and proportion of each}
\CommentTok{\# level of Sex by Dep.Var}
\NormalTok{td }\SpecialCharTok{\%\textgreater{}\%}
    \FunctionTok{group\_by}\NormalTok{(Dep.Var, Sex) }\SpecialCharTok{\%\textgreater{}\%}
    \FunctionTok{summarize}\NormalTok{(}\AttributeTok{Count =} \FunctionTok{n}\NormalTok{()) }\SpecialCharTok{\%\textgreater{}\%}
    \FunctionTok{mutate}\NormalTok{(}\AttributeTok{Prop =}\NormalTok{ Count}\SpecialCharTok{/}\FunctionTok{sum}\NormalTok{(Count))}
\end{Highlighting}
\end{Shaded}

\begin{verbatim}
# A tibble: 4 x 4
# Groups:   Dep.Var [2]
  Dep.Var  Sex   Count  Prop
  <fct>    <fct> <int> <dbl>
1 Deletion F       188 0.487
2 Deletion M       198 0.513
3 Realized F       471 0.587
4 Realized M       332 0.413
\end{verbatim}

Unlike the \texttt{CrossTable()} function, we can include multiple
independent variables. To include every combination (including those for
which there are no tokens), we can add \texttt{.drop\ =\ FALSE} to the
\texttt{group\_by()} function.

\begin{Shaded}
\begin{Highlighting}[]
\CommentTok{\# Generate tibble of combination of Sex,}
\CommentTok{\# Edcuation, Age.Group, and Dep.Var with all}
\CommentTok{\# combinations included, with token counts and}
\CommentTok{\# proportion of each level of Dep.Var by each}
\CommentTok{\# combination of other variables}
\NormalTok{td }\SpecialCharTok{\%\textgreater{}\%}
    \FunctionTok{group\_by}\NormalTok{(Sex, Education, Age.Group, Dep.Var, }\AttributeTok{.drop =} \ConstantTok{FALSE}\NormalTok{) }\SpecialCharTok{\%\textgreater{}\%}
    \FunctionTok{summarize}\NormalTok{(}\AttributeTok{Count =} \FunctionTok{n}\NormalTok{()) }\SpecialCharTok{\%\textgreater{}\%}
    \FunctionTok{mutate}\NormalTok{(}\AttributeTok{Prop =}\NormalTok{ Count}\SpecialCharTok{/}\FunctionTok{sum}\NormalTok{(Count)) }\SpecialCharTok{\%\textgreater{}\%}
    \FunctionTok{print}\NormalTok{(}\AttributeTok{n =} \ConstantTok{Inf}\NormalTok{)}
\end{Highlighting}
\end{Shaded}

\begin{verbatim}
# A tibble: 36 x 6
# Groups:   Sex, Education, Age.Group [18]
   Sex   Education    Age.Group Dep.Var  Count     Prop
   <fct> <fct>        <fct>     <fct>    <int>    <dbl>
 1 F     Educated     Old       Deletion     2   0.0625
 2 F     Educated     Old       Realized    30   0.938 
 3 F     Educated     Middle    Deletion    68   0.308 
 4 F     Educated     Middle    Realized   153   0.692 
 5 F     Educated     Young     Deletion    20   0.278 
 6 F     Educated     Young     Realized    52   0.722 
 7 F     Not Educated Old       Deletion    41   0.347 
 8 F     Not Educated Old       Realized    77   0.653 
 9 F     Not Educated Middle    Deletion     5   0.294 
10 F     Not Educated Middle    Realized    12   0.706 
11 F     Not Educated Young     Deletion     0 NaN     
12 F     Not Educated Young     Realized     0 NaN     
13 F     Student      Old       Deletion     0 NaN     
14 F     Student      Old       Realized     0 NaN     
15 F     Student      Middle    Deletion     0 NaN     
16 F     Student      Middle    Realized     0 NaN     
17 F     Student      Young     Deletion    52   0.261 
18 F     Student      Young     Realized   147   0.739 
19 M     Educated     Old       Deletion     0 NaN     
20 M     Educated     Old       Realized     0 NaN     
21 M     Educated     Middle    Deletion    16   0.348 
22 M     Educated     Middle    Realized    30   0.652 
23 M     Educated     Young     Deletion    48   0.384 
24 M     Educated     Young     Realized    77   0.616 
25 M     Not Educated Old       Deletion    24   0.471 
26 M     Not Educated Old       Realized    27   0.529 
27 M     Not Educated Middle    Deletion    36   0.474 
28 M     Not Educated Middle    Realized    40   0.526 
29 M     Not Educated Young     Deletion    24   0.436 
30 M     Not Educated Young     Realized    31   0.564 
31 M     Student      Old       Deletion     0 NaN     
32 M     Student      Old       Realized     0 NaN     
33 M     Student      Middle    Deletion     0 NaN     
34 M     Student      Middle    Realized     0 NaN     
35 M     Student      Young     Deletion    50   0.282 
36 M     Student      Young     Realized   127   0.718 
\end{verbatim}

Notice that for the missing combinations the \texttt{count()} is 0, and
the percentage is \texttt{NaN}, which stands for ``not a number'', the
result of trying to divide 0 by something. \texttt{NaN} is similar to
\texttt{NA}, but \texttt{NA} stands for ``no data'', and is used for
empty cells.

\begin{Shaded}
\begin{Highlighting}[]
\CommentTok{\# Assign the tibble generated in the previous}
\CommentTok{\# code to an object called results}
\NormalTok{results }\OtherTok{\textless{}{-}}\NormalTok{ td }\SpecialCharTok{\%\textgreater{}\%}
    \FunctionTok{group\_by}\NormalTok{(Sex, Education, Age.Group, Dep.Var, }\AttributeTok{.drop =} \ConstantTok{FALSE}\NormalTok{) }\SpecialCharTok{\%\textgreater{}\%}
    \FunctionTok{summarize}\NormalTok{(}\AttributeTok{Count =} \FunctionTok{n}\NormalTok{()) }\SpecialCharTok{\%\textgreater{}\%}
    \FunctionTok{mutate}\NormalTok{(}\AttributeTok{Prop =}\NormalTok{ Count}\SpecialCharTok{/}\FunctionTok{sum}\NormalTok{(Count))}

\CommentTok{\# Recode all NaN in results to 0}
\NormalTok{results}\SpecialCharTok{$}\NormalTok{Prop[}\FunctionTok{is.nan}\NormalTok{(results}\SpecialCharTok{$}\NormalTok{Prop)] }\OtherTok{\textless{}{-}} \DecValTok{0}
\CommentTok{\# Print results}
\FunctionTok{print}\NormalTok{(results, }\AttributeTok{n =} \ConstantTok{Inf}\NormalTok{)}
\end{Highlighting}
\end{Shaded}

\begin{verbatim}
# A tibble: 36 x 6
# Groups:   Sex, Education, Age.Group [18]
   Sex   Education    Age.Group Dep.Var  Count   Prop
   <fct> <fct>        <fct>     <fct>    <int>  <dbl>
 1 F     Educated     Old       Deletion     2 0.0625
 2 F     Educated     Old       Realized    30 0.938 
 3 F     Educated     Middle    Deletion    68 0.308 
 4 F     Educated     Middle    Realized   153 0.692 
 5 F     Educated     Young     Deletion    20 0.278 
 6 F     Educated     Young     Realized    52 0.722 
 7 F     Not Educated Old       Deletion    41 0.347 
 8 F     Not Educated Old       Realized    77 0.653 
 9 F     Not Educated Middle    Deletion     5 0.294 
10 F     Not Educated Middle    Realized    12 0.706 
11 F     Not Educated Young     Deletion     0 0     
12 F     Not Educated Young     Realized     0 0     
13 F     Student      Old       Deletion     0 0     
14 F     Student      Old       Realized     0 0     
15 F     Student      Middle    Deletion     0 0     
16 F     Student      Middle    Realized     0 0     
17 F     Student      Young     Deletion    52 0.261 
18 F     Student      Young     Realized   147 0.739 
19 M     Educated     Old       Deletion     0 0     
20 M     Educated     Old       Realized     0 0     
21 M     Educated     Middle    Deletion    16 0.348 
22 M     Educated     Middle    Realized    30 0.652 
23 M     Educated     Young     Deletion    48 0.384 
24 M     Educated     Young     Realized    77 0.616 
25 M     Not Educated Old       Deletion    24 0.471 
26 M     Not Educated Old       Realized    27 0.529 
27 M     Not Educated Middle    Deletion    36 0.474 
28 M     Not Educated Middle    Realized    40 0.526 
29 M     Not Educated Young     Deletion    24 0.436 
30 M     Not Educated Young     Realized    31 0.564 
31 M     Student      Old       Deletion     0 0     
32 M     Student      Old       Realized     0 0     
33 M     Student      Middle    Deletion     0 0     
34 M     Student      Middle    Realized     0 0     
35 M     Student      Young     Deletion    50 0.282 
36 M     Student      Young     Realized   127 0.718 
\end{verbatim}

The easiest way to convert \texttt{NaN} (or \texttt{Na}) to 0 is to
assign the above to a variable, then replace \texttt{NaN} with 0 using
the function \texttt{is.nan()}. If there were \texttt{NA} values, you
can do the same thing as above, but replace \texttt{is.nan()} with
\texttt{is.na()}

When we report proportions in sociolinguistics manuscripts, we often
only report the proportion of one level of the dependent variable
(called the application value). To only display one of the two levels of
\texttt{Dep.Var} --- for instance, if we just want to show the rates of
\texttt{Deletion}, which we might decide is our application value --- we
can use the \texttt{subset()} function.

\begin{Shaded}
\begin{Highlighting}[]
\CommentTok{\# Create the results object, but subsetted to}
\CommentTok{\# include only Deletion tokens}
\NormalTok{results }\OtherTok{\textless{}{-}}\NormalTok{ td }\SpecialCharTok{\%\textgreater{}\%}
    \FunctionTok{group\_by}\NormalTok{(Sex, Education, Age.Group, Dep.Var, }\AttributeTok{.drop =} \ConstantTok{FALSE}\NormalTok{) }\SpecialCharTok{\%\textgreater{}\%}
    \FunctionTok{summarize}\NormalTok{(}\AttributeTok{Count =} \FunctionTok{n}\NormalTok{()) }\SpecialCharTok{\%\textgreater{}\%}
    \FunctionTok{mutate}\NormalTok{(}\AttributeTok{Prop =}\NormalTok{ Count}\SpecialCharTok{/}\FunctionTok{sum}\NormalTok{(Count)) }\SpecialCharTok{\%\textgreater{}\%}
    \FunctionTok{subset}\NormalTok{(Dep.Var }\SpecialCharTok{==} \StringTok{"Deletion"}\NormalTok{)}

\CommentTok{\# Recode NaN to 0}
\NormalTok{results}\SpecialCharTok{$}\NormalTok{Prop[}\FunctionTok{is.nan}\NormalTok{(results}\SpecialCharTok{$}\NormalTok{Prop)] }\OtherTok{\textless{}{-}} \DecValTok{0}
\CommentTok{\# Print results}
\FunctionTok{print}\NormalTok{(results, }\AttributeTok{n =} \ConstantTok{Inf}\NormalTok{)}
\end{Highlighting}
\end{Shaded}

\begin{verbatim}
# A tibble: 18 x 6
# Groups:   Sex, Education, Age.Group [18]
   Sex   Education    Age.Group Dep.Var  Count   Prop
   <fct> <fct>        <fct>     <fct>    <int>  <dbl>
 1 F     Educated     Old       Deletion     2 0.0625
 2 F     Educated     Middle    Deletion    68 0.308 
 3 F     Educated     Young     Deletion    20 0.278 
 4 F     Not Educated Old       Deletion    41 0.347 
 5 F     Not Educated Middle    Deletion     5 0.294 
 6 F     Not Educated Young     Deletion     0 0     
 7 F     Student      Old       Deletion     0 0     
 8 F     Student      Middle    Deletion     0 0     
 9 F     Student      Young     Deletion    52 0.261 
10 M     Educated     Old       Deletion     0 0     
11 M     Educated     Middle    Deletion    16 0.348 
12 M     Educated     Young     Deletion    48 0.384 
13 M     Not Educated Old       Deletion    24 0.471 
14 M     Not Educated Middle    Deletion    36 0.474 
15 M     Not Educated Young     Deletion    24 0.436 
16 M     Student      Old       Deletion     0 0     
17 M     Student      Middle    Deletion     0 0     
18 M     Student      Young     Deletion    50 0.282 
\end{verbatim}

Finally, if we also want to add the total number of tokens per category
(something we usually report alongside the application value) we can add
another column using \texttt{mutate()}. Also, if we want the percentage
instead of proportion, we can add \texttt{100\ *} to the proportion
equation (as percentage is proportion \(\times 100\))

\begin{Shaded}
\begin{Highlighting}[]
\CommentTok{\# Generate results object with percentage instead}
\CommentTok{\# of proportion and a column with total tokens}
\CommentTok{\# per combination.}
\NormalTok{results }\OtherTok{\textless{}{-}}\NormalTok{ td }\SpecialCharTok{\%\textgreater{}\%}
    \FunctionTok{group\_by}\NormalTok{(Sex, Education, Age.Group, Dep.Var, }\AttributeTok{.drop =} \ConstantTok{FALSE}\NormalTok{) }\SpecialCharTok{\%\textgreater{}\%}
    \FunctionTok{summarize}\NormalTok{(}\AttributeTok{Count =} \FunctionTok{n}\NormalTok{()) }\SpecialCharTok{\%\textgreater{}\%}
    \FunctionTok{mutate}\NormalTok{(}\AttributeTok{Percentage =} \DecValTok{100} \SpecialCharTok{*}\NormalTok{ Count}\SpecialCharTok{/}\FunctionTok{sum}\NormalTok{(Count), }\AttributeTok{Total.N =} \FunctionTok{sum}\NormalTok{(Count)) }\SpecialCharTok{\%\textgreater{}\%}
    \FunctionTok{subset}\NormalTok{(Dep.Var }\SpecialCharTok{==} \StringTok{"Deletion"}\NormalTok{)}

\CommentTok{\# Recode NaN to 0}
\NormalTok{results}\SpecialCharTok{$}\NormalTok{Percentage[}\FunctionTok{is.nan}\NormalTok{(results}\SpecialCharTok{$}\NormalTok{Percentage)] }\OtherTok{\textless{}{-}} \DecValTok{0}
\CommentTok{\# Print results}
\FunctionTok{print}\NormalTok{(results, }\AttributeTok{n =} \ConstantTok{Inf}\NormalTok{)}
\end{Highlighting}
\end{Shaded}

\begin{verbatim}
# A tibble: 18 x 7
# Groups:   Sex, Education, Age.Group [18]
   Sex   Education    Age.Group Dep.Var  Count Percentage Total.N
   <fct> <fct>        <fct>     <fct>    <int>      <dbl>   <int>
 1 F     Educated     Old       Deletion     2       6.25      32
 2 F     Educated     Middle    Deletion    68      30.8      221
 3 F     Educated     Young     Deletion    20      27.8       72
 4 F     Not Educated Old       Deletion    41      34.7      118
 5 F     Not Educated Middle    Deletion     5      29.4       17
 6 F     Not Educated Young     Deletion     0       0          0
 7 F     Student      Old       Deletion     0       0          0
 8 F     Student      Middle    Deletion     0       0          0
 9 F     Student      Young     Deletion    52      26.1      199
10 M     Educated     Old       Deletion     0       0          0
11 M     Educated     Middle    Deletion    16      34.8       46
12 M     Educated     Young     Deletion    48      38.4      125
13 M     Not Educated Old       Deletion    24      47.1       51
14 M     Not Educated Middle    Deletion    36      47.4       76
15 M     Not Educated Young     Deletion    24      43.6       55
16 M     Student      Old       Deletion     0       0          0
17 M     Student      Middle    Deletion     0       0          0
18 M     Student      Young     Deletion    50      28.2      177
\end{verbatim}

The above results show that there are 32 tokens from old, educated
females, 2 of which (or 6.25\%) are \texttt{Deletion}.



\end{document}
