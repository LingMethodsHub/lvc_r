% Options for packages loaded elsewhere
\PassOptionsToPackage{unicode}{hyperref}
\PassOptionsToPackage{hyphens}{url}
\PassOptionsToPackage{dvipsnames,svgnames,x11names}{xcolor}
%
\documentclass[
  10pt,
  letterpaper]{article}

\usepackage{amsmath,amssymb}
\usepackage{lmodern}
\usepackage{iftex}
\ifPDFTeX
  \usepackage[T1]{fontenc}
  \usepackage[utf8]{inputenc}
  \usepackage{textcomp} % provide euro and other symbols
\else % if luatex or xetex
  \usepackage{unicode-math}
  \defaultfontfeatures{Scale=MatchLowercase}
  \defaultfontfeatures[\rmfamily]{Ligatures=TeX,Scale=1}
  \setmainfont[]{Charis SIL}
  \setmonofont[]{Monaco}
  \setmathfont[]{Monaco}
\fi
% Use upquote if available, for straight quotes in verbatim environments
\IfFileExists{upquote.sty}{\usepackage{upquote}}{}
\IfFileExists{microtype.sty}{% use microtype if available
  \usepackage[]{microtype}
  \UseMicrotypeSet[protrusion]{basicmath} % disable protrusion for tt fonts
}{}
\makeatletter
\@ifundefined{KOMAClassName}{% if non-KOMA class
  \IfFileExists{parskip.sty}{%
    \usepackage{parskip}
  }{% else
    \setlength{\parindent}{0pt}
    \setlength{\parskip}{6pt plus 2pt minus 1pt}}
}{% if KOMA class
  \KOMAoptions{parskip=half}}
\makeatother
\usepackage{xcolor}
\usepackage[margin = 1in]{geometry}
\setlength{\emergencystretch}{3em} % prevent overfull lines
\setcounter{secnumdepth}{-\maxdimen} % remove section numbering
% Make \paragraph and \subparagraph free-standing
\ifx\paragraph\undefined\else
  \let\oldparagraph\paragraph
  \renewcommand{\paragraph}[1]{\oldparagraph{#1}\mbox{}}
\fi
\ifx\subparagraph\undefined\else
  \let\oldsubparagraph\subparagraph
  \renewcommand{\subparagraph}[1]{\oldsubparagraph{#1}\mbox{}}
\fi

\usepackage{color}
\usepackage{fancyvrb}
\newcommand{\VerbBar}{|}
\newcommand{\VERB}{\Verb[commandchars=\\\{\}]}
\DefineVerbatimEnvironment{Highlighting}{Verbatim}{commandchars=\\\{\}}
% Add ',fontsize=\small' for more characters per line
\usepackage{framed}
\definecolor{shadecolor}{RGB}{241,243,245}
\newenvironment{Shaded}{\begin{snugshade}}{\end{snugshade}}
\newcommand{\AlertTok}[1]{\textcolor[rgb]{0.68,0.00,0.00}{#1}}
\newcommand{\AnnotationTok}[1]{\textcolor[rgb]{0.37,0.37,0.37}{#1}}
\newcommand{\AttributeTok}[1]{\textcolor[rgb]{0.40,0.45,0.13}{#1}}
\newcommand{\BaseNTok}[1]{\textcolor[rgb]{0.68,0.00,0.00}{#1}}
\newcommand{\BuiltInTok}[1]{\textcolor[rgb]{0.00,0.23,0.31}{#1}}
\newcommand{\CharTok}[1]{\textcolor[rgb]{0.13,0.47,0.30}{#1}}
\newcommand{\CommentTok}[1]{\textcolor[rgb]{0.37,0.37,0.37}{#1}}
\newcommand{\CommentVarTok}[1]{\textcolor[rgb]{0.37,0.37,0.37}{\textit{#1}}}
\newcommand{\ConstantTok}[1]{\textcolor[rgb]{0.56,0.35,0.01}{#1}}
\newcommand{\ControlFlowTok}[1]{\textcolor[rgb]{0.00,0.23,0.31}{#1}}
\newcommand{\DataTypeTok}[1]{\textcolor[rgb]{0.68,0.00,0.00}{#1}}
\newcommand{\DecValTok}[1]{\textcolor[rgb]{0.68,0.00,0.00}{#1}}
\newcommand{\DocumentationTok}[1]{\textcolor[rgb]{0.37,0.37,0.37}{\textit{#1}}}
\newcommand{\ErrorTok}[1]{\textcolor[rgb]{0.68,0.00,0.00}{#1}}
\newcommand{\ExtensionTok}[1]{\textcolor[rgb]{0.00,0.23,0.31}{#1}}
\newcommand{\FloatTok}[1]{\textcolor[rgb]{0.68,0.00,0.00}{#1}}
\newcommand{\FunctionTok}[1]{\textcolor[rgb]{0.28,0.35,0.67}{#1}}
\newcommand{\ImportTok}[1]{\textcolor[rgb]{0.00,0.46,0.62}{#1}}
\newcommand{\InformationTok}[1]{\textcolor[rgb]{0.37,0.37,0.37}{#1}}
\newcommand{\KeywordTok}[1]{\textcolor[rgb]{0.00,0.23,0.31}{#1}}
\newcommand{\NormalTok}[1]{\textcolor[rgb]{0.00,0.23,0.31}{#1}}
\newcommand{\OperatorTok}[1]{\textcolor[rgb]{0.37,0.37,0.37}{#1}}
\newcommand{\OtherTok}[1]{\textcolor[rgb]{0.00,0.23,0.31}{#1}}
\newcommand{\PreprocessorTok}[1]{\textcolor[rgb]{0.68,0.00,0.00}{#1}}
\newcommand{\RegionMarkerTok}[1]{\textcolor[rgb]{0.00,0.23,0.31}{#1}}
\newcommand{\SpecialCharTok}[1]{\textcolor[rgb]{0.37,0.37,0.37}{#1}}
\newcommand{\SpecialStringTok}[1]{\textcolor[rgb]{0.13,0.47,0.30}{#1}}
\newcommand{\StringTok}[1]{\textcolor[rgb]{0.13,0.47,0.30}{#1}}
\newcommand{\VariableTok}[1]{\textcolor[rgb]{0.07,0.07,0.07}{#1}}
\newcommand{\VerbatimStringTok}[1]{\textcolor[rgb]{0.13,0.47,0.30}{#1}}
\newcommand{\WarningTok}[1]{\textcolor[rgb]{0.37,0.37,0.37}{\textit{#1}}}

\providecommand{\tightlist}{%
  \setlength{\itemsep}{0pt}\setlength{\parskip}{0pt}}\usepackage{longtable,booktabs,array}
\usepackage{calc} % for calculating minipage widths
% Correct order of tables after \paragraph or \subparagraph
\usepackage{etoolbox}
\makeatletter
\patchcmd\longtable{\par}{\if@noskipsec\mbox{}\fi\par}{}{}
\makeatother
% Allow footnotes in longtable head/foot
\IfFileExists{footnotehyper.sty}{\usepackage{footnotehyper}}{\usepackage{footnote}}
\makesavenoteenv{longtable}
\usepackage{graphicx}
\makeatletter
\def\maxwidth{\ifdim\Gin@nat@width>\linewidth\linewidth\else\Gin@nat@width\fi}
\def\maxheight{\ifdim\Gin@nat@height>\textheight\textheight\else\Gin@nat@height\fi}
\makeatother
% Scale images if necessary, so that they will not overflow the page
% margins by default, and it is still possible to overwrite the defaults
% using explicit options in \includegraphics[width, height, ...]{}
\setkeys{Gin}{width=\maxwidth,height=\maxheight,keepaspectratio}
% Set default figure placement to htbp
\makeatletter
\def\fps@figure{htbp}
\makeatother
\newlength{\cslhangindent}
\setlength{\cslhangindent}{1.5em}
\newlength{\csllabelwidth}
\setlength{\csllabelwidth}{3em}
\newlength{\cslentryspacingunit} % times entry-spacing
\setlength{\cslentryspacingunit}{\parskip}
\newenvironment{CSLReferences}[2] % #1 hanging-ident, #2 entry spacing
 {% don't indent paragraphs
  \setlength{\parindent}{0pt}
  % turn on hanging indent if param 1 is 1
  \ifodd #1
  \let\oldpar\par
  \def\par{\hangindent=\cslhangindent\oldpar}
  \fi
  % set entry spacing
  \setlength{\parskip}{#2\cslentryspacingunit}
 }%
 {}
\usepackage{calc}
\newcommand{\CSLBlock}[1]{#1\hfill\break}
\newcommand{\CSLLeftMargin}[1]{\parbox[t]{\csllabelwidth}{#1}}
\newcommand{\CSLRightInline}[1]{\parbox[t]{\linewidth - \csllabelwidth}{#1}\break}
\newcommand{\CSLIndent}[1]{\hspace{\cslhangindent}#1}

\usepackage{tabularx}
\usepackage{threeparttable}
\usepackage{booktabs}
\usepackage{tipa}
\let\Oldtexttt\texttt
\renewcommand\texttt[1]{{\ttfamily\color{BrickRed}#1}}
\usepackage{authoraftertitle}
\usepackage{fancyhdr}
\pagestyle{fancy}
\rfoot{\copyright Matt Hunt Gardner}
\cfoot{\thepage}
\lhead{Doing LVC with \textit{R}: \MyTitle}
\rhead{}
\makeatletter
\@ifpackageloaded{tcolorbox}{}{\usepackage[many]{tcolorbox}}
\@ifpackageloaded{fontawesome5}{}{\usepackage{fontawesome5}}
\definecolor{quarto-callout-color}{HTML}{909090}
\definecolor{quarto-callout-note-color}{HTML}{0758E5}
\definecolor{quarto-callout-important-color}{HTML}{CC1914}
\definecolor{quarto-callout-warning-color}{HTML}{EB9113}
\definecolor{quarto-callout-tip-color}{HTML}{00A047}
\definecolor{quarto-callout-caution-color}{HTML}{FC5300}
\definecolor{quarto-callout-color-frame}{HTML}{acacac}
\definecolor{quarto-callout-note-color-frame}{HTML}{4582ec}
\definecolor{quarto-callout-important-color-frame}{HTML}{d9534f}
\definecolor{quarto-callout-warning-color-frame}{HTML}{f0ad4e}
\definecolor{quarto-callout-tip-color-frame}{HTML}{02b875}
\definecolor{quarto-callout-caution-color-frame}{HTML}{fd7e14}
\makeatother
\makeatletter
\makeatother
\makeatletter
\makeatother
\makeatletter
\@ifpackageloaded{caption}{}{\usepackage{caption}}
\AtBeginDocument{%
\ifdefined\contentsname
  \renewcommand*\contentsname{Table of contents}
\else
  \newcommand\contentsname{Table of contents}
\fi
\ifdefined\listfigurename
  \renewcommand*\listfigurename{List of Figures}
\else
  \newcommand\listfigurename{List of Figures}
\fi
\ifdefined\listtablename
  \renewcommand*\listtablename{List of Tables}
\else
  \newcommand\listtablename{List of Tables}
\fi
\ifdefined\figurename
  \renewcommand*\figurename{Figure}
\else
  \newcommand\figurename{Figure}
\fi
\ifdefined\tablename
  \renewcommand*\tablename{Table}
\else
  \newcommand\tablename{Table}
\fi
}
\@ifpackageloaded{float}{}{\usepackage{float}}
\floatstyle{ruled}
\@ifundefined{c@chapter}{\newfloat{codelisting}{h}{lop}}{\newfloat{codelisting}{h}{lop}[chapter]}
\floatname{codelisting}{Listing}
\newcommand*\listoflistings{\listof{codelisting}{List of Listings}}
\makeatother
\makeatletter
\@ifpackageloaded{caption}{}{\usepackage{caption}}
\@ifpackageloaded{subcaption}{}{\usepackage{subcaption}}
\makeatother
\makeatletter
\@ifpackageloaded{tcolorbox}{}{\usepackage[many]{tcolorbox}}
\makeatother
\makeatletter
\@ifundefined{shadecolor}{\definecolor{shadecolor}{rgb}{.97, .97, .97}}
\makeatother
\makeatletter
\makeatother
\ifLuaTeX
  \usepackage{selnolig}  % disable illegal ligatures
\fi
\IfFileExists{bookmark.sty}{\usepackage{bookmark}}{\usepackage{hyperref}}
\IfFileExists{xurl.sty}{\usepackage{xurl}}{} % add URL line breaks if available
\urlstyle{same} % disable monospaced font for URLs
% Make links footnotes instead of hotlinks:
\DeclareRobustCommand{\href}[2]{#2\footnote{\url{#1}}}
\hypersetup{
  pdftitle={Getting Your Data Into R},
  pdfauthor={Matt Hunt Gardner},
  colorlinks=true,
  linkcolor={blue},
  filecolor={Maroon},
  citecolor={Blue},
  urlcolor={Blue},
  pdfcreator={LaTeX via pandoc}}

\title{Getting Your Data Into \emph{R}}
\usepackage{etoolbox}
\makeatletter
\providecommand{\subtitle}[1]{% add subtitle to \maketitle
  \apptocmd{\@title}{\par {\large #1 \par}}{}{}
}
\makeatother
\subtitle{from
\href{https://lingmethodshub.github.io/content/R/lvc_r/}{Doing LVC with
\emph{R}}}
\author{Matt Hunt Gardner}
\date{3/10/23}

\begin{document}
\maketitle
\ifdefined\Shaded\renewenvironment{Shaded}{\begin{tcolorbox}[enhanced, interior hidden, sharp corners, breakable, borderline west={3pt}{0pt}{shadecolor}, frame hidden, boxrule=0pt]}{\end{tcolorbox}}\fi

The best way to organize your tokens is in a spreadsheet in Microsoft
\emph{Excel}. There are a lot of useful tools in Microsoft \emph{Excel}
for automatically coding and re-coding your data. Some of those tools
overlap with what I present here for \emph{R}. If you want to explore
\emph{Excel}'s functionality, I recommend this website as a springboard:
\url{https://support.office.com/en-us/article/Excel-training-9bc05390-e94c-46af-a5b3-d7c22f6990bb}.
There are other programs for spreadsheet management: \emph{Numbers} in
OSX, or Google's \emph{Sheets}, for example. They generally function
similarly to \emph{Excel}. \emph{R} does not easily read \emph{Excel}'s
default file type (\textbf{.xls} or \textbf{.xlsx},
\href{https://cran.r-project.org/web/packages/xlsx/xlsx.pdf}{though it
can be done}) therefore, you must save your token file as a
tab-delimited-text file (\textbf{.txt}) or a comma-separated-values
(\textbf{.csv}) file.

\begin{tcolorbox}[enhanced jigsaw, opacitybacktitle=0.6, toprule=.15mm, leftrule=.75mm, colback=white, left=2mm, rightrule=.15mm, bottomrule=.15mm, opacityback=0, titlerule=0mm, title=\textcolor{quarto-callout-warning-color}{\faExclamationTriangle}\hspace{0.5em}{Warning}, colbacktitle=quarto-callout-warning-color!10!white, colframe=quarto-callout-warning-color-frame, coltitle=black, bottomtitle=1mm, toptitle=1mm, breakable, arc=.35mm]

I recommend NOT saving your token files as a .\textbf{csv file}. Often
your token files will include a column of cells containing the broader
context the token was extracted from. For example, the sentence in which
it appears in a transcript. This broader context usually includes
commas. If this is the case, when you save your file as a \textbf{.csv
file}, it will appear as if there are column breaks in the middle of
your broader context because commas are used as the column delimiter.

\end{tcolorbox}

There are four or five ways to read data into \emph{R}. Which method you
choose is really up to you, but because I'm advocating the use of
\emph{R} script files and maximum replicability, I suggest using the
following function at the top of your script file:

\begin{Shaded}
\begin{Highlighting}[]
\NormalTok{td }\OtherTok{\textless{}{-}} \FunctionTok{read.delim}\NormalTok{(}\StringTok{"Data/deletiondata.txt"}\NormalTok{)}
\end{Highlighting}
\end{Shaded}

The function creates an \emph{R} ``object'' called \texttt{td} and then
uses the assignment operator \texttt{\textless{}-} to specify what that
object is. In this case \texttt{td} is the contents of the tab-delimited
text file called \texttt{deletiondata.txt} that is located in a folder
called \texttt{Data}, in the same directory as my script file. If your
data is saved somewhere else on your computer, replace
\texttt{\textquotesingle{}Data/deletiondata.txt"} with the full path of
your data file. The \texttt{deletiondata.txt} is a tab-delimited text
file and is the data file I will use to teach you about \emph{R}. You
can download this same file
\href{https://www.dropbox.com/s/pi8xz1kuo6cz60l/deletiondata.txt?dl=1}{here}.
Wherever you save this file, write that file path in quotation marks
inside the \texttt{read.delim()} function. On a PC this file path will
likely begin \texttt{"C:/..."}. You can actually just read the file
directly into \emph{R} from the web link using the following function:

\begin{Shaded}
\begin{Highlighting}[]
\NormalTok{td }\OtherTok{\textless{}{-}}\FunctionTok{read.delim}\NormalTok{(}\StringTok{"https://www.dropbox.com/s/pi8xz1kuo6cz60l/deletiondata.txt?dl=1"}\NormalTok{)}
\end{Highlighting}
\end{Shaded}

This data file contains tokens of word-final (t, d) and was created for
a project studying different pronunciations of word-final (t, d),
including deletion. The data comes from a corpus collected for Gardner
(2010, 2013, 2017) among English speakers on
\href{https://en.wikipedia.org/wiki/Cape_Breton_Island}{Cape Breton
Island, Nova Scotia, Canada}.

If you prefer to save your data files as comma-separated-values files
(even though you shouldn't, see above), you can read them into \emph{R}
using the function \texttt{read.csv2()}. If you find it tricky to figure
out the file path of your data file you can instead write
\texttt{file.choose()} (OS X) or \texttt{choose.files()} (PC) instead of
the file path inside the \texttt{read.delim()}/\texttt{read.csv2()}
function, with no quotation marks. This will create a pop-up window
where you can browse through your files and select one. While this
\emph{seems} easier, it is not worth it. By not explicitly writing out
the file path you introduce a non-replicable element in your script file
because there is no record of what you browse to in the actual script
file. This means that if you return to your project a year later, or
someone else is looking over your code, it might not be clear which data
file is supposed to be used. If you choose to use an \emph{R} script
file you can actually just drag and drop your data file (or any file)
into the script window itself and the full file path will be
automatically inserted.

You can also copy a file's filepath to the clipboard on a Mac by
pressing \textbf{Control} while clicking on the file, then pressing
\textbf{Option} and selecting \textbf{Copy ``{[}your file{]}'' as
Pathname}. On a PC you can do the same thing by right-clicking on a
file, or (if using Windows 10) using the \textbf{Copy path} button on
the \textbf{Home} tab ribbon in \emph{Windows File Explorer}.

For more information about reading files into \emph{R}, go
\href{https://stat.ethz.ch/R-manual/R-devel/library/utils/html/read.table.html}{here}

\hypertarget{references}{%
\subsubsection{References}\label{references}}

\hypertarget{refs}{}
\begin{CSLReferences}{1}{0}
\leavevmode\vadjust pre{\hypertarget{ref-Gardner2010}{}}%
Gardner, Matt Hunt. 2010. {``{O}at and a {B}oat: {I}dentity Creation
Through Regional Speech Features in {I}ndustrial {C}ape {B}reton.''} MA
thesis, {S}t. {J}ohn's, {N}{L}, {C}anada: Memorial University.

\leavevmode\vadjust pre{\hypertarget{ref-Gardner2013c}{}}%
---------. 2013. {``Variable Word-Final (t,d) in {C}ape {B}reton
{E}nglish.''} Generals Paper, {T}oronto: {U}niversity of {T}oronto.

\leavevmode\vadjust pre{\hypertarget{ref-Gardner2017}{}}%
---------. 2017. {``Grammatical Variation and Change in {I}ndustrial
{C}ape {B}reton.''} PhD dissertation, {U}niversity of {T}oronto.

\end{CSLReferences}



\end{document}
