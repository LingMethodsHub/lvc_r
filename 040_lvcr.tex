% Options for packages loaded elsewhere
\PassOptionsToPackage{unicode}{hyperref}
\PassOptionsToPackage{hyphens}{url}
\PassOptionsToPackage{dvipsnames,svgnames,x11names}{xcolor}
%
\documentclass[
  12pt,
  letterpaper]{article}

\usepackage{amsmath,amssymb}
\usepackage{lmodern}
\usepackage{iftex}
\ifPDFTeX
  \usepackage[T1]{fontenc}
  \usepackage[utf8]{inputenc}
  \usepackage{textcomp} % provide euro and other symbols
\else % if luatex or xetex
  \usepackage{unicode-math}
  \defaultfontfeatures{Scale=MatchLowercase}
  \defaultfontfeatures[\rmfamily]{Ligatures=TeX,Scale=1}
  \setmainfont[]{Charis SIL}
  \setmathfont[]{Monaco}
\fi
% Use upquote if available, for straight quotes in verbatim environments
\IfFileExists{upquote.sty}{\usepackage{upquote}}{}
\IfFileExists{microtype.sty}{% use microtype if available
  \usepackage[]{microtype}
  \UseMicrotypeSet[protrusion]{basicmath} % disable protrusion for tt fonts
}{}
\makeatletter
\@ifundefined{KOMAClassName}{% if non-KOMA class
  \IfFileExists{parskip.sty}{%
    \usepackage{parskip}
  }{% else
    \setlength{\parindent}{0pt}
    \setlength{\parskip}{6pt plus 2pt minus 1pt}}
}{% if KOMA class
  \KOMAoptions{parskip=half}}
\makeatother
\usepackage{xcolor}
\usepackage[margin = 1in]{geometry}
\setlength{\emergencystretch}{3em} % prevent overfull lines
\setcounter{secnumdepth}{-\maxdimen} % remove section numbering
% Make \paragraph and \subparagraph free-standing
\ifx\paragraph\undefined\else
  \let\oldparagraph\paragraph
  \renewcommand{\paragraph}[1]{\oldparagraph{#1}\mbox{}}
\fi
\ifx\subparagraph\undefined\else
  \let\oldsubparagraph\subparagraph
  \renewcommand{\subparagraph}[1]{\oldsubparagraph{#1}\mbox{}}
\fi

\usepackage{color}
\usepackage{fancyvrb}
\newcommand{\VerbBar}{|}
\newcommand{\VERB}{\Verb[commandchars=\\\{\}]}
\DefineVerbatimEnvironment{Highlighting}{Verbatim}{commandchars=\\\{\}}
% Add ',fontsize=\small' for more characters per line
\usepackage{framed}
\definecolor{shadecolor}{RGB}{241,243,245}
\newenvironment{Shaded}{\begin{snugshade}}{\end{snugshade}}
\newcommand{\AlertTok}[1]{\textcolor[rgb]{0.68,0.00,0.00}{#1}}
\newcommand{\AnnotationTok}[1]{\textcolor[rgb]{0.37,0.37,0.37}{#1}}
\newcommand{\AttributeTok}[1]{\textcolor[rgb]{0.40,0.45,0.13}{#1}}
\newcommand{\BaseNTok}[1]{\textcolor[rgb]{0.68,0.00,0.00}{#1}}
\newcommand{\BuiltInTok}[1]{\textcolor[rgb]{0.00,0.23,0.31}{#1}}
\newcommand{\CharTok}[1]{\textcolor[rgb]{0.13,0.47,0.30}{#1}}
\newcommand{\CommentTok}[1]{\textcolor[rgb]{0.37,0.37,0.37}{#1}}
\newcommand{\CommentVarTok}[1]{\textcolor[rgb]{0.37,0.37,0.37}{\textit{#1}}}
\newcommand{\ConstantTok}[1]{\textcolor[rgb]{0.56,0.35,0.01}{#1}}
\newcommand{\ControlFlowTok}[1]{\textcolor[rgb]{0.00,0.23,0.31}{#1}}
\newcommand{\DataTypeTok}[1]{\textcolor[rgb]{0.68,0.00,0.00}{#1}}
\newcommand{\DecValTok}[1]{\textcolor[rgb]{0.68,0.00,0.00}{#1}}
\newcommand{\DocumentationTok}[1]{\textcolor[rgb]{0.37,0.37,0.37}{\textit{#1}}}
\newcommand{\ErrorTok}[1]{\textcolor[rgb]{0.68,0.00,0.00}{#1}}
\newcommand{\ExtensionTok}[1]{\textcolor[rgb]{0.00,0.23,0.31}{#1}}
\newcommand{\FloatTok}[1]{\textcolor[rgb]{0.68,0.00,0.00}{#1}}
\newcommand{\FunctionTok}[1]{\textcolor[rgb]{0.28,0.35,0.67}{#1}}
\newcommand{\ImportTok}[1]{\textcolor[rgb]{0.00,0.46,0.62}{#1}}
\newcommand{\InformationTok}[1]{\textcolor[rgb]{0.37,0.37,0.37}{#1}}
\newcommand{\KeywordTok}[1]{\textcolor[rgb]{0.00,0.23,0.31}{#1}}
\newcommand{\NormalTok}[1]{\textcolor[rgb]{0.00,0.23,0.31}{#1}}
\newcommand{\OperatorTok}[1]{\textcolor[rgb]{0.37,0.37,0.37}{#1}}
\newcommand{\OtherTok}[1]{\textcolor[rgb]{0.00,0.23,0.31}{#1}}
\newcommand{\PreprocessorTok}[1]{\textcolor[rgb]{0.68,0.00,0.00}{#1}}
\newcommand{\RegionMarkerTok}[1]{\textcolor[rgb]{0.00,0.23,0.31}{#1}}
\newcommand{\SpecialCharTok}[1]{\textcolor[rgb]{0.37,0.37,0.37}{#1}}
\newcommand{\SpecialStringTok}[1]{\textcolor[rgb]{0.13,0.47,0.30}{#1}}
\newcommand{\StringTok}[1]{\textcolor[rgb]{0.13,0.47,0.30}{#1}}
\newcommand{\VariableTok}[1]{\textcolor[rgb]{0.07,0.07,0.07}{#1}}
\newcommand{\VerbatimStringTok}[1]{\textcolor[rgb]{0.13,0.47,0.30}{#1}}
\newcommand{\WarningTok}[1]{\textcolor[rgb]{0.37,0.37,0.37}{\textit{#1}}}

\providecommand{\tightlist}{%
  \setlength{\itemsep}{0pt}\setlength{\parskip}{0pt}}\usepackage{longtable,booktabs,array}
\usepackage{calc} % for calculating minipage widths
% Correct order of tables after \paragraph or \subparagraph
\usepackage{etoolbox}
\makeatletter
\patchcmd\longtable{\par}{\if@noskipsec\mbox{}\fi\par}{}{}
\makeatother
% Allow footnotes in longtable head/foot
\IfFileExists{footnotehyper.sty}{\usepackage{footnotehyper}}{\usepackage{footnote}}
\makesavenoteenv{longtable}
\usepackage{graphicx}
\makeatletter
\def\maxwidth{\ifdim\Gin@nat@width>\linewidth\linewidth\else\Gin@nat@width\fi}
\def\maxheight{\ifdim\Gin@nat@height>\textheight\textheight\else\Gin@nat@height\fi}
\makeatother
% Scale images if necessary, so that they will not overflow the page
% margins by default, and it is still possible to overwrite the defaults
% using explicit options in \includegraphics[width, height, ...]{}
\setkeys{Gin}{width=\maxwidth,height=\maxheight,keepaspectratio}
% Set default figure placement to htbp
\makeatletter
\def\fps@figure{htbp}
\makeatother

\usepackage{tabularx}
\usepackage{threeparttable}
\usepackage{booktabs}
\usepackage{tipa}
\let\Oldtexttt\texttt
\renewcommand\texttt[1]{{\ttfamily\color{BrickRed}#1}}
\usepackage{authoraftertitle}
\usepackage{fancyhdr}
\pagestyle{fancy}
\rfoot{\copyright Matt Hunt Gardner}
\cfoot{\thepage}
\lhead{Doing LVC with \textit{R}: \MyTitle}
\rhead{}
\makeatletter
\@ifpackageloaded{tcolorbox}{}{\usepackage[many]{tcolorbox}}
\@ifpackageloaded{fontawesome5}{}{\usepackage{fontawesome5}}
\definecolor{quarto-callout-color}{HTML}{909090}
\definecolor{quarto-callout-note-color}{HTML}{0758E5}
\definecolor{quarto-callout-important-color}{HTML}{CC1914}
\definecolor{quarto-callout-warning-color}{HTML}{EB9113}
\definecolor{quarto-callout-tip-color}{HTML}{00A047}
\definecolor{quarto-callout-caution-color}{HTML}{FC5300}
\definecolor{quarto-callout-color-frame}{HTML}{acacac}
\definecolor{quarto-callout-note-color-frame}{HTML}{4582ec}
\definecolor{quarto-callout-important-color-frame}{HTML}{d9534f}
\definecolor{quarto-callout-warning-color-frame}{HTML}{f0ad4e}
\definecolor{quarto-callout-tip-color-frame}{HTML}{02b875}
\definecolor{quarto-callout-caution-color-frame}{HTML}{fd7e14}
\makeatother
\makeatletter
\makeatother
\makeatletter
\makeatother
\makeatletter
\@ifpackageloaded{caption}{}{\usepackage{caption}}
\AtBeginDocument{%
\ifdefined\contentsname
  \renewcommand*\contentsname{Table of contents}
\else
  \newcommand\contentsname{Table of contents}
\fi
\ifdefined\listfigurename
  \renewcommand*\listfigurename{List of Figures}
\else
  \newcommand\listfigurename{List of Figures}
\fi
\ifdefined\listtablename
  \renewcommand*\listtablename{List of Tables}
\else
  \newcommand\listtablename{List of Tables}
\fi
\ifdefined\figurename
  \renewcommand*\figurename{Figure}
\else
  \newcommand\figurename{Figure}
\fi
\ifdefined\tablename
  \renewcommand*\tablename{Table}
\else
  \newcommand\tablename{Table}
\fi
}
\@ifpackageloaded{float}{}{\usepackage{float}}
\floatstyle{ruled}
\@ifundefined{c@chapter}{\newfloat{codelisting}{h}{lop}}{\newfloat{codelisting}{h}{lop}[chapter]}
\floatname{codelisting}{Listing}
\newcommand*\listoflistings{\listof{codelisting}{List of Listings}}
\makeatother
\makeatletter
\@ifpackageloaded{caption}{}{\usepackage{caption}}
\@ifpackageloaded{subcaption}{}{\usepackage{subcaption}}
\makeatother
\makeatletter
\@ifpackageloaded{tcolorbox}{}{\usepackage[many]{tcolorbox}}
\makeatother
\makeatletter
\@ifundefined{shadecolor}{\definecolor{shadecolor}{rgb}{.97, .97, .97}}
\makeatother
\makeatletter
\makeatother
\ifLuaTeX
  \usepackage{selnolig}  % disable illegal ligatures
\fi
\IfFileExists{bookmark.sty}{\usepackage{bookmark}}{\usepackage{hyperref}}
\IfFileExists{xurl.sty}{\usepackage{xurl}}{} % add URL line breaks if available
\urlstyle{same} % disable monospaced font for URLs
% Make links footnotes instead of hotlinks:
\DeclareRobustCommand{\href}[2]{#2\footnote{\url{#1}}}
\hypersetup{
  pdftitle={Modifying Data},
  pdfauthor={Matt Hunt Gardner},
  colorlinks=true,
  linkcolor={blue},
  filecolor={Maroon},
  citecolor={Blue},
  urlcolor={Blue},
  pdfcreator={LaTeX via pandoc}}

\title{Modifying Data}
\usepackage{etoolbox}
\makeatletter
\providecommand{\subtitle}[1]{% add subtitle to \maketitle
  \apptocmd{\@title}{\par {\large #1 \par}}{}{}
}
\makeatother
\subtitle{from
\href{https://lingmethodshub.github.io/content/R/lvc_r/}{Doing LVC with
\emph{R}}}
\author{Matt Hunt Gardner}
\date{2022-12-14}

\begin{document}
\maketitle
\ifdefined\Shaded\renewenvironment{Shaded}{\begin{tcolorbox}[frame hidden, sharp corners, breakable, boxrule=0pt, enhanced, borderline west={3pt}{0pt}{shadecolor}, interior hidden]}{\end{tcolorbox}}\fi

\renewcommand*\contentsname{Table of contents}
{
\hypersetup{linkcolor=}
\setcounter{tocdepth}{3}
\tableofcontents
}
If you followed the previous section you now have an object in \emph{R}
called \texttt{td} in which all character columns are converted to
factors. If not, you can load it now with either of the following codes.

\begin{Shaded}
\begin{Highlighting}[]
\NormalTok{td }\OtherTok{\textless{}{-}} \FunctionTok{read.delim}\NormalTok{(}\StringTok{"https://www.dropbox.com/s/jxlfuogea3lx2pu/deletiondata.txt?dl=1"}\NormalTok{)}
\end{Highlighting}
\end{Shaded}

\begin{Shaded}
\begin{Highlighting}[]
\NormalTok{td }\OtherTok{\textless{}{-}} \FunctionTok{read.delim}\NormalTok{(}\StringTok{"Data/deletiondata.txt"}\NormalTok{)}
\end{Highlighting}
\end{Shaded}

Re-class character columns as factors.

\begin{Shaded}
\begin{Highlighting}[]
\NormalTok{td}\SpecialCharTok{$}\NormalTok{Dep.Var }\OtherTok{\textless{}{-}} \FunctionTok{factor}\NormalTok{(td}\SpecialCharTok{$}\NormalTok{Dep.Var)}
\NormalTok{td}\SpecialCharTok{$}\NormalTok{Stress }\OtherTok{\textless{}{-}} \FunctionTok{factor}\NormalTok{(td}\SpecialCharTok{$}\NormalTok{Stress)}
\NormalTok{td}\SpecialCharTok{$}\NormalTok{Category }\OtherTok{\textless{}{-}} \FunctionTok{factor}\NormalTok{(td}\SpecialCharTok{$}\NormalTok{Category)}
\NormalTok{td}\SpecialCharTok{$}\NormalTok{Morph.Type }\OtherTok{\textless{}{-}} \FunctionTok{factor}\NormalTok{(td}\SpecialCharTok{$}\NormalTok{Morph.Type)}
\NormalTok{td}\SpecialCharTok{$}\NormalTok{Before }\OtherTok{\textless{}{-}} \FunctionTok{factor}\NormalTok{(td}\SpecialCharTok{$}\NormalTok{Before)}
\NormalTok{td}\SpecialCharTok{$}\NormalTok{After }\OtherTok{\textless{}{-}} \FunctionTok{factor}\NormalTok{(td}\SpecialCharTok{$}\NormalTok{After)}
\NormalTok{td}\SpecialCharTok{$}\NormalTok{Speaker }\OtherTok{\textless{}{-}} \FunctionTok{factor}\NormalTok{(td}\SpecialCharTok{$}\NormalTok{Speaker)}
\NormalTok{td}\SpecialCharTok{$}\NormalTok{Sex }\OtherTok{\textless{}{-}} \FunctionTok{factor}\NormalTok{(td}\SpecialCharTok{$}\NormalTok{Sex)}
\NormalTok{td}\SpecialCharTok{$}\NormalTok{Education }\OtherTok{\textless{}{-}} \FunctionTok{factor}\NormalTok{(td}\SpecialCharTok{$}\NormalTok{Education)}
\NormalTok{td}\SpecialCharTok{$}\NormalTok{Job }\OtherTok{\textless{}{-}} \FunctionTok{factor}\NormalTok{(td}\SpecialCharTok{$}\NormalTok{Job)}
\NormalTok{td}\SpecialCharTok{$}\NormalTok{Phoneme.Dep.Var }\OtherTok{\textless{}{-}} \FunctionTok{factor}\NormalTok{(td}\SpecialCharTok{$}\NormalTok{Phoneme.Dep.Var)}
\end{Highlighting}
\end{Shaded}

Deleting tokens, re-coding tokens, combining factor groups, etc. is, in
my opinion, easier to do in \emph{Excel}. But, if you want to do this in
\emph{R}, or you only want to perform the modification for a specific
analysis or graph, you may find the following functions useful.

\hypertarget{removing-rows}{%
\subsection{Removing Rows}\label{removing-rows}}

The (t, d) deletion data includes tokens in which the previous phoneme
is a vowel. Many analyses of (t, d) deletion exclude this context. Here
is how to remove these contexts from your data frame.

\begin{Shaded}
\begin{Highlighting}[]
\NormalTok{td }\OtherTok{\textless{}{-}}\NormalTok{ td[td}\SpecialCharTok{$}\NormalTok{Before }\SpecialCharTok{!=} \StringTok{"Vowel"}\NormalTok{, ]}
\FunctionTok{summary}\NormalTok{(td)}
\end{Highlighting}
\end{Shaded}

\begin{verbatim}
     Dep.Var           Stress         Category        Morph.Type 
 Deletion:386   Stressed  :1047   Function:  57   Mono     :762  
 Realized:803   Unstressed: 142   Lexical :1132   Past     :311  
                                                  Semi-Weak:116  
                                                                 
                                                                 
                                                                 
                                                                 
             Before          After        Speaker         YOB       Sex    
 Liquid         :269   Consonant:215   INGM84 : 57   Min.   :1915   F:659  
 Nasal          :209   H        :157   MARM92 : 53   1st Qu.:1952   M:530  
 Other Fricative:130   Pause    :558   GARF87 : 52   Median :1984          
 S              :332   Vowel    :259   HUNF22 : 32   Mean   :1969          
 Stop           :249                   DONF15 : 28   3rd Qu.:1991          
 Vowel          :  0                   GARF37 : 28   Max.   :1999          
                                       (Other):939                         
        Education        Job          Phoneme.Dep.Var
 Educated    :496   Blue   :130   t--T        :352   
 Not Educated:317   Service:468   t--Deletion :296   
 Student     :376   Student:376   t--Fricative:131   
                    White  :215   d--Deletion : 90   
                                  d--T        : 73   
                                  d--D        : 46   
                                  (Other)     :201   
\end{verbatim}

The first line of code creates a new object called \texttt{td}. The
\texttt{\textless{}-} operator means you're telling \emph{R} that this
new \texttt{td} is the same as the old \texttt{td}, but filtered
according to some specific condition. You could also use the
\texttt{\textless{}-} operator to create a new \texttt{td} with a
different name --- e.g.,
\texttt{td.new\ \textless{}-\ td{[}td\textbackslash{}\$Before\ !=\ "Vowel",{]}\}}
--- thus giving you two data frames (\texttt{td} and \texttt{td.new}) to
work with.

Throughout \emph{R}, square brackets \texttt{{[}\ {]}} are used for
specifying filtering conditions. The first line of code therefore says
make new \texttt{td} the same as old \texttt{td}, but only where the
value in the \texttt{Before} column of old \texttt{td} (indicated by
\texttt{td\$Before}) does not equal \texttt{"Vowel"}. \texttt{!=} is the
standard operator meaning \emph{does not equal}. The comma after
\texttt{"Vowel"} is important. If you are filtering a data frame, as you
are here, \emph{R} needs to know where to look for the values you want
to keep or throw out. It follows this format:
\texttt{data\ frame{[}rows,columns{]}}. So the comma in the above
indicates that the filtering condition relates to values found while
searching row by row. It selects any row in which the value in the
\texttt{Before} column does not equal \texttt{"Vowel"}. The quotation
marks around \texttt{"Vowel"} are also important because the values in
the \texttt{Before} column are all strings of characters, which (you'll
remember from above), are always enclosed in quotation marks.

\begin{Shaded}
\begin{Highlighting}[]
\NormalTok{td}\SpecialCharTok{$}\NormalTok{Before }\OtherTok{\textless{}{-}} \FunctionTok{factor}\NormalTok{(td}\SpecialCharTok{$}\NormalTok{Before)}
\end{Highlighting}
\end{Shaded}

An additional line of code is needed for resetting the data structure.
The new \texttt{td} inherits the data structure from the old
\texttt{td}. This means \emph{R} thinks the new \texttt{td} still has a
level in the column \texttt{Before} called \texttt{"Vowel"} --- even
though there are zero tokens now in the column with this value. If you
run \texttt{summary(td)} after the first code above you'll see that
\emph{R} still lists \texttt{"Vowel"} as a possible level of
\texttt{Before}, but with zero instances. The additional line of code
tells \emph{R} to make a new column \texttt{Before} in the data frame
\texttt{td} (this will replace the old \texttt{Before} column) in which
the possible factors (i.e., values) in that column are those that
actually exist in the new \texttt{Before} column in \texttt{td}. Run
\texttt{summary()} or \texttt{str()} now and you'll see that the empty
\texttt{"Vowel"} level is gone.

\begin{Shaded}
\begin{Highlighting}[]
\FunctionTok{summary}\NormalTok{(td)}
\end{Highlighting}
\end{Shaded}

\begin{verbatim}
     Dep.Var           Stress         Category        Morph.Type 
 Deletion:386   Stressed  :1047   Function:  57   Mono     :762  
 Realized:803   Unstressed: 142   Lexical :1132   Past     :311  
                                                  Semi-Weak:116  
                                                                 
                                                                 
                                                                 
                                                                 
             Before          After        Speaker         YOB       Sex    
 Liquid         :269   Consonant:215   INGM84 : 57   Min.   :1915   F:659  
 Nasal          :209   H        :157   MARM92 : 53   1st Qu.:1952   M:530  
 Other Fricative:130   Pause    :558   GARF87 : 52   Median :1984          
 S              :332   Vowel    :259   HUNF22 : 32   Mean   :1969          
 Stop           :249                   DONF15 : 28   3rd Qu.:1991          
                                       GARF37 : 28   Max.   :1999          
                                       (Other):939                         
        Education        Job          Phoneme.Dep.Var
 Educated    :496   Blue   :130   t--T        :352   
 Not Educated:317   Service:468   t--Deletion :296   
 Student     :376   Student:376   t--Fricative:131   
                    White  :215   d--Deletion : 90   
                                  d--T        : 73   
                                  d--D        : 46   
                                  (Other)     :201   
\end{verbatim}

\begin{Shaded}
\begin{Highlighting}[]
\FunctionTok{str}\NormalTok{(td)}
\end{Highlighting}
\end{Shaded}

\begin{verbatim}
'data.frame':   1189 obs. of  12 variables:
 $ Dep.Var        : Factor w/ 2 levels "Deletion","Realized": 2 1 1 1 2 1 1 1 1 1 ...
 $ Stress         : Factor w/ 2 levels "Stressed","Unstressed": 1 1 1 1 1 1 1 1 1 1 ...
 $ Category       : Factor w/ 2 levels "Function","Lexical": 2 2 2 2 2 2 2 2 2 2 ...
 $ Morph.Type     : Factor w/ 3 levels "Mono","Past",..: 1 1 1 1 1 1 1 1 1 1 ...
 $ Before         : Factor w/ 5 levels "Liquid","Nasal",..: 5 5 5 5 5 5 5 5 5 5 ...
 $ After          : Factor w/ 4 levels "Consonant","H",..: 1 1 1 1 1 1 1 1 1 1 ...
 $ Speaker        : Factor w/ 66 levels "ARSM91","BEAM91",..: 3 5 5 6 13 14 15 23 29 33 ...
 $ YOB            : int  1965 1955 1955 1952 1953 1958 1946 1942 1945 1949 ...
 $ Sex            : Factor w/ 2 levels "F","M": 1 1 1 1 2 2 1 2 2 1 ...
 $ Education      : Factor w/ 3 levels "Educated","Not Educated",..: 1 1 1 1 1 2 1 2 2 1 ...
 $ Job            : Factor w/ 4 levels "Blue","Service",..: 4 4 4 2 2 2 2 1 1 2 ...
 $ Phoneme.Dep.Var: Factor w/ 18 levels "d--Affricate",..: 18 12 12 12 18 12 12 12 12 12 ...
\end{verbatim}

The following code does exactly the same thing as the previous code, but
instead of filtering out \texttt{"Vowel"}, it specifies keeping all the
other levels. Here \texttt{==} is the standard operator meaning
\emph{equals and only equals} and \texttt{\textbar{}} (called
\emph{pipe} or \emph{bar}) is the operator meaning \emph{or}. Again,
note the very important comma following the last column condition.

\begin{Shaded}
\begin{Highlighting}[]
\NormalTok{td }\OtherTok{\textless{}{-}}\NormalTok{ td[td}\SpecialCharTok{$}\NormalTok{Before }\SpecialCharTok{==} \StringTok{"Liquid"} \SpecialCharTok{|}\NormalTok{ td}\SpecialCharTok{$}\NormalTok{Before }\SpecialCharTok{==} \StringTok{"Nasal"} \SpecialCharTok{|}
\NormalTok{    td}\SpecialCharTok{$}\NormalTok{Before }\SpecialCharTok{==} \StringTok{"Other Fricative"} \SpecialCharTok{|}\NormalTok{ td}\SpecialCharTok{$}\NormalTok{Before }\SpecialCharTok{==} \StringTok{"S"} \SpecialCharTok{|}
\NormalTok{    td}\SpecialCharTok{$}\NormalTok{Before }\SpecialCharTok{==} \StringTok{"Stop"}\NormalTok{, ]}
\NormalTok{td}\SpecialCharTok{$}\NormalTok{Before }\OtherTok{\textless{}{-}} \FunctionTok{factor}\NormalTok{(td}\SpecialCharTok{$}\NormalTok{Before)}
\end{Highlighting}
\end{Shaded}

As you did previously, you must run the second line of code to get rid
of the empty \texttt{"Vowel"} level.

\hypertarget{re-coding-variables}{%
\subsection{Re-coding Variables}\label{re-coding-variables}}

While it is possible to do \emph{ad-hoc} re-codes in \emph{R}, you must
keep in mind that these re-codes will only exist in your \emph{R} data
frame, not in your saved tab-delimited-text file. Personally, I think
it's both easier and more useful to use Microsoft \emph{Excel} to create
new columns in your tab-delimited-text file for every re-code or
configuration of factors in a factor group (i.e., levels in a column)
that you might need. That being said, there are definitely situations
where \emph{ad-hoc} re-codes may be preferable.

If you want to change anything in your data frame you can generate an
editable data frame in a popup window with the following function:

\begin{Shaded}
\begin{Highlighting}[]
\FunctionTok{fix}\NormalTok{(td)}
\end{Highlighting}
\end{Shaded}

I don't recommend using this method. Like \texttt{file.choose()} and
\texttt{choose.files()}, it introduces non-replicability because the
changes you make using \texttt{fix()} are not recorded in your script
file and therefore cannot be automatically replicated. A better practice
for re-coding while maintaining replicability is to specify the cells in
a column that include the specific values you want to change, and then
to reassign a new value to those cells:

\begin{Shaded}
\begin{Highlighting}[]
\NormalTok{td}\SpecialCharTok{$}\NormalTok{After[td}\SpecialCharTok{$}\NormalTok{After }\SpecialCharTok{==} \StringTok{"H"}\NormalTok{] }\OtherTok{\textless{}{-}} \StringTok{"Consonant"}
\NormalTok{td}\SpecialCharTok{$}\NormalTok{After }\OtherTok{\textless{}{-}} \FunctionTok{factor}\NormalTok{(td}\SpecialCharTok{$}\NormalTok{After)}
\end{Highlighting}
\end{Shaded}

The code above uses the \texttt{\textless{}-} operator to say that any
cells in the column \texttt{After} that contain the value \texttt{"H"}
should be changed to \texttt{"Consonant"} --- another value in the
\texttt{After} column. Many studies of (t, d) deletion do not
distinguish between pre-/h/ and other pre-consonantal contexts. The
above re-code might be needed for comparing this (t, d) deletion data to
other studies. Notice that there is no comma following \texttt{"H"} in
the above code. This is because you are filtering a column; see how the
filtering brackets \texttt{{[}\ {]}} come after the column specifier
\texttt{\$}. The comma is only used when filtering whole data frames
because data frames can be filtered along two dimensions (e.g., rows and
columns) and \emph{R} needs to know which dimensions the filtering
conditions apply to. When you filter just a column (or just a row), you
don't need the comma because the filtering only occurs along one
dimension (in that column, or in that row only). You also need to run
the second \texttt{factor()} function to get rid of the now empty
\texttt{"H"} level.

If you get an error message when trying to re-code using this method and
your column contains words (rather than numbers) try first
re-classifying (i.e., changing the type of) your original column to a
\emph{character} column:
\texttt{td\$After.New\ \textless{}-\ as.character(td\$After.New)}, then
proceeding with the above method.

The previous function sequence re-codes all \texttt{"H"} cells in the
existing \texttt{After} column. If instead you wanted to create a new
column with your re-code (so both possible coding options were available
for later analyses), you could do so by creating a new column with the
exact same values as \texttt{After} and then re-code that column. If
you've been following along in \emph{R}, your \texttt{After} column is
already re-coded. To go back to the original form of the data as it
exists in the tab-delimited-text file, simply reload that text file and
assign it to the object \texttt{td}. Of course, this resets the deletion
of the tokens following a vowel, so you must do that again too. Luckily,
you've written all of your functions in a script file (rather than
directly into the console) so this is easy to do: just highlight the
code and press the execution command.

\begin{Shaded}
\begin{Highlighting}[]
\CommentTok{\# Start with a fresh import of the (t, d) data}
\CommentTok{\# into R}
\NormalTok{td }\OtherTok{\textless{}{-}} \FunctionTok{read.delim}\NormalTok{(}\StringTok{"Data/deletiondata.txt"}\NormalTok{)}
\CommentTok{\# Convert each character column into a factor}
\NormalTok{td}\SpecialCharTok{$}\NormalTok{Dep.Var }\OtherTok{\textless{}{-}} \FunctionTok{factor}\NormalTok{(td}\SpecialCharTok{$}\NormalTok{Dep.Var)}
\NormalTok{td}\SpecialCharTok{$}\NormalTok{Stress }\OtherTok{\textless{}{-}} \FunctionTok{factor}\NormalTok{(td}\SpecialCharTok{$}\NormalTok{Stress)}
\NormalTok{td}\SpecialCharTok{$}\NormalTok{Category }\OtherTok{\textless{}{-}} \FunctionTok{factor}\NormalTok{(td}\SpecialCharTok{$}\NormalTok{Category)}
\NormalTok{td}\SpecialCharTok{$}\NormalTok{Morph.Type }\OtherTok{\textless{}{-}} \FunctionTok{factor}\NormalTok{(td}\SpecialCharTok{$}\NormalTok{Morph.Type)}
\NormalTok{td}\SpecialCharTok{$}\NormalTok{Before }\OtherTok{\textless{}{-}} \FunctionTok{factor}\NormalTok{(td}\SpecialCharTok{$}\NormalTok{Before)}
\NormalTok{td}\SpecialCharTok{$}\NormalTok{After }\OtherTok{\textless{}{-}} \FunctionTok{factor}\NormalTok{(td}\SpecialCharTok{$}\NormalTok{After)}
\NormalTok{td}\SpecialCharTok{$}\NormalTok{Speaker }\OtherTok{\textless{}{-}} \FunctionTok{factor}\NormalTok{(td}\SpecialCharTok{$}\NormalTok{Speaker)}
\NormalTok{td}\SpecialCharTok{$}\NormalTok{Sex }\OtherTok{\textless{}{-}} \FunctionTok{factor}\NormalTok{(td}\SpecialCharTok{$}\NormalTok{Sex)}
\NormalTok{td}\SpecialCharTok{$}\NormalTok{Education }\OtherTok{\textless{}{-}} \FunctionTok{factor}\NormalTok{(td}\SpecialCharTok{$}\NormalTok{Education)}
\NormalTok{td}\SpecialCharTok{$}\NormalTok{Job }\OtherTok{\textless{}{-}} \FunctionTok{factor}\NormalTok{(td}\SpecialCharTok{$}\NormalTok{Job)}
\NormalTok{td}\SpecialCharTok{$}\NormalTok{Phoneme.Dep.Var }\OtherTok{\textless{}{-}} \FunctionTok{factor}\NormalTok{(td}\SpecialCharTok{$}\NormalTok{Phoneme.Dep.Var)}
\CommentTok{\# Subset data to remove previous \textquotesingle{}Vowel\textquotesingle{} contexts}
\NormalTok{td }\OtherTok{\textless{}{-}}\NormalTok{ td[td}\SpecialCharTok{$}\NormalTok{Before }\SpecialCharTok{!=} \StringTok{"Vowel"}\NormalTok{, ]}
\NormalTok{td}\SpecialCharTok{$}\NormalTok{Before }\OtherTok{\textless{}{-}} \FunctionTok{factor}\NormalTok{(td}\SpecialCharTok{$}\NormalTok{Before)}
\CommentTok{\# Re{-}code \textquotesingle{}H\textquotesingle{} to be \textquotesingle{}Consonant\textquotesingle{} in a new column}
\NormalTok{td}\SpecialCharTok{$}\NormalTok{After.New }\OtherTok{\textless{}{-}}\NormalTok{ td}\SpecialCharTok{$}\NormalTok{After}
\NormalTok{td}\SpecialCharTok{$}\NormalTok{After.New[td}\SpecialCharTok{$}\NormalTok{After.New }\SpecialCharTok{==} \StringTok{"H"}\NormalTok{] }\OtherTok{\textless{}{-}} \StringTok{"Consonant"}
\NormalTok{td}\SpecialCharTok{$}\NormalTok{After.New }\OtherTok{\textless{}{-}} \FunctionTok{factor}\NormalTok{(td}\SpecialCharTok{$}\NormalTok{After.New)}
\end{Highlighting}
\end{Shaded}

The new column you create (\texttt{After.New}) is added at the end of
the data frame. You can conceptualized this as the right edge of the
data in an \emph{Excel} spreadsheet. You can see this with the
\texttt{summary()} or \texttt{str()} functions. The name of the new
column you create doesn't really matter, but it cannot include any
spaces.

\begin{Shaded}
\begin{Highlighting}[]
\FunctionTok{summary}\NormalTok{(td)}
\end{Highlighting}
\end{Shaded}

\begin{verbatim}
     Dep.Var           Stress         Category        Morph.Type 
 Deletion:386   Stressed  :1047   Function:  57   Mono     :762  
 Realized:803   Unstressed: 142   Lexical :1132   Past     :311  
                                                  Semi-Weak:116  
                                                                 
                                                                 
                                                                 
                                                                 
             Before          After        Speaker         YOB       Sex    
 Liquid         :269   Consonant:215   INGM84 : 57   Min.   :1915   F:659  
 Nasal          :209   H        :157   MARM92 : 53   1st Qu.:1952   M:530  
 Other Fricative:130   Pause    :558   GARF87 : 52   Median :1984          
 S              :332   Vowel    :259   HUNF22 : 32   Mean   :1969          
 Stop           :249                   DONF15 : 28   3rd Qu.:1991          
                                       GARF37 : 28   Max.   :1999          
                                       (Other):939                         
        Education        Job          Phoneme.Dep.Var     After.New  
 Educated    :496   Blue   :130   t--T        :352    Consonant:372  
 Not Educated:317   Service:468   t--Deletion :296    Pause    :558  
 Student     :376   Student:376   t--Fricative:131    Vowel    :259  
                    White  :215   d--Deletion : 90                   
                                  d--T        : 73                   
                                  d--D        : 46                   
                                  (Other)     :201                   
\end{verbatim}

\begin{Shaded}
\begin{Highlighting}[]
\FunctionTok{str}\NormalTok{(td)}
\end{Highlighting}
\end{Shaded}

\begin{verbatim}
'data.frame':   1189 obs. of  13 variables:
 $ Dep.Var        : Factor w/ 2 levels "Deletion","Realized": 2 1 1 1 2 1 1 1 1 1 ...
 $ Stress         : Factor w/ 2 levels "Stressed","Unstressed": 1 1 1 1 1 1 1 1 1 1 ...
 $ Category       : Factor w/ 2 levels "Function","Lexical": 2 2 2 2 2 2 2 2 2 2 ...
 $ Morph.Type     : Factor w/ 3 levels "Mono","Past",..: 1 1 1 1 1 1 1 1 1 1 ...
 $ Before         : Factor w/ 5 levels "Liquid","Nasal",..: 5 5 5 5 5 5 5 5 5 5 ...
 $ After          : Factor w/ 4 levels "Consonant","H",..: 1 1 1 1 1 1 1 1 1 1 ...
 $ Speaker        : Factor w/ 66 levels "ARSM91","BEAM91",..: 3 5 5 6 13 14 15 23 29 33 ...
 $ YOB            : int  1965 1955 1955 1952 1953 1958 1946 1942 1945 1949 ...
 $ Sex            : Factor w/ 2 levels "F","M": 1 1 1 1 2 2 1 2 2 1 ...
 $ Education      : Factor w/ 3 levels "Educated","Not Educated",..: 1 1 1 1 1 2 1 2 2 1 ...
 $ Job            : Factor w/ 4 levels "Blue","Service",..: 4 4 4 2 2 2 2 1 1 2 ...
 $ Phoneme.Dep.Var: Factor w/ 18 levels "d--Affricate",..: 18 12 12 12 18 12 12 12 12 12 ...
 $ After.New      : Factor w/ 3 levels "Consonant","Pause",..: 1 1 1 1 1 1 1 1 1 1 ...
\end{verbatim}

\hypertarget{centering-continuous-variables}{%
\subsection{Centering Continuous
Variables}\label{centering-continuous-variables}}

Some variables, like year of birth, are not discrete but are continuous.
Some statisticians advocate centering continuous variables before
including them in certain tests or models. In the Regression Analysis
section you'll learn when/if you need to center your variables.

\emph{Centering} entails expressing each value of a continuous variable
as that value's difference from the mean of all values of the variable.
For example, the \emph{td} data frame has a column for speaker year of
birth: \texttt{YOB}. The mean of all the years of birth (after the
pre-vowel tokens are removed) is \texttt{1969}. Centering this variable
simply means expressing years of birth like 1952 and 1989, as 17
(\(=1969-1952\)) and -20 (\(=1969-1989\)).

\begin{Shaded}
\begin{Highlighting}[]
\CommentTok{\# Center YOB}
\NormalTok{td}\SpecialCharTok{$}\NormalTok{Center.Age }\OtherTok{\textless{}{-}} \FunctionTok{scale}\NormalTok{(td}\SpecialCharTok{$}\NormalTok{YOB, }\AttributeTok{scale =} \ConstantTok{FALSE}\NormalTok{)}
\NormalTok{td}\SpecialCharTok{$}\NormalTok{Center.Age }\OtherTok{\textless{}{-}} \FunctionTok{as.numeric}\NormalTok{(td}\SpecialCharTok{$}\NormalTok{Center.Age)}
\end{Highlighting}
\end{Shaded}

\begin{Shaded}
\begin{Highlighting}[]
\NormalTok{td}\SpecialCharTok{$}\NormalTok{Center.Age }\OtherTok{\textless{}{-}} \FunctionTok{as.numeric}\NormalTok{(}\FunctionTok{scale}\NormalTok{(td}\SpecialCharTok{$}\NormalTok{YOB, }\AttributeTok{scale =} \ConstantTok{FALSE}\NormalTok{))}
\end{Highlighting}
\end{Shaded}

The \texttt{scale()} function centers the values in the column
\texttt{YOB} and assigns those Centered values to a new column called
\texttt{Center.Age}. The option \texttt{scale\ =\ FALSE} indicates that
the centered values remain in the original units (in this case years).
If you change this option to \texttt{scale\ =\ TRUE}, the overall mean
is subtracted from each value and then each values is divided by the
overall standard deviation. This is needed if you are including multiple
continuous variables in a model that are expressed using different units
and vary along differing scales. When the data is centred and scalled
the new values are called \textbf{z-scores} or \textbf{standardized
scores} and you can think of each value as describing the raw value in
terms of its distance from the mean when measured in standard deviation
units.

For example, imagine a study of variable word-initial voice onset time
in which you want to look at the effect of following vowel backness and
year of birth of the speaker. Each of these variables are continuous,
but voice onset time is \(\pm20\) ms around its mean, year of birth is
\(\pm45\) years around its mean, and F2 is \(\pm300\) Hz around its
mean. By expressing these values as \textbf{z-scores}, you can account
for both the differing units and differing scales of each. Even if you
have one variable, here just \texttt{YOB}, it is okay to scale the
values, but do \emph{NOT} do this here. Leave \texttt{YOB} in years.
This will make the interpretation of later statistical estimates a
little easier.

After running your \texttt{scale()} function you must also run a
function to tell \emph{R} how to treat your new column
\texttt{Center.Age.} \emph{R} knows the column is filled with numbers,
but \emph{R} doesn't know if those numbers are continuous, if they are
ordered in a specific way, or if they are factors with names expressed
as digits. You use the function \texttt{as.numeric()} to tell \emph{R}
that the values are continuous numbers. This is very similar to what you
did above with the function \texttt{factor()}, which tells \emph{R} to
consider whatever is inside the function to be a factor. Also, with
\texttt{After.New} above, where the data structure is inherited from
\texttt{After}, you used \texttt{factor()} to remove the remaining
\texttt{H} level name. You did this by telling \texttt{R} to consider
whatever was inside the \texttt{factor()} function as a factor, and, as
part of that, \texttt{R} reads all the levels inside that column and
chooses them as the factors. Here you are doing the same thing. You're
telling \texttt{R} to look at the values inside \texttt{Center.Age} and
consider them as being continuous numbers. You can do this as a
secondary step after you create the column \texttt{Center.Age}, as shown
in the first two rows. Or you can embed the \texttt{scale()} function
inside the \texttt{as.numeric()} function, \texttt{as.numeric(scale())},
as shown in the last line.

\hypertarget{dividing-a-continuous-variable}{%
\subsection{Dividing a Continuous
Variable}\label{dividing-a-continuous-variable}}

There are other ways you can represent age. Instead of a continuous
variable, you can categorize speakers as belonging to a ``young'',
``middle aged'', or ``old'' age group. You might also group speakers
based on decade of birth or as being born before or after some event ---
whatever makes sense for your study or community.

How to divide speakers by age is something that should be informed by
sociolinguistic theory, demographics, and your own understanding of your
data. The birth years 1980 and 1945, used here, represent generational
divides in Cape Breton that can be independently justified (Gardner
2013). Grouping speakers like this is very common, and not particularly
difficult to do in \emph{R}.

\begin{Shaded}
\begin{Highlighting}[]
\CommentTok{\# Create a 3{-}way Age Group}
\NormalTok{td}\SpecialCharTok{$}\NormalTok{Age.Group[td}\SpecialCharTok{$}\NormalTok{YOB }\SpecialCharTok{\textgreater{}} \DecValTok{1979}\NormalTok{] }\OtherTok{\textless{}{-}} \StringTok{"Young"}
\NormalTok{td}\SpecialCharTok{$}\NormalTok{Age.Group[td}\SpecialCharTok{$}\NormalTok{YOB }\SpecialCharTok{\textgreater{}} \DecValTok{1944} \SpecialCharTok{\&}\NormalTok{ td}\SpecialCharTok{$}\NormalTok{YOB }\SpecialCharTok{\textless{}} \DecValTok{1980}\NormalTok{] }\OtherTok{\textless{}{-}} \StringTok{"Middle"}
\NormalTok{td}\SpecialCharTok{$}\NormalTok{Age.Group[td}\SpecialCharTok{$}\NormalTok{YOB }\SpecialCharTok{\textless{}} \DecValTok{1945}\NormalTok{] }\OtherTok{\textless{}{-}} \StringTok{"Old"}
\end{Highlighting}
\end{Shaded}

First you will need a new column for your age group variable. Here, call
it \texttt{Age.Group}.You use the assignment operator
\texttt{\textless{}-} to fill all the cells in the new column
\texttt{Age.Group} based on the values that are in the already existing
\texttt{YOB} column. The first line says that for any rows in which
\texttt{YOB} is greater than 1979, fill the empty cell in those same
rows in the column \texttt{Age.Group} with the value \texttt{Young}. The
second line does the same thing but includes two conditions: that
\texttt{YOB} is greater than 1944 and that it is also less than 1980.
For these rows, the value \texttt{Middle} is inserted in the
\texttt{Age.Group} column. Even though the two conditions
(\texttt{YOB\ \textgreater{}\ 1944}, \texttt{YOB\ \textless{}\ 1980})
refer to the same column, you need to fully specify each condition with
both a column reference and a condition with an operator. So, for
example, writing
\texttt{td\$Age.Group{[}td\$YOB\ \textgreater{}1944\ \&\ \textless{}1970{]}}
will not work. The third line instructs \texttt{R} to put \texttt{Old}
in the \texttt{Age.Group} column for any row where \texttt{YOB} is less
than 1945. Because this new column includes words, \emph{R} will
automatically categorize the column as \texttt{character} data. To
rectify this, see the next section.

\begin{Shaded}
\begin{Highlighting}[]
\FunctionTok{class}\NormalTok{(td}\SpecialCharTok{$}\NormalTok{Age.Group)}
\end{Highlighting}
\end{Shaded}

\begin{verbatim}
[1] "character"
\end{verbatim}

\hypertarget{changing-the-order-of-levels}{%
\subsection{Changing the Order of
Levels}\label{changing-the-order-of-levels}}

\begin{Shaded}
\begin{Highlighting}[]
\NormalTok{td}\SpecialCharTok{$}\NormalTok{Age.Group }\OtherTok{\textless{}{-}} \FunctionTok{factor}\NormalTok{(td}\SpecialCharTok{$}\NormalTok{Age.Group, }\AttributeTok{levels =} \FunctionTok{c}\NormalTok{(}\StringTok{"Young"}\NormalTok{,}
    \StringTok{"Middle"}\NormalTok{, }\StringTok{"Old"}\NormalTok{))}
\end{Highlighting}
\end{Shaded}

The code above does two things. First, it tells \texttt{R} that you want
the new \texttt{Age.Group} column to be a column of factors, and second,
it also tells \texttt{R} how you want those factors to be ordered. When
\texttt{R} extracts the name of the factors of a column (based on the
levels that are in that column) it orders the factors by name
alphabetically. For example, the \texttt{Sex} column contains two
levels: \texttt{M} and \texttt{F}. The first 500 tokens in your data
frame might comes from males but \texttt{R} will still list the names of
the factors in the column (which are based on those two levels) as
\texttt{F} and then \texttt{M}, because \texttt{F} is closer to the
start of the alphabet than \texttt{M}. If you run \texttt{summary(td)}
you can see that the factor names listed in all of the factor columns
are in alphabetical order. Sometimes this alphabetical ordering doesn't
matter. Other times it is makes a big difference.

Any time factors are used in a statistical test or appear in a graph,
the order of factor names is very important. For example, which factor
is selected as the application value and which factor(s) are the
non-application value(s) of a dependent variable is determined by factor
name order. As for graphs, all kinds of layout parameters are set by the
factor order of a variable. For example, if you left the
\texttt{Age.Group} factor as it is, it would always list the
\texttt{Age.Group} levels as \texttt{Middle}, \texttt{Old},
\texttt{Young}. On a graph like a bar graph it would arrange the bars
for each age group alphabetically from left to right. This is not
desirable. You will always want \texttt{Age.Group} ordered as either
\texttt{Young}, \texttt{Middle}, \texttt{Old} or \texttt{Old},
\texttt{Middle}, \texttt{Young}. This is what the second part of the
\texttt{factor()} command does. First it tells \texttt{R} to consider
the values in the \texttt{Age.Group} to be factors, then it specifies
that you want the factors (which get their names from the levels) to be
ordered in certain way. You use \texttt{levels\ =} and the concatenating
\texttt{c()} function to specify that you want the factors to get their
names from the levels in the following order
\texttt{"Young",\ "Middle",\ "Old"}, and thus also be ordered in that
way.

This might seem confusing. Think about coins. Imagine you have a bag of
change. Each coin in that bag is like a token in your column in the data
frame. The levels of the column are just the different kinds of coins
there are. You can root around in the bag and figure out what coins are
in it without having to put the coins in any particular order. This is
the ``levels'' of the coins. If you ask \emph{R} what coins are in your
bag, it will tell you there are ``dimes'', ``nickels'', ``pennies'', and
``quarters''. These levels are in alphabetical order, but only for a
lack of a better way to tell you them. Telling you the levels doesn't
imply any sort of ordering of the coins. Making the coin bag into a
factor column is like putting the coins into a change tray, like the
change trays in cash registers. Coins inside of a change tray go from
being unstructured to being organized based on a structure. This makes
them factors. In a cash tray each coin is grouped with other coins just
like it. Each slot in the drawer also has a name. \emph{R} just
automatically gives the slots the same name as the coins (e.g.~levels)
that are in it. The order of the slots from right to left is also by
default alphabetical based on the name of each slot. So, coin (= level),
slot name (= factor name) and slot order (= factor order) are three
independent parameters. The function
\texttt{td\$Age.Group\ \textless{}-factor(td\$Age.Group,\ levels\ =\ c("Young",\ "Middle",\ "Old"))}
is basically saying take all the coins out of the cash drawer, then put
them back in the drawer in in a specific order, with \texttt{Young}
going in the first slot. The slots then take the names of the coins that
go into them.

\hypertarget{reversing-the-order-of-levels}{%
\subsection{Reversing the Order of
Levels}\label{reversing-the-order-of-levels}}

To reverse the order of levels you can embed the \texttt{rev()} function
inside your \texttt{factor()} function

\begin{Shaded}
\begin{Highlighting}[]
\FunctionTok{levels}\NormalTok{(td}\SpecialCharTok{$}\NormalTok{Age.Group)}
\end{Highlighting}
\end{Shaded}

\begin{verbatim}
[1] "Young"  "Middle" "Old"   
\end{verbatim}

\begin{Shaded}
\begin{Highlighting}[]
\NormalTok{td}\SpecialCharTok{$}\NormalTok{Age.Group }\OtherTok{\textless{}{-}} \FunctionTok{factor}\NormalTok{(td}\SpecialCharTok{$}\NormalTok{Age.Group, }\AttributeTok{levels =} \FunctionTok{rev}\NormalTok{(}\FunctionTok{levels}\NormalTok{(td}\SpecialCharTok{$}\NormalTok{Age.Group)))}
\FunctionTok{levels}\NormalTok{(td}\SpecialCharTok{$}\NormalTok{Age.Group)}
\end{Highlighting}
\end{Shaded}

\begin{verbatim}
[1] "Old"    "Middle" "Young" 
\end{verbatim}

\hypertarget{combining-columns}{%
\subsection{Combining Columns}\label{combining-columns}}

It is often useful to combine two columns, or factor groups, into one.
For example, it might be useful to have a way of grouping tokens not by
\texttt{Age.Group} or \texttt{Sex}, but instead by the combination of
\texttt{Age.Group} and \texttt{Sex}. While it is not difficult to test
the potential interaction of these two factor groups in statistical
tests in \emph{R}, for specifically generating summary statistics it is
much easier to examine the combination of \texttt{Age.Group} and
\texttt{Sex} (or any two columns) by first creating a new column
combining them.

Combining columns is done using the function \texttt{paste()}, which you
embed inside of the \texttt{factor()} function so that the resulting
column will be a column of factors. Inside \texttt{paste()} you list the
two columns you want to combine (you could list more) and then tell
\emph{R} how to separate the values from each of the columns. Here the
code tells \emph{R} to put an underscore *\_* between the values for
\texttt{Age.Group} and \texttt{Sex}, resulting in new values like
\texttt{Old\_M} and \texttt{Middle\_F}. The new column will be called
\texttt{Age\_Sex}. If we want to order these values in any particular
way, we can do that with the \texttt{level=} option in \texttt{factor()}

\begin{Shaded}
\begin{Highlighting}[]
\CommentTok{\# Combine Columns}
\NormalTok{td}\SpecialCharTok{$}\NormalTok{Age\_Sex }\OtherTok{\textless{}{-}} \FunctionTok{factor}\NormalTok{(}\FunctionTok{paste}\NormalTok{(td}\SpecialCharTok{$}\NormalTok{Age.Group, td}\SpecialCharTok{$}\NormalTok{Sex, }\AttributeTok{sep =} \StringTok{"\_"}\NormalTok{),}
    \AttributeTok{levels =} \FunctionTok{c}\NormalTok{(}\StringTok{"Young\_F"}\NormalTok{, }\StringTok{"Middle\_F"}\NormalTok{, }\StringTok{"Old\_F"}\NormalTok{, }\StringTok{"Young\_M"}\NormalTok{,}
        \StringTok{"Middle\_M"}\NormalTok{, }\StringTok{"Old\_M"}\NormalTok{))}
\FunctionTok{levels}\NormalTok{(td}\SpecialCharTok{$}\NormalTok{Age\_Sex)}
\end{Highlighting}
\end{Shaded}

\begin{verbatim}
[1] "Young_F"  "Middle_F" "Old_F"    "Young_M"  "Middle_M" "Old_M"   
\end{verbatim}

\begin{Shaded}
\begin{Highlighting}[]
\CommentTok{\# Reorder factor levels}
\NormalTok{td}\SpecialCharTok{$}\NormalTok{Age\_Sex }\OtherTok{\textless{}{-}} \FunctionTok{factor}\NormalTok{(td}\SpecialCharTok{$}\NormalTok{Age\_Sex, }\AttributeTok{levels =} \FunctionTok{c}\NormalTok{(}\StringTok{"Young\_F"}\NormalTok{,}
    \StringTok{"Young\_M"}\NormalTok{, }\StringTok{"Middle\_F"}\NormalTok{, }\StringTok{"Middle\_M"}\NormalTok{, }\StringTok{"Old\_F"}\NormalTok{, }\StringTok{"Old\_M"}\NormalTok{))}
\FunctionTok{levels}\NormalTok{(td}\SpecialCharTok{$}\NormalTok{Age\_Sex)}
\end{Highlighting}
\end{Shaded}

\begin{verbatim}
[1] "Young_F"  "Young_M"  "Middle_F" "Middle_M" "Old_F"    "Old_M"   
\end{verbatim}

\hypertarget{splitting-columns}{%
\subsection{Splitting Columns}\label{splitting-columns}}

You may have noticed that the data frame \texttt{td} has one column that
is actually two variables. The column \texttt{Phoneme.Dep.Var} combines
both the underlying phoneme of the token, either \texttt{t} (t) or
\texttt{d} (d), with a more descriptive coding of the dependent
variable. In the dialect where this data comes from (t) and (d) can be
realized in up to nine different ways, only one of which is
\texttt{Deletion}. In your analysis you might want to consider if
deletion is more likely if the underlying phoneme is (t) or (d). In
order to do this you must break \texttt{Phoneme} away from the more
elaborate \texttt{Dep.Var} coding. To do this, you will use a function
that is beyond \emph{R}'s base functionality and part of the
\texttt{dplyr} package. First you load the \texttt{dplyr} package from
your library using \texttt{library()}, then you use the function
\texttt{mutate()} to tell \emph{R} how to break up the column. The
\texttt{dplyr} package is very powerful. We will look at additional uses
as this guide progresses.

To be honest, splitting columns is an operation that is much simpler and
usually faster to do using \emph{Excel}, or another spreadsheet program.
I never use \emph{R} to split columns because, while the
\texttt{mutate()} function itself is not tricky, figuring out the exact
sequence of \texttt{regular\ expressions} I need to split my columns is
challenging. If you absolutely must split columns in \emph{R}, below are
instructions. Keep in mind that this method will only work for columns
with predictable structures, like \texttt{Phoneme.Dep.Var}.

The values in the column \texttt{Phoneme.Dep.Var} are predictable, in
other words, they all follow the same pattern. You can exploit this
regularity to tell \emph{R} exactly where to break the
\texttt{Phoneme.Dep.Var} values into two. If you execute the command
\texttt{levels(td\$Phoneme.Dep.Var)} you can very easily see the
pattern.

\begin{Shaded}
\begin{Highlighting}[]
\FunctionTok{levels}\NormalTok{(td}\SpecialCharTok{$}\NormalTok{Phoneme.Dep.Var)}
\end{Highlighting}
\end{Shaded}

\begin{verbatim}
 [1] "d--Affricate"       "d--D"               "d--Deletion"       
 [4] "d--Ejective"        "d--Flap"            "d--Fricative"      
 [7] "d--Glottal Stop"    "d--Other Weakening" "d--T"              
[10] "t--Affricate"       "t--D"               "t--Deletion"       
[13] "t--Ejective"        "t--Flap"            "t--Fricative"      
[16] "t--Glottal Stop"    "t--Other Weakening" "t--T"              
\end{verbatim}

The first part of every value is either \texttt{t} or \texttt{d}, then
there are two hyphens, then a word describing the realization of (t) or
(d). If you split \texttt{Phoneme.Dep.Var} at the hyphens you'll be left
with two values: one that is either \texttt{t} or \texttt{d}, and
another that describes the realization of (t) or (d). So, for the new
\texttt{Phoneme} column you want the one character before the hyphens
from \texttt{Phoneme.Dep.Var}, and for the new \texttt{Dep.Var.Full}
column, you want all the characters --- however many there are -- that
come after the two hyphens.

\begin{Shaded}
\begin{Highlighting}[]
\CommentTok{\# Break Phoneme.Dep.Var into two columns}
\FunctionTok{library}\NormalTok{(dplyr)}
\NormalTok{td }\OtherTok{\textless{}{-}} \FunctionTok{mutate}\NormalTok{(td, }\AttributeTok{Phoneme =} \FunctionTok{sub}\NormalTok{(}\StringTok{"\^{}(.)({-}{-}.*)$"}\NormalTok{, }\StringTok{"}\SpecialCharTok{\textbackslash{}\textbackslash{}}\StringTok{1"}\NormalTok{,}
\NormalTok{    Phoneme.Dep.Var), }\AttributeTok{Dep.Var.Full =} \FunctionTok{sub}\NormalTok{(}\StringTok{"\^{}(.{-}{-})(.*)$"}\NormalTok{,}
    \StringTok{"}\SpecialCharTok{\textbackslash{}\textbackslash{}}\StringTok{2"}\NormalTok{, Phoneme.Dep.Var), }\AttributeTok{Phoneme.Dep.Var =} \ConstantTok{NULL}\NormalTok{)}
\NormalTok{td}\SpecialCharTok{$}\NormalTok{Dep.Var.Full }\OtherTok{\textless{}{-}} \FunctionTok{factor}\NormalTok{(td}\SpecialCharTok{$}\NormalTok{Dep.Var.Full)}
\NormalTok{td}\SpecialCharTok{$}\NormalTok{Phoneme }\OtherTok{\textless{}{-}} \FunctionTok{factor}\NormalTok{(td}\SpecialCharTok{$}\NormalTok{Phoneme)}
\end{Highlighting}
\end{Shaded}

\begin{tcolorbox}[enhanced jigsaw, breakable, colframe=quarto-callout-note-color-frame, opacitybacktitle=0.6, colbacktitle=quarto-callout-note-color!10!white, leftrule=.75mm, coltitle=black, bottomrule=.15mm, bottomtitle=1mm, toprule=.15mm, left=2mm, title=\textcolor{quarto-callout-note-color}{\faInfo}\hspace{0.5em}{Note}, toptitle=1mm, opacityback=0, titlerule=0mm, arc=.35mm, rightrule=.15mm, colback=white]
The procedure here is somewhat complicated. I always have to
double-check how these complicated procedures work. Either by looking up
the functions or checking my old script files. You can look up how a
function like \texttt{mutate()} works by placing a question mark before
the function in the console window, e.g.~\texttt{?mutate()}.
\end{tcolorbox}

Above, inside the \texttt{mutate()} function, first you specify the data
frame object you want to mutate (here \texttt{td}), and then how to
mutate it. The mutations include creating a new column called
\texttt{Phoneme} and then telling \emph{R} what do put in it, creating a
new column called \texttt{Dep.Var.Full} and then telling \emph{R} what
to put in it, and then taking the \texttt{Phoneme.Dep.Var} column and
making it equal \texttt{NULL} --- in other words, deleting it. For the
\texttt{Phoneme} and \texttt{Dep.Var.Full} columns you specify what
values to insert in each cell using the \texttt{sub()} function, which
returns a ``sub'' or divided section of a value within some cell. The
\texttt{sub()} function's first argument is a pattern of \emph{regular
expressions} to search for, the second is a character string to replace
the pattern with, and the third is the column in which to search for the
pattern.

\begin{tcolorbox}[enhanced jigsaw, breakable, colframe=quarto-callout-note-color-frame, opacitybacktitle=0.6, colbacktitle=quarto-callout-note-color!10!white, leftrule=.75mm, coltitle=black, bottomrule=.15mm, bottomtitle=1mm, toprule=.15mm, left=2mm, title=\textcolor{quarto-callout-note-color}{\faInfo}\hspace{0.5em}{Note}, toptitle=1mm, opacityback=0, titlerule=0mm, arc=.35mm, rightrule=.15mm, colback=white]
I don't have \emph{R}'s regular expressions memorized; I always have to
look them up
\href{https://stat.ethz.ch/R-manual/R-devel/library/base/html/regex.html}{here}.
\end{tcolorbox}

For the \texttt{sub()} functions meant to create values for
\texttt{Phoneme} and for \texttt{Dep.Var.Full} you tell \emph{R} to
search \texttt{Phoneme.Dep.Var} for the pattern
\texttt{"\^{}.-\/-.*\$"}. This series of regular expressions describes
the following pattern: a text string beginning with one single
character, followed by two hyphens, and then any number of characters.
The quotation marks \texttt{"\ "} indicate a text string. The
\texttt{\^{}} and \texttt{\$} indicate the beginning and end of a text
string, respectively.\footnote{You use \texttt{\$} in other places for
  specifying columns within data frames, but here it's inside quotation
  marks and serving a different function.} The period \texttt{.}
indicates a single character and the asterisk\texttt{*} indicates zero
or more of whatever comes before that asterisk. So \texttt{.*} means any
one or more characters. The two hyphens \texttt{-\/-} are just two
hyphens. The parentheses in the regular expressions relate to the second
substitution element. In the \texttt{sub()} function responsible for
creating values for the new \texttt{Phoneme} column, you break the
values that match the regular expression pattern into two parts: the one
character before the two hyphens \texttt{(.)} and then everything after
it \texttt{(-\/-.*)}; then you tell \emph{R} to substitute (or rather
place) the first of those two elements
\texttt{\textbackslash{}\textbackslash{}1} as text \texttt{"\ "} in the
new column. In the \texttt{sub()} function responsible for creating the
values for the new \texttt{Dep.Var.Full} column, you break the values
that match the regular expression pattern into two parts: the two
hyphens and the one character that comes before them \texttt{(.-\/-)}
and then the one or more characters that come after the hyphens
\texttt{(.*)}; then tell \emph{R} to to substitute (or rather place) the
second of those elements \texttt{\textbackslash{}\textbackslash{}2} as
text \texttt{"\ "} in the new column.

The last two lines simply make these two new columns into factor
columns.

\hypertarget{partitioning-data-frames}{%
\subsection{Partitioning Data Frames}\label{partitioning-data-frames}}

Many times in my own work I have only wanted to work in \emph{R} with a
subset of my full dataset. For example, I frequently want to run
separate regression analyses on data from old speakers, middle age
speakers, and young speakers. Other times I've had large datasets that
combine multiple corpora and have wanted to run tests on data from just
one corpus. Being able to partition (subset) my data frame has therefore
been very useful.

There are two main ways to partition your data. One method involves
using the filtering functionality of \emph{R} detailed above. For
example, the first line code below represents the \texttt{td} data from
\texttt{young} speakers only. The second line of codes assigns that
subset of \texttt{td} to a completely new data frame.

\begin{Shaded}
\begin{Highlighting}[]
\CommentTok{\# Subset of td where Age.Group only equals Young}
\NormalTok{td[td}\SpecialCharTok{$}\NormalTok{Age.Group }\SpecialCharTok{==} \StringTok{"Young"}\NormalTok{, ]}
\end{Highlighting}
\end{Shaded}

\begin{Shaded}
\begin{Highlighting}[]
\CommentTok{\# Create td.young which equals subset of td where}
\CommentTok{\# Age. Group only equals Young}
\NormalTok{td.young }\OtherTok{\textless{}{-}}\NormalTok{ td[td}\SpecialCharTok{$}\NormalTok{Age.Group }\SpecialCharTok{==} \StringTok{"Young"}\NormalTok{, ]}
\end{Highlighting}
\end{Shaded}

In your tests you can use the filtered version of \texttt{td} or you can
use the new data frame \texttt{td.young}. I recommend using the new data
frame if you are using any centered continuous variables. This is
because centered continuous variables are centered around the mean of
the values in the column that is centered. If you partition the data,
that mean will be different because the range of values within the the
centered column are now different. So, if you partition your data,
especially by age, you should re-center your continuous variables.

The second way to partition your data is using the \texttt{subset()}
function.

\begin{Shaded}
\begin{Highlighting}[]
\CommentTok{\# Create three partitions based on Age.Group}
\NormalTok{td.young }\OtherTok{\textless{}{-}} \FunctionTok{subset}\NormalTok{(td, Age.Group }\SpecialCharTok{==} \StringTok{"Young"}\NormalTok{)}
\NormalTok{td.middle }\OtherTok{\textless{}{-}} \FunctionTok{subset}\NormalTok{(td, Age.Group }\SpecialCharTok{==} \StringTok{"Middle"}\NormalTok{)}
\NormalTok{td.old }\OtherTok{\textless{}{-}} \FunctionTok{subset}\NormalTok{(td, Age.Group }\SpecialCharTok{==} \StringTok{"Old"}\NormalTok{)}
\CommentTok{\# Re{-}Center Center.Age}
\NormalTok{td.young}\SpecialCharTok{$}\NormalTok{Center.Age }\OtherTok{\textless{}{-}} \FunctionTok{scale}\NormalTok{(td.young}\SpecialCharTok{$}\NormalTok{YOB, }\AttributeTok{scale =} \ConstantTok{FALSE}\NormalTok{)}
\NormalTok{td.middle}\SpecialCharTok{$}\NormalTok{Center.Age }\OtherTok{\textless{}{-}} \FunctionTok{scale}\NormalTok{(td.middle}\SpecialCharTok{$}\NormalTok{YOB, }\AttributeTok{scale =} \ConstantTok{FALSE}\NormalTok{)}
\NormalTok{td.old}\SpecialCharTok{$}\NormalTok{Center.Age }\OtherTok{\textless{}{-}} \FunctionTok{scale}\NormalTok{(td.old}\SpecialCharTok{$}\NormalTok{YOB, }\AttributeTok{scale =} \ConstantTok{FALSE}\NormalTok{)}
\end{Highlighting}
\end{Shaded}

The usefulness of \texttt{subset()} is when you want to partition your
data frame based on several factors. For example, compare the filtering
versus \texttt{subset()} methods when partitioning \texttt{td} for just
middle-age, blue-collar men. Using \texttt{subset()} saves you from
having to type \texttt{td\$} multiple times, and you don't need the
final \texttt{,}. Below \texttt{\&} is the standard operator meaning
``and''.

\begin{Shaded}
\begin{Highlighting}[]
\CommentTok{\# Subset of td where Age.Group is Middle, Sex is}
\CommentTok{\# M, and Job is Blue}
\NormalTok{td.midmenblue }\OtherTok{\textless{}{-}}\NormalTok{ td[td}\SpecialCharTok{$}\NormalTok{Age.Group }\SpecialCharTok{==} \StringTok{"Middle"} \SpecialCharTok{\&}\NormalTok{ td}\SpecialCharTok{$}\NormalTok{Sex }\SpecialCharTok{==}
    \StringTok{"M"} \SpecialCharTok{\&}\NormalTok{ td}\SpecialCharTok{$}\NormalTok{Job }\SpecialCharTok{==} \StringTok{"Blue"}\NormalTok{, ]}
\CommentTok{\# Subset of td where Age.Group is Middle, Sex is}
\CommentTok{\# M, and Job is Blue}
\NormalTok{td.midmenblue }\OtherTok{\textless{}{-}} \FunctionTok{subset}\NormalTok{(td, Age.Group }\SpecialCharTok{==} \StringTok{"Middle"} \SpecialCharTok{\&}
\NormalTok{    Sex }\SpecialCharTok{==} \StringTok{"M"} \SpecialCharTok{\&}\NormalTok{ Job }\SpecialCharTok{==} \StringTok{"Blue"}\NormalTok{)}
\end{Highlighting}
\end{Shaded}

When you subset your data frame, using either method, and assign it to a
new data frame, your new data frame inherits the structure of your old
data frame. This means that there will be lots of columns that list
empty levels. If you execute \texttt{summary(td.midmenblue)\}} you'll
see see that \texttt{Job} still lists \texttt{Service},
\texttt{Student}, and \texttt{White} as factors with 0 tokens each. To
remove these empty levels you could use the \texttt{factor()} function
on each column with empty levels (as you did above), or you can use the
function \texttt{droplevels()} to tell \emph{R} to drop any empty levels
from each column in the data frame. You need to make sure to also assign
the dropped levels data frame to the original data frame, as shown
below.

\begin{Shaded}
\begin{Highlighting}[]
\FunctionTok{summary}\NormalTok{(td.midmenblue)}
\end{Highlighting}
\end{Shaded}

\begin{verbatim}
     Dep.Var          Stress       Category      Morph.Type
 Deletion:30   Stressed  :49   Function: 3   Mono     :38  
 Realized:27   Unstressed: 8   Lexical :54   Past     :15  
                                             Semi-Weak: 4  
                                                           
                                                           
                                                           
                                                           
             Before         After       Speaker        YOB       Sex   
 Liquid         :13   Consonant:16   GREM45 :18   Min.   :1945   F: 0  
 Nasal          : 3   H        : 9   HOLM52 :16   1st Qu.:1945   M:57  
 Other Fricative: 6   Pause    :20   SMIM61 :16   Median :1952         
 S              :18   Vowel    :12   CLAM73 : 4   Mean   :1954         
 Stop           :17                  HANM57 : 3   3rd Qu.:1961         
                                     ARSM91 : 0   Max.   :1973         
                                     (Other): 0                        
        Education       Job         After.New    Center.Age       Age.Group 
 Educated    : 0   Blue   :57   Consonant:25   Min.   :-24.447   Old   : 0  
 Not Educated:57   Service: 0   Pause    :20   1st Qu.:-24.447   Middle:57  
 Student     : 0   Student: 0   Vowel    :12   Median :-17.447   Young : 0  
                   White  : 0                  Mean   :-15.394              
                                               3rd Qu.: -8.447              
                                               Max.   :  3.553              
                                                                            
     Age_Sex   Phoneme    Dep.Var.Full
 Young_F : 0   d:11    Deletion :30   
 Young_M : 0   t:46    T        :19   
 Middle_F: 0           Fricative: 6   
 Middle_M:57           D        : 1   
 Old_F   : 0           Flap     : 1   
 Old_M   : 0           Affricate: 0   
                       (Other)  : 0   
\end{verbatim}

\begin{Shaded}
\begin{Highlighting}[]
\CommentTok{\# Drop empty levels across dataset}
\NormalTok{td.midmenblue }\OtherTok{\textless{}{-}} \FunctionTok{droplevels}\NormalTok{(td.midmenblue)}
\FunctionTok{summary}\NormalTok{(td.midmenblue)}
\end{Highlighting}
\end{Shaded}

\begin{verbatim}
     Dep.Var          Stress       Category      Morph.Type
 Deletion:30   Stressed  :49   Function: 3   Mono     :38  
 Realized:27   Unstressed: 8   Lexical :54   Past     :15  
                                             Semi-Weak: 4  
                                                           
                                                           
                                                           
             Before         After      Speaker        YOB       Sex   
 Liquid         :13   Consonant:16   CLAM73: 4   Min.   :1945   M:57  
 Nasal          : 3   H        : 9   GREM45:18   1st Qu.:1945         
 Other Fricative: 6   Pause    :20   HANM57: 3   Median :1952         
 S              :18   Vowel    :12   HOLM52:16   Mean   :1954         
 Stop           :17                  SMIM61:16   3rd Qu.:1961         
                                                 Max.   :1973         
        Education    Job         After.New    Center.Age       Age.Group 
 Not Educated:57   Blue:57   Consonant:25   Min.   :-24.447   Middle:57  
                             Pause    :20   1st Qu.:-24.447              
                             Vowel    :12   Median :-17.447              
                                            Mean   :-15.394              
                                            3rd Qu.: -8.447              
                                            Max.   :  3.553              
     Age_Sex   Phoneme    Dep.Var.Full
 Middle_M:57   d:11    D        : 1   
               t:46    Deletion :30   
                       Flap     : 1   
                       Fricative: 6   
                       T        :19   
                                      
\end{verbatim}

\hypertarget{interim-summary}{%
\subsection{Interim Summary}\label{interim-summary}}

Below is the full code for loading the data file and then re-coding it.
As you progress through the next sections, if for any reason you have a
problem, it may be useful to recreate the \emph{R} object \texttt{td}
from scratch. Going forward in these instructions, the \texttt{td}
object will be the object as it exists at the end of this code.

\begin{Shaded}
\begin{Highlighting}[]
\NormalTok{td }\OtherTok{\textless{}{-}} \FunctionTok{read.delim}\NormalTok{(}\StringTok{"https://www.dropbox.com/s/jxlfuogea3lx2pu/deletiondata.txt?dl=1"}\NormalTok{)}
\end{Highlighting}
\end{Shaded}

Or\ldots{}

\begin{Shaded}
\begin{Highlighting}[]
\NormalTok{td }\OtherTok{\textless{}{-}} \FunctionTok{read.delim}\NormalTok{(}\StringTok{"Data/deletiondata.txt"}\NormalTok{)}
\end{Highlighting}
\end{Shaded}

Then\ldots{}

\begin{Shaded}
\begin{Highlighting}[]
\CommentTok{\# Subset data to remove previous \textquotesingle{}Vowel\textquotesingle{} contexts}
\NormalTok{td }\OtherTok{\textless{}{-}}\NormalTok{ td[td}\SpecialCharTok{$}\NormalTok{Before }\SpecialCharTok{!=} \StringTok{"Vowel"}\NormalTok{, ]}
\NormalTok{td}\SpecialCharTok{$}\NormalTok{Before }\OtherTok{\textless{}{-}} \FunctionTok{factor}\NormalTok{(td}\SpecialCharTok{$}\NormalTok{Before)}
\CommentTok{\# Re{-}code \textquotesingle{}H\textquotesingle{} to be \textquotesingle{}Consonant\textquotesingle{} in a new column}
\NormalTok{td}\SpecialCharTok{$}\NormalTok{After.New }\OtherTok{\textless{}{-}}\NormalTok{ td}\SpecialCharTok{$}\NormalTok{After}
\NormalTok{td}\SpecialCharTok{$}\NormalTok{After.New[td}\SpecialCharTok{$}\NormalTok{After.New }\SpecialCharTok{==} \StringTok{"H"}\NormalTok{] }\OtherTok{\textless{}{-}} \StringTok{"Consonant"}
\NormalTok{td}\SpecialCharTok{$}\NormalTok{After.New }\OtherTok{\textless{}{-}} \FunctionTok{factor}\NormalTok{(td}\SpecialCharTok{$}\NormalTok{After.New)}
\CommentTok{\# Center Year of Birth}
\NormalTok{td}\SpecialCharTok{$}\NormalTok{Center.Age }\OtherTok{\textless{}{-}} \FunctionTok{as.numeric}\NormalTok{(}\FunctionTok{scale}\NormalTok{(td}\SpecialCharTok{$}\NormalTok{YOB, }\AttributeTok{scale =} \ConstantTok{FALSE}\NormalTok{))}
\CommentTok{\# Create Age.Group}
\NormalTok{td}\SpecialCharTok{$}\NormalTok{Age.Group[td}\SpecialCharTok{$}\NormalTok{YOB }\SpecialCharTok{\textgreater{}} \DecValTok{1979}\NormalTok{] }\OtherTok{\textless{}{-}} \StringTok{"Young"}
\NormalTok{td}\SpecialCharTok{$}\NormalTok{Age.Group[td}\SpecialCharTok{$}\NormalTok{YOB }\SpecialCharTok{\textgreater{}} \DecValTok{1944} \SpecialCharTok{\&}\NormalTok{ td}\SpecialCharTok{$}\NormalTok{YOB }\SpecialCharTok{\textless{}} \DecValTok{1980}\NormalTok{] }\OtherTok{\textless{}{-}} \StringTok{"Middle"}
\NormalTok{td}\SpecialCharTok{$}\NormalTok{Age.Group[td}\SpecialCharTok{$}\NormalTok{YOB }\SpecialCharTok{\textless{}} \DecValTok{1945}\NormalTok{] }\OtherTok{\textless{}{-}} \StringTok{"Old"}
\NormalTok{td}\SpecialCharTok{$}\NormalTok{Age.Group }\OtherTok{\textless{}{-}} \FunctionTok{factor}\NormalTok{(td}\SpecialCharTok{$}\NormalTok{Age.Group, }\AttributeTok{levels =} \FunctionTok{c}\NormalTok{(}\StringTok{"Young"}\NormalTok{,}
    \StringTok{"Middle"}\NormalTok{, }\StringTok{"Old"}\NormalTok{))}
\CommentTok{\# Combine Age and Sex}
\NormalTok{td}\SpecialCharTok{$}\NormalTok{Age\_Sex }\OtherTok{\textless{}{-}} \FunctionTok{factor}\NormalTok{(}\FunctionTok{paste}\NormalTok{(td}\SpecialCharTok{$}\NormalTok{Age.Group, td}\SpecialCharTok{$}\NormalTok{Sex, }\AttributeTok{sep =} \StringTok{"\_"}\NormalTok{))}
\CommentTok{\# Break Phoneme.Dep.Var into two columns}
\FunctionTok{library}\NormalTok{(dplyr)}
\NormalTok{td }\OtherTok{\textless{}{-}} \FunctionTok{mutate}\NormalTok{(td, }\AttributeTok{Phoneme =} \FunctionTok{sub}\NormalTok{(}\StringTok{"\^{}(.)({-}{-}.*)$"}\NormalTok{, }\StringTok{"}\SpecialCharTok{\textbackslash{}\textbackslash{}}\StringTok{1"}\NormalTok{,}
\NormalTok{    Phoneme.Dep.Var), }\AttributeTok{Dep.Var.Full =} \FunctionTok{sub}\NormalTok{(}\StringTok{"\^{}(.{-}{-})(.*)$"}\NormalTok{,}
    \StringTok{"}\SpecialCharTok{\textbackslash{}\textbackslash{}}\StringTok{2"}\NormalTok{, Phoneme.Dep.Var), }\AttributeTok{Phoneme.Dep.Var =} \ConstantTok{NULL}\NormalTok{)}
\NormalTok{td}\SpecialCharTok{$}\NormalTok{Phoneme }\OtherTok{\textless{}{-}} \FunctionTok{factor}\NormalTok{(td}\SpecialCharTok{$}\NormalTok{Phoneme)}
\NormalTok{td}\SpecialCharTok{$}\NormalTok{Dep.Var.Full }\OtherTok{\textless{}{-}} \FunctionTok{factor}\NormalTok{(td}\SpecialCharTok{$}\NormalTok{Dep.Var.Full)}
\CommentTok{\# Create three partitions based on Age.Group}
\NormalTok{td.young }\OtherTok{\textless{}{-}} \FunctionTok{droplevels}\NormalTok{(}\FunctionTok{subset}\NormalTok{(td, Age.Group }\SpecialCharTok{==} \StringTok{"Young"}\NormalTok{))}
\NormalTok{td.middle }\OtherTok{\textless{}{-}} \FunctionTok{droplevels}\NormalTok{(}\FunctionTok{subset}\NormalTok{(td, Age.Group }\SpecialCharTok{==} \StringTok{"Middle"}\NormalTok{))}
\NormalTok{td.old }\OtherTok{\textless{}{-}} \FunctionTok{droplevels}\NormalTok{(}\FunctionTok{subset}\NormalTok{(td, Age.Group }\SpecialCharTok{==} \StringTok{"Old"}\NormalTok{))}
\CommentTok{\# Re{-}center Center.Age}
\NormalTok{td.young}\SpecialCharTok{$}\NormalTok{Center.Age }\OtherTok{\textless{}{-}} \FunctionTok{scale}\NormalTok{(td.young}\SpecialCharTok{$}\NormalTok{YOB, }\AttributeTok{scale =} \ConstantTok{FALSE}\NormalTok{)}
\NormalTok{td.middle}\SpecialCharTok{$}\NormalTok{Center.Age }\OtherTok{\textless{}{-}} \FunctionTok{scale}\NormalTok{(td.middle}\SpecialCharTok{$}\NormalTok{YOB, }\AttributeTok{scale =} \ConstantTok{FALSE}\NormalTok{)}
\NormalTok{td.old}\SpecialCharTok{$}\NormalTok{Center.Age }\OtherTok{\textless{}{-}} \FunctionTok{scale}\NormalTok{(td.old}\SpecialCharTok{$}\NormalTok{YOB, }\AttributeTok{scale =} \ConstantTok{FALSE}\NormalTok{)}
\end{Highlighting}
\end{Shaded}




\end{document}
