% Options for packages loaded elsewhere
\PassOptionsToPackage{unicode}{hyperref}
\PassOptionsToPackage{hyphens}{url}
\PassOptionsToPackage{dvipsnames,svgnames,x11names}{xcolor}
%
\documentclass[
  10pt,
  letterpaper]{article}

\usepackage{amsmath,amssymb}
\usepackage{lmodern}
\usepackage{iftex}
\ifPDFTeX
  \usepackage[T1]{fontenc}
  \usepackage[utf8]{inputenc}
  \usepackage{textcomp} % provide euro and other symbols
\else % if luatex or xetex
  \usepackage{unicode-math}
  \defaultfontfeatures{Scale=MatchLowercase}
  \defaultfontfeatures[\rmfamily]{Ligatures=TeX,Scale=1}
  \setmainfont[]{Charis SIL}
  \setmathfont[]{Monaco}
\fi
% Use upquote if available, for straight quotes in verbatim environments
\IfFileExists{upquote.sty}{\usepackage{upquote}}{}
\IfFileExists{microtype.sty}{% use microtype if available
  \usepackage[]{microtype}
  \UseMicrotypeSet[protrusion]{basicmath} % disable protrusion for tt fonts
}{}
\makeatletter
\@ifundefined{KOMAClassName}{% if non-KOMA class
  \IfFileExists{parskip.sty}{%
    \usepackage{parskip}
  }{% else
    \setlength{\parindent}{0pt}
    \setlength{\parskip}{6pt plus 2pt minus 1pt}}
}{% if KOMA class
  \KOMAoptions{parskip=half}}
\makeatother
\usepackage{xcolor}
\usepackage[margin = 1in]{geometry}
\setlength{\emergencystretch}{3em} % prevent overfull lines
\setcounter{secnumdepth}{-\maxdimen} % remove section numbering
% Make \paragraph and \subparagraph free-standing
\ifx\paragraph\undefined\else
  \let\oldparagraph\paragraph
  \renewcommand{\paragraph}[1]{\oldparagraph{#1}\mbox{}}
\fi
\ifx\subparagraph\undefined\else
  \let\oldsubparagraph\subparagraph
  \renewcommand{\subparagraph}[1]{\oldsubparagraph{#1}\mbox{}}
\fi

\usepackage{color}
\usepackage{fancyvrb}
\newcommand{\VerbBar}{|}
\newcommand{\VERB}{\Verb[commandchars=\\\{\}]}
\DefineVerbatimEnvironment{Highlighting}{Verbatim}{commandchars=\\\{\}}
% Add ',fontsize=\small' for more characters per line
\usepackage{framed}
\definecolor{shadecolor}{RGB}{241,243,245}
\newenvironment{Shaded}{\begin{snugshade}}{\end{snugshade}}
\newcommand{\AlertTok}[1]{\textcolor[rgb]{0.68,0.00,0.00}{#1}}
\newcommand{\AnnotationTok}[1]{\textcolor[rgb]{0.37,0.37,0.37}{#1}}
\newcommand{\AttributeTok}[1]{\textcolor[rgb]{0.40,0.45,0.13}{#1}}
\newcommand{\BaseNTok}[1]{\textcolor[rgb]{0.68,0.00,0.00}{#1}}
\newcommand{\BuiltInTok}[1]{\textcolor[rgb]{0.00,0.23,0.31}{#1}}
\newcommand{\CharTok}[1]{\textcolor[rgb]{0.13,0.47,0.30}{#1}}
\newcommand{\CommentTok}[1]{\textcolor[rgb]{0.37,0.37,0.37}{#1}}
\newcommand{\CommentVarTok}[1]{\textcolor[rgb]{0.37,0.37,0.37}{\textit{#1}}}
\newcommand{\ConstantTok}[1]{\textcolor[rgb]{0.56,0.35,0.01}{#1}}
\newcommand{\ControlFlowTok}[1]{\textcolor[rgb]{0.00,0.23,0.31}{#1}}
\newcommand{\DataTypeTok}[1]{\textcolor[rgb]{0.68,0.00,0.00}{#1}}
\newcommand{\DecValTok}[1]{\textcolor[rgb]{0.68,0.00,0.00}{#1}}
\newcommand{\DocumentationTok}[1]{\textcolor[rgb]{0.37,0.37,0.37}{\textit{#1}}}
\newcommand{\ErrorTok}[1]{\textcolor[rgb]{0.68,0.00,0.00}{#1}}
\newcommand{\ExtensionTok}[1]{\textcolor[rgb]{0.00,0.23,0.31}{#1}}
\newcommand{\FloatTok}[1]{\textcolor[rgb]{0.68,0.00,0.00}{#1}}
\newcommand{\FunctionTok}[1]{\textcolor[rgb]{0.28,0.35,0.67}{#1}}
\newcommand{\ImportTok}[1]{\textcolor[rgb]{0.00,0.46,0.62}{#1}}
\newcommand{\InformationTok}[1]{\textcolor[rgb]{0.37,0.37,0.37}{#1}}
\newcommand{\KeywordTok}[1]{\textcolor[rgb]{0.00,0.23,0.31}{#1}}
\newcommand{\NormalTok}[1]{\textcolor[rgb]{0.00,0.23,0.31}{#1}}
\newcommand{\OperatorTok}[1]{\textcolor[rgb]{0.37,0.37,0.37}{#1}}
\newcommand{\OtherTok}[1]{\textcolor[rgb]{0.00,0.23,0.31}{#1}}
\newcommand{\PreprocessorTok}[1]{\textcolor[rgb]{0.68,0.00,0.00}{#1}}
\newcommand{\RegionMarkerTok}[1]{\textcolor[rgb]{0.00,0.23,0.31}{#1}}
\newcommand{\SpecialCharTok}[1]{\textcolor[rgb]{0.37,0.37,0.37}{#1}}
\newcommand{\SpecialStringTok}[1]{\textcolor[rgb]{0.13,0.47,0.30}{#1}}
\newcommand{\StringTok}[1]{\textcolor[rgb]{0.13,0.47,0.30}{#1}}
\newcommand{\VariableTok}[1]{\textcolor[rgb]{0.07,0.07,0.07}{#1}}
\newcommand{\VerbatimStringTok}[1]{\textcolor[rgb]{0.13,0.47,0.30}{#1}}
\newcommand{\WarningTok}[1]{\textcolor[rgb]{0.37,0.37,0.37}{\textit{#1}}}

\providecommand{\tightlist}{%
  \setlength{\itemsep}{0pt}\setlength{\parskip}{0pt}}\usepackage{longtable,booktabs,array}
\usepackage{calc} % for calculating minipage widths
% Correct order of tables after \paragraph or \subparagraph
\usepackage{etoolbox}
\makeatletter
\patchcmd\longtable{\par}{\if@noskipsec\mbox{}\fi\par}{}{}
\makeatother
% Allow footnotes in longtable head/foot
\IfFileExists{footnotehyper.sty}{\usepackage{footnotehyper}}{\usepackage{footnote}}
\makesavenoteenv{longtable}
\usepackage{graphicx}
\makeatletter
\def\maxwidth{\ifdim\Gin@nat@width>\linewidth\linewidth\else\Gin@nat@width\fi}
\def\maxheight{\ifdim\Gin@nat@height>\textheight\textheight\else\Gin@nat@height\fi}
\makeatother
% Scale images if necessary, so that they will not overflow the page
% margins by default, and it is still possible to overwrite the defaults
% using explicit options in \includegraphics[width, height, ...]{}
\setkeys{Gin}{width=\maxwidth,height=\maxheight,keepaspectratio}
% Set default figure placement to htbp
\makeatletter
\def\fps@figure{htbp}
\makeatother

\usepackage{tabularx}
\usepackage{threeparttable}
\usepackage{booktabs}
\usepackage{tipa}
\let\Oldtexttt\texttt
\renewcommand\texttt[1]{{\ttfamily\color{BrickRed}#1}}
\usepackage{authoraftertitle}
\usepackage{fancyhdr}
\pagestyle{fancy}
\rfoot{\copyright Matt Hunt Gardner}
\cfoot{\thepage}
\lhead{Doing LVC with \textit{R}: \MyTitle}
\rhead{}
\makeatletter
\@ifpackageloaded{tcolorbox}{}{\usepackage[many]{tcolorbox}}
\@ifpackageloaded{fontawesome5}{}{\usepackage{fontawesome5}}
\definecolor{quarto-callout-color}{HTML}{909090}
\definecolor{quarto-callout-note-color}{HTML}{0758E5}
\definecolor{quarto-callout-important-color}{HTML}{CC1914}
\definecolor{quarto-callout-warning-color}{HTML}{EB9113}
\definecolor{quarto-callout-tip-color}{HTML}{00A047}
\definecolor{quarto-callout-caution-color}{HTML}{FC5300}
\definecolor{quarto-callout-color-frame}{HTML}{acacac}
\definecolor{quarto-callout-note-color-frame}{HTML}{4582ec}
\definecolor{quarto-callout-important-color-frame}{HTML}{d9534f}
\definecolor{quarto-callout-warning-color-frame}{HTML}{f0ad4e}
\definecolor{quarto-callout-tip-color-frame}{HTML}{02b875}
\definecolor{quarto-callout-caution-color-frame}{HTML}{fd7e14}
\makeatother
\makeatletter
\makeatother
\makeatletter
\makeatother
\makeatletter
\@ifpackageloaded{caption}{}{\usepackage{caption}}
\AtBeginDocument{%
\ifdefined\contentsname
  \renewcommand*\contentsname{Table of contents}
\else
  \newcommand\contentsname{Table of contents}
\fi
\ifdefined\listfigurename
  \renewcommand*\listfigurename{List of Figures}
\else
  \newcommand\listfigurename{List of Figures}
\fi
\ifdefined\listtablename
  \renewcommand*\listtablename{List of Tables}
\else
  \newcommand\listtablename{List of Tables}
\fi
\ifdefined\figurename
  \renewcommand*\figurename{Figure}
\else
  \newcommand\figurename{Figure}
\fi
\ifdefined\tablename
  \renewcommand*\tablename{Table}
\else
  \newcommand\tablename{Table}
\fi
}
\@ifpackageloaded{float}{}{\usepackage{float}}
\floatstyle{ruled}
\@ifundefined{c@chapter}{\newfloat{codelisting}{h}{lop}}{\newfloat{codelisting}{h}{lop}[chapter]}
\floatname{codelisting}{Listing}
\newcommand*\listoflistings{\listof{codelisting}{List of Listings}}
\makeatother
\makeatletter
\@ifpackageloaded{caption}{}{\usepackage{caption}}
\@ifpackageloaded{subcaption}{}{\usepackage{subcaption}}
\makeatother
\makeatletter
\@ifpackageloaded{tcolorbox}{}{\usepackage[many]{tcolorbox}}
\makeatother
\makeatletter
\@ifundefined{shadecolor}{\definecolor{shadecolor}{rgb}{.97, .97, .97}}
\makeatother
\makeatletter
\makeatother
\ifLuaTeX
  \usepackage{selnolig}  % disable illegal ligatures
\fi
\IfFileExists{bookmark.sty}{\usepackage{bookmark}}{\usepackage{hyperref}}
\IfFileExists{xurl.sty}{\usepackage{xurl}}{} % add URL line breaks if available
\urlstyle{same} % disable monospaced font for URLs
% Make links footnotes instead of hotlinks:
\DeclareRobustCommand{\href}[2]{#2\footnote{\url{#1}}}
\hypersetup{
  pdftitle={Conditional Inference Trees},
  pdfauthor={Matt Hunt Gardner},
  colorlinks=true,
  linkcolor={blue},
  filecolor={Maroon},
  citecolor={Blue},
  urlcolor={Blue},
  pdfcreator={LaTeX via pandoc}}

\title{Conditional Inference Trees}
\usepackage{etoolbox}
\makeatletter
\providecommand{\subtitle}[1]{% add subtitle to \maketitle
  \apptocmd{\@title}{\par {\large #1 \par}}{}{}
}
\makeatother
\subtitle{from
\href{https://lingmethodshub.github.io/content/R/lvc_r/}{Doing LVC with
\emph{R}}}
\author{Matt Hunt Gardner}
\date{2/16/23}

\begin{document}
\maketitle
\ifdefined\Shaded\renewenvironment{Shaded}{\begin{tcolorbox}[borderline west={3pt}{0pt}{shadecolor}, enhanced, frame hidden, breakable, boxrule=0pt, sharp corners, interior hidden]}{\end{tcolorbox}}\fi

A useful data exploration technique is using conditional inference
recursive partitioning trees. These analyses, represented as plotted
partitioned trees, show us where there are significant differences
between levels of factor groups.

Conditional inference trees are created by the package \texttt{party},
though I prefer the newer package \texttt{partykit}, which does the same
analysis as \texttt{party} but has more customizable plots.\footnote{The
  package \texttt{ggparty} provides even further customization via
  \texttt{ggplot2} plotting.} The \texttt{ctree()} function is set up
like most analysis functions in \texttt{R}. You start by specifying the
dependent variable (here \texttt{Dep.Var}) followed by a
\texttt{\textasciitilde{}}. Everything to the right of the
\texttt{\textasciitilde{}} is a potential predictor (e.g., independent
variable). Here you only specify \texttt{Sex} as a potential predictor.
Finally, you specify that the data is \texttt{td}.

\begin{tcolorbox}[enhanced jigsaw, colback=white, left=2mm, bottomtitle=1mm, opacitybacktitle=0.6, breakable, coltitle=black, toptitle=1mm, leftrule=.75mm, title=\textcolor{quarto-callout-tip-color}{\faLightbulb}\hspace{0.5em}{Get the data first}, opacityback=0, rightrule=.15mm, arc=.35mm, colbacktitle=quarto-callout-tip-color!10!white, bottomrule=.15mm, colframe=quarto-callout-tip-color-frame, toprule=.15mm, titlerule=0mm]

If you don't have the \texttt{td} data loaded in \emph{R}, go back to
\href{https://lingmethodshub.github.io/content/R/lvc_r/050_lvcr.html}{Doing
it all again, but \texttt{tidy}} and run the code.

\end{tcolorbox}

\begin{Shaded}
\begin{Highlighting}[]
\CommentTok{\# Make a Conditional Inference Tree testing Sex}
\FunctionTok{library}\NormalTok{(partykit)}
\NormalTok{td.ctree }\OtherTok{\textless{}{-}} \FunctionTok{ctree}\NormalTok{(Dep.Var }\SpecialCharTok{\textasciitilde{}}\NormalTok{ Sex, }\AttributeTok{data =}\NormalTok{ td)}
\FunctionTok{plot}\NormalTok{(td.ctree)}
\end{Highlighting}
\end{Shaded}

\begin{figure}[H]

{\centering \includegraphics{080_lvcr_files/figure-pdf/unnamed-chunk-2-1.pdf}

}

\end{figure}

The plot above visualizes the conditional inference tree analysis. There
is a significant difference (p=0.001) between \texttt{F} females and
\texttt{M} males in the data, with males using a higher percentage of
\texttt{Deletion} variants versus \texttt{Realized} variants compared to
females. Here the black part of the bars represent \texttt{Realization},
but this might not be how you want to represent this variable in a
figure. Sometimes in variationist sociolinguistics variation is
expressed in terms of the non-standard variant or the variant that is
the most unlike the underlying representation. This is especially true
for (t, d) deletion --- most of the literature on the variable discusses
``rates of deletion'' not ``rates of realization''. Therefore, for a
manuscript, you might decide to make the black bars (and the proportions
they show on the right) represent \texttt{Deletion}. You can do this by
reordering the levels of \texttt{Dep.Var} and then re-running your
\texttt{ctree()} function.

\begin{Shaded}
\begin{Highlighting}[]
\CommentTok{\# Reorder levels of Dep.Var}
\NormalTok{td}\SpecialCharTok{$}\NormalTok{Dep.Var }\OtherTok{\textless{}{-}} \FunctionTok{factor}\NormalTok{(td}\SpecialCharTok{$}\NormalTok{Dep.Var, }\AttributeTok{levels =} \FunctionTok{c}\NormalTok{(}\StringTok{"Realized"}\NormalTok{,}
    \StringTok{"Deletion"}\NormalTok{))}

\CommentTok{\# Make a Conditional Inference Tree testing Sex}
\NormalTok{td.ctree }\OtherTok{\textless{}{-}} \FunctionTok{ctree}\NormalTok{(Dep.Var }\SpecialCharTok{\textasciitilde{}}\NormalTok{ Sex, }\AttributeTok{data =}\NormalTok{ td)}
\FunctionTok{plot}\NormalTok{(td.ctree)}
\end{Highlighting}
\end{Shaded}

\begin{figure}[H]

{\centering \includegraphics{080_lvcr_files/figure-pdf/unnamed-chunk-3-1.pdf}

}

\end{figure}

We can add more predictors to this analysis. Each predictor is separated
by \texttt{+}.

One specific way variationist sociolinguists have used conditional
inference trees over the last few years has been to identify significant
divisions or ``shock points'\,' in continuous age or year of birth
variables. These divisions are the ages before and after which speakers
show a significant difference in the use of the dependent variable.

\begin{Shaded}
\begin{Highlighting}[]
\CommentTok{\# Make a Conditional Inference Tree testing YOB}
\CommentTok{\# and Sex}
\NormalTok{td.ctree }\OtherTok{\textless{}{-}} \FunctionTok{ctree}\NormalTok{(Dep.Var }\SpecialCharTok{\textasciitilde{}}\NormalTok{ YOB }\SpecialCharTok{+}\NormalTok{ Sex, }\AttributeTok{data =}\NormalTok{ td)}
\FunctionTok{plot}\NormalTok{(td.ctree)}
\end{Highlighting}
\end{Shaded}

\begin{figure}[H]

{\centering \includegraphics{080_lvcr_files/figure-pdf/unnamed-chunk-4-1.pdf}

}

\end{figure}

This tree shows the conditional inference tree where both \texttt{Sex}
and \texttt{YOB} are included as potential predictors. The tree tells us
that there is a significant gender effect, and a significant age effect,
but only for men. First, it shows that the predictor with the greatest
explanatory value (or which has the greatest magnitude of effect, if
that's how you want to think about it) is \texttt{Sex}. There is a
significant difference (\(p=0.002\)) between females, who use
\texttt{Deletion} about \(29\%\) of the time and males. Among the males
there is a significant difference (\(p=0.017\)) between men born in and
before 1990 --- who use \texttt{Deletion} more, at about \(43\%\) ---
and men born after 1990 -- who use \texttt{Deletion} less, at about
\(30\%\). This, of course, doesn't take any other factors into account.

Another useful way to employ conditional inference trees is when you
have an independent variable with many levels. The \texttt{ctree()}
function can help you decide how to collapse some of the levels of the
variable. Of course, this should only be done as part of a theory-driven
reflection on the relevant distinctions in the level. The
\texttt{ctree()} should function as statistical validation for your
theory-driven choices to merge certain levels of an independent
variable.

Let's examine following phonological context. Below we generate a
\texttt{ctree()} with \texttt{After} as the independent variable.

\begin{Shaded}
\begin{Highlighting}[]
\NormalTok{td.ctree }\OtherTok{\textless{}{-}} \FunctionTok{ctree}\NormalTok{(Dep.Var }\SpecialCharTok{\textasciitilde{}}\NormalTok{ After, }\AttributeTok{data =}\NormalTok{ td)}
\FunctionTok{plot}\NormalTok{(td.ctree)}
\end{Highlighting}
\end{Shaded}

\begin{figure}[H]

{\centering \includegraphics{080_lvcr_files/figure-pdf/unnamed-chunk-5-1.pdf}

}

\end{figure}

We can see that the first division is between \texttt{Consonant} and all
other options. Among the other options there is a division between
\texttt{Pause}/\texttt{H} and \texttt{Vowels}. This might give us
justification to merge \texttt{Pause/H} together in subsequent analyses.
But, there also needs to be a theoretically-driven reason to do this.
Technically \texttt{H} represents a consonant, so why shouldn't it be
grouped with other consonants? Other analyses of (t, d) deletion group
pre-/h/ contexts with other pre-consonant contexts, so why shouldn't we
here? This is why your decisions need to be guided by theory. Is there a
good theoretical reason why pre-pausal and pre-/h/ contexts should be
merged? Well, in this case, yes there is.

If \texttt{Deletion} vs \texttt{Realization} is effected by following
phonological context, why might that be? Firstly, if there is a
following consonant, deletion could be due to neutralization with that
following consonant. This type of neutralization is generally not
possible with /h/. Additionally, deletion may be part of a process
whereby a word-final consonant clusters are re-phonologized as the onset
to a following vowel-initial word, and certain onsets, like /nt-/ or
/ld-/, are not licit and (t, d) deletion results. In this case, pre-/h/
and pre-pausal contexts represent the category of phonological contexts
for which neither neutralization or re-phonologization effects are
possible. For this reason, combining \texttt{H} and \texttt{Pause} would
be theoretically justified, and this decision can be supported by the
results of the conditional inference tree analysis.

\begin{tcolorbox}[enhanced jigsaw, colback=white, left=2mm, bottomtitle=1mm, opacitybacktitle=0.6, breakable, coltitle=black, toptitle=1mm, leftrule=.75mm, title=\textcolor{quarto-callout-warning-color}{\faExclamationTriangle}\hspace{0.5em}{Warning}, opacityback=0, rightrule=.15mm, arc=.35mm, colbacktitle=quarto-callout-warning-color!10!white, bottomrule=.15mm, colframe=quarto-callout-warning-color-frame, toprule=.15mm, titlerule=0mm]

You'll remember from
\href{https://lingmethodshub.github.io/content/R/lvc_r/040_lvcr.html}{previous
chapters} that we merged \texttt{H} with \texttt{Consonant} in the new
column \texttt{Afer.New}. This was motivated by the desire to have our
data match other analyses of (t, d) deletion. Finding out that
\texttt{H} patterns with \texttt{Pause} leaves us with a methodological
choice. We can group \texttt{H} with \texttt{Pause}, which might be the
most appropriate for the distribution of \texttt{Deletion} and
\texttt{Realized} in our data, but doing so means we violate one of the
assumptions of the comparative method (discussed in
\href{https://lingmethodshub.github.io/content/R/lvc_r/100_lvcr.html}{subsequent
chapters}). We can group \texttt{H} with other consonants in order to
compare our data to past analyses using the comparative method, but
whether this produces the most accurate analysis of our own data is
questionable. A good rule of thumb in these scenarios is to analyse the
data both ways, and be honest about your methodological choices in your
manuscript.

\end{tcolorbox}



\end{document}
