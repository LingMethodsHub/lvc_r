% Options for packages loaded elsewhere
\PassOptionsToPackage{unicode}{hyperref}
\PassOptionsToPackage{hyphens}{url}
\PassOptionsToPackage{dvipsnames,svgnames,x11names}{xcolor}
%
\documentclass[
  12pt,
  letterpaper]{article}

\usepackage{amsmath,amssymb}
\usepackage{lmodern}
\usepackage{iftex}
\ifPDFTeX
  \usepackage[T1]{fontenc}
  \usepackage[utf8]{inputenc}
  \usepackage{textcomp} % provide euro and other symbols
\else % if luatex or xetex
  \usepackage{unicode-math}
  \defaultfontfeatures{Scale=MatchLowercase}
  \defaultfontfeatures[\rmfamily]{Ligatures=TeX,Scale=1}
  \setmainfont[]{Charis SIL}
  \setmathfont[]{Monaco}
\fi
% Use upquote if available, for straight quotes in verbatim environments
\IfFileExists{upquote.sty}{\usepackage{upquote}}{}
\IfFileExists{microtype.sty}{% use microtype if available
  \usepackage[]{microtype}
  \UseMicrotypeSet[protrusion]{basicmath} % disable protrusion for tt fonts
}{}
\makeatletter
\@ifundefined{KOMAClassName}{% if non-KOMA class
  \IfFileExists{parskip.sty}{%
    \usepackage{parskip}
  }{% else
    \setlength{\parindent}{0pt}
    \setlength{\parskip}{6pt plus 2pt minus 1pt}}
}{% if KOMA class
  \KOMAoptions{parskip=half}}
\makeatother
\usepackage{xcolor}
\usepackage[margin = 1in]{geometry}
\setlength{\emergencystretch}{3em} % prevent overfull lines
\setcounter{secnumdepth}{-\maxdimen} % remove section numbering
% Make \paragraph and \subparagraph free-standing
\ifx\paragraph\undefined\else
  \let\oldparagraph\paragraph
  \renewcommand{\paragraph}[1]{\oldparagraph{#1}\mbox{}}
\fi
\ifx\subparagraph\undefined\else
  \let\oldsubparagraph\subparagraph
  \renewcommand{\subparagraph}[1]{\oldsubparagraph{#1}\mbox{}}
\fi

\usepackage{color}
\usepackage{fancyvrb}
\newcommand{\VerbBar}{|}
\newcommand{\VERB}{\Verb[commandchars=\\\{\}]}
\DefineVerbatimEnvironment{Highlighting}{Verbatim}{commandchars=\\\{\}}
% Add ',fontsize=\small' for more characters per line
\usepackage{framed}
\definecolor{shadecolor}{RGB}{241,243,245}
\newenvironment{Shaded}{\begin{snugshade}}{\end{snugshade}}
\newcommand{\AlertTok}[1]{\textcolor[rgb]{0.68,0.00,0.00}{#1}}
\newcommand{\AnnotationTok}[1]{\textcolor[rgb]{0.37,0.37,0.37}{#1}}
\newcommand{\AttributeTok}[1]{\textcolor[rgb]{0.40,0.45,0.13}{#1}}
\newcommand{\BaseNTok}[1]{\textcolor[rgb]{0.68,0.00,0.00}{#1}}
\newcommand{\BuiltInTok}[1]{\textcolor[rgb]{0.00,0.23,0.31}{#1}}
\newcommand{\CharTok}[1]{\textcolor[rgb]{0.13,0.47,0.30}{#1}}
\newcommand{\CommentTok}[1]{\textcolor[rgb]{0.37,0.37,0.37}{#1}}
\newcommand{\CommentVarTok}[1]{\textcolor[rgb]{0.37,0.37,0.37}{\textit{#1}}}
\newcommand{\ConstantTok}[1]{\textcolor[rgb]{0.56,0.35,0.01}{#1}}
\newcommand{\ControlFlowTok}[1]{\textcolor[rgb]{0.00,0.23,0.31}{#1}}
\newcommand{\DataTypeTok}[1]{\textcolor[rgb]{0.68,0.00,0.00}{#1}}
\newcommand{\DecValTok}[1]{\textcolor[rgb]{0.68,0.00,0.00}{#1}}
\newcommand{\DocumentationTok}[1]{\textcolor[rgb]{0.37,0.37,0.37}{\textit{#1}}}
\newcommand{\ErrorTok}[1]{\textcolor[rgb]{0.68,0.00,0.00}{#1}}
\newcommand{\ExtensionTok}[1]{\textcolor[rgb]{0.00,0.23,0.31}{#1}}
\newcommand{\FloatTok}[1]{\textcolor[rgb]{0.68,0.00,0.00}{#1}}
\newcommand{\FunctionTok}[1]{\textcolor[rgb]{0.28,0.35,0.67}{#1}}
\newcommand{\ImportTok}[1]{\textcolor[rgb]{0.00,0.46,0.62}{#1}}
\newcommand{\InformationTok}[1]{\textcolor[rgb]{0.37,0.37,0.37}{#1}}
\newcommand{\KeywordTok}[1]{\textcolor[rgb]{0.00,0.23,0.31}{#1}}
\newcommand{\NormalTok}[1]{\textcolor[rgb]{0.00,0.23,0.31}{#1}}
\newcommand{\OperatorTok}[1]{\textcolor[rgb]{0.37,0.37,0.37}{#1}}
\newcommand{\OtherTok}[1]{\textcolor[rgb]{0.00,0.23,0.31}{#1}}
\newcommand{\PreprocessorTok}[1]{\textcolor[rgb]{0.68,0.00,0.00}{#1}}
\newcommand{\RegionMarkerTok}[1]{\textcolor[rgb]{0.00,0.23,0.31}{#1}}
\newcommand{\SpecialCharTok}[1]{\textcolor[rgb]{0.37,0.37,0.37}{#1}}
\newcommand{\SpecialStringTok}[1]{\textcolor[rgb]{0.13,0.47,0.30}{#1}}
\newcommand{\StringTok}[1]{\textcolor[rgb]{0.13,0.47,0.30}{#1}}
\newcommand{\VariableTok}[1]{\textcolor[rgb]{0.07,0.07,0.07}{#1}}
\newcommand{\VerbatimStringTok}[1]{\textcolor[rgb]{0.13,0.47,0.30}{#1}}
\newcommand{\WarningTok}[1]{\textcolor[rgb]{0.37,0.37,0.37}{\textit{#1}}}

\providecommand{\tightlist}{%
  \setlength{\itemsep}{0pt}\setlength{\parskip}{0pt}}\usepackage{longtable,booktabs,array}
\usepackage{calc} % for calculating minipage widths
% Correct order of tables after \paragraph or \subparagraph
\usepackage{etoolbox}
\makeatletter
\patchcmd\longtable{\par}{\if@noskipsec\mbox{}\fi\par}{}{}
\makeatother
% Allow footnotes in longtable head/foot
\IfFileExists{footnotehyper.sty}{\usepackage{footnotehyper}}{\usepackage{footnote}}
\makesavenoteenv{longtable}
\usepackage{graphicx}
\makeatletter
\def\maxwidth{\ifdim\Gin@nat@width>\linewidth\linewidth\else\Gin@nat@width\fi}
\def\maxheight{\ifdim\Gin@nat@height>\textheight\textheight\else\Gin@nat@height\fi}
\makeatother
% Scale images if necessary, so that they will not overflow the page
% margins by default, and it is still possible to overwrite the defaults
% using explicit options in \includegraphics[width, height, ...]{}
\setkeys{Gin}{width=\maxwidth,height=\maxheight,keepaspectratio}
% Set default figure placement to htbp
\makeatletter
\def\fps@figure{htbp}
\makeatother

\usepackage{tabularx}
\usepackage{threeparttable}
\usepackage{booktabs}
\usepackage{tipa}
\let\Oldtexttt\texttt
\renewcommand\texttt[1]{{\ttfamily\color{BrickRed}#1}}
\usepackage{authoraftertitle}
\usepackage{fancyhdr}
\pagestyle{fancy}
\rfoot{\copyright Matt Hunt Gardner}
\cfoot{\thepage}
\lhead{Doing LVC with \textit{R}: \MyTitle}
\rhead{}
\makeatletter
\@ifpackageloaded{tcolorbox}{}{\usepackage[many]{tcolorbox}}
\@ifpackageloaded{fontawesome5}{}{\usepackage{fontawesome5}}
\definecolor{quarto-callout-color}{HTML}{909090}
\definecolor{quarto-callout-note-color}{HTML}{0758E5}
\definecolor{quarto-callout-important-color}{HTML}{CC1914}
\definecolor{quarto-callout-warning-color}{HTML}{EB9113}
\definecolor{quarto-callout-tip-color}{HTML}{00A047}
\definecolor{quarto-callout-caution-color}{HTML}{FC5300}
\definecolor{quarto-callout-color-frame}{HTML}{acacac}
\definecolor{quarto-callout-note-color-frame}{HTML}{4582ec}
\definecolor{quarto-callout-important-color-frame}{HTML}{d9534f}
\definecolor{quarto-callout-warning-color-frame}{HTML}{f0ad4e}
\definecolor{quarto-callout-tip-color-frame}{HTML}{02b875}
\definecolor{quarto-callout-caution-color-frame}{HTML}{fd7e14}
\makeatother
\makeatletter
\makeatother
\makeatletter
\makeatother
\makeatletter
\@ifpackageloaded{caption}{}{\usepackage{caption}}
\AtBeginDocument{%
\ifdefined\contentsname
  \renewcommand*\contentsname{Table of contents}
\else
  \newcommand\contentsname{Table of contents}
\fi
\ifdefined\listfigurename
  \renewcommand*\listfigurename{List of Figures}
\else
  \newcommand\listfigurename{List of Figures}
\fi
\ifdefined\listtablename
  \renewcommand*\listtablename{List of Tables}
\else
  \newcommand\listtablename{List of Tables}
\fi
\ifdefined\figurename
  \renewcommand*\figurename{Figure}
\else
  \newcommand\figurename{Figure}
\fi
\ifdefined\tablename
  \renewcommand*\tablename{Table}
\else
  \newcommand\tablename{Table}
\fi
}
\@ifpackageloaded{float}{}{\usepackage{float}}
\floatstyle{ruled}
\@ifundefined{c@chapter}{\newfloat{codelisting}{h}{lop}}{\newfloat{codelisting}{h}{lop}[chapter]}
\floatname{codelisting}{Listing}
\newcommand*\listoflistings{\listof{codelisting}{List of Listings}}
\makeatother
\makeatletter
\@ifpackageloaded{caption}{}{\usepackage{caption}}
\@ifpackageloaded{subcaption}{}{\usepackage{subcaption}}
\makeatother
\makeatletter
\@ifpackageloaded{tcolorbox}{}{\usepackage[many]{tcolorbox}}
\makeatother
\makeatletter
\@ifundefined{shadecolor}{\definecolor{shadecolor}{rgb}{.97, .97, .97}}
\makeatother
\makeatletter
\makeatother
\ifLuaTeX
  \usepackage{selnolig}  % disable illegal ligatures
\fi
\IfFileExists{bookmark.sty}{\usepackage{bookmark}}{\usepackage{hyperref}}
\IfFileExists{xurl.sty}{\usepackage{xurl}}{} % add URL line breaks if available
\urlstyle{same} % disable monospaced font for URLs
% Make links footnotes instead of hotlinks:
\DeclareRobustCommand{\href}[2]{#2\footnote{\url{#1}}}
\hypersetup{
  pdftitle={Doing it all again, but tidy},
  pdfauthor={Matt Hunt Gardner},
  colorlinks=true,
  linkcolor={blue},
  filecolor={Maroon},
  citecolor={Blue},
  urlcolor={Blue},
  pdfcreator={LaTeX via pandoc}}

\title{Doing it all again, but \texttt{tidy}}
\usepackage{etoolbox}
\makeatletter
\providecommand{\subtitle}[1]{% add subtitle to \maketitle
  \apptocmd{\@title}{\par {\large #1 \par}}{}{}
}
\makeatother
\subtitle{from
\href{https://lingmethodshub.github.io/content/R/lvc_r/}{Doing LVC with
\emph{R}}}
\author{Matt Hunt Gardner}
\date{2/16/23}

\begin{document}
\maketitle
\ifdefined\Shaded\renewenvironment{Shaded}{\begin{tcolorbox}[borderline west={3pt}{0pt}{shadecolor}, interior hidden, enhanced, breakable, boxrule=0pt, frame hidden, sharp corners]}{\end{tcolorbox}}\fi

\renewcommand*\contentsname{Table of contents}
{
\hypersetup{linkcolor=}
\setcounter{tocdepth}{3}
\tableofcontents
}
\hypertarget{doing-it-all-again-but-tidy}{%
\subsection{\texorpdfstring{Doing It All Again, But
\texttt{tidy}}{Doing It All Again, But tidy}}\label{doing-it-all-again-but-tidy}}

The package \texttt{dplyr} is part of a larger ``universe'' of \emph{R}
packages called \texttt{tidyverse}. This collection of packages is
specifically focused on data science and offers some shortcuts that are
useful to learn. The packages that make up the \texttt{tidyverse} are
\texttt{dplyr}, \texttt{ggplot2}, \texttt{purr}, \texttt{tibble},
\texttt{tidyr}, \texttt{stingr}, \texttt{readr}, and \texttt{forcats},
among others. Throughout this guide I try to use the most basic \emph{R}
syntax for accomplishing a task. This way you learn how \emph{R} works.
I will also show how to complete the same task using packages from the
\texttt{tidyverse}. Using the \texttt{tidyverse} methods is usually
optional --- though once you get the hang of it, you might always use
the \texttt{tidyverse} methods.

\begin{Shaded}
\begin{Highlighting}[]
\CommentTok{\# Install the tidyverse package}
\FunctionTok{install.packages}\NormalTok{(}\StringTok{"tidyverse"}\NormalTok{)}
\end{Highlighting}
\end{Shaded}

\begin{Shaded}
\begin{Highlighting}[]
\CommentTok{\# Load the tidyverse package}
\FunctionTok{library}\NormalTok{(tidyverse)}
\end{Highlighting}
\end{Shaded}

\begin{Shaded}
\begin{Highlighting}[]
\CommentTok{\# List the packages loaded by the tidyverse}
\CommentTok{\# package}
\FunctionTok{tidyverse\_packages}\NormalTok{()}
\end{Highlighting}
\end{Shaded}

\begin{verbatim}
 [1] "broom"         "cli"           "crayon"        "dbplyr"       
 [5] "dplyr"         "dtplyr"        "forcats"       "ggplot2"      
 [9] "googledrive"   "googlesheets4" "haven"         "hms"          
[13] "httr"          "jsonlite"      "lubridate"     "magrittr"     
[17] "modelr"        "pillar"        "purrr"         "readr"        
[21] "readxl"        "reprex"        "rlang"         "rstudioapi"   
[25] "rvest"         "stringr"       "tibble"        "tidyr"        
[29] "xml2"          "tidyverse"    
\end{verbatim}

Before we get started with the \texttt{tidyverse}, there are two
important new things to learn about. The first is the pipe operator
\texttt{\%\textgreater{}\%} and the second is the the alternative to a
\emph{data frame} called a \emph{tibble}.

\hypertarget{the-pipe}{%
\subsubsection{The Pipe \%\textgreater\%}\label{the-pipe}}

The pipe operator \texttt{\%\textgreater{}\%}\footnote{Not to be
  confused with the operator \texttt{\textbar{}}, which means ``or'' and
  whose symbol is also called ``pipe''.} is introduced by the
\texttt{magrittr} package\footnote{Loading \texttt{dplyr} will also let
  you use it.} and it is extremely useful. The pipe operator passes the
output of a function to the first argument of the next function, which
mean you can chain several steps together.

For example, lets find the mean year of birth in our data. We already
know that when the pre-vowel contexts are removed, the mean year of
birth is 1969.

\begin{tcolorbox}[enhanced jigsaw, breakable, opacityback=0, colframe=quarto-callout-tip-color-frame, opacitybacktitle=0.6, toptitle=1mm, toprule=.15mm, title=\textcolor{quarto-callout-tip-color}{\faLightbulb}\hspace{0.5em}{Get the data first}, coltitle=black, colback=white, colbacktitle=quarto-callout-tip-color!10!white, left=2mm, leftrule=.75mm, titlerule=0mm, arc=.35mm, bottomtitle=1mm, rightrule=.15mm, bottomrule=.15mm]

If you don't have the \texttt{td} data loaded in \emph{R}, go back to
\href{https://lingmethodshub.github.io/content/R/lvc_r/020_lvcr.html}{Getting
Your Data into \emph{R}} and run the code.

\end{tcolorbox}

\begin{Shaded}
\begin{Highlighting}[]
\CommentTok{\# Find mean YOB using mean() function}
\FunctionTok{mean}\NormalTok{(td}\SpecialCharTok{$}\NormalTok{YOB)}
\end{Highlighting}
\end{Shaded}

\begin{verbatim}
[1] 1969.447
\end{verbatim}

\begin{Shaded}
\begin{Highlighting}[]
\CommentTok{\# Find the mean YOB by piping the td data to the}
\CommentTok{\# mean() function}
\NormalTok{td}\SpecialCharTok{$}\NormalTok{YOB }\SpecialCharTok{\%\textgreater{}\%}
    \FunctionTok{mean}\NormalTok{()}
\end{Highlighting}
\end{Shaded}

\begin{verbatim}
[1] 1969.447
\end{verbatim}

The functionality of \texttt{\%\textgreater{}\%} might seem trivial at
this point; however, when you need to perform multiple tasks
sequentially, it saves a lot of time and space when writing your code.

\hypertarget{tibbles}{%
\subsubsection{Tibbles}\label{tibbles}}

A \emph{tibble} is an updated version of a \emph{data frame}.
\emph{Tibbles} keep the features that have stood the test of time, and
drop the features that used to be convenient but are now frustrating
(i.e.~converting character vectors to factors). For our purposes, the
difference between the two is negligible, but you should be aware that
\emph{tibbles} look a bit different from \emph{data frames}. Run these
two commands and compare.

\begin{Shaded}
\begin{Highlighting}[]
\FunctionTok{as.data.frame}\NormalTok{(td)}
\end{Highlighting}
\end{Shaded}

\begin{Shaded}
\begin{Highlighting}[]
\FunctionTok{as\_tibble}\NormalTok{(td)}
\end{Highlighting}
\end{Shaded}

Notice that the \emph{tibble} lists the dimensions of the tibble at the
top, as well as the class of each of the columns. It also only displays
the first 10 rows. You'll also notice that the row numbers have reset
when we converted \texttt{td} to a \emph{tibble}. If we want to view the
entire tibble, we can use the \texttt{print()} function and specify the
\texttt{n=} plus the number of rows we want to see, including all rows
(\texttt{n=Inf}). You can see below how the pipe operator makes doing
this pretty easy.

\begin{Shaded}
\begin{Highlighting}[]
\CommentTok{\# Embedding functions}
\FunctionTok{print}\NormalTok{(}\FunctionTok{as\_tibble}\NormalTok{(td), }\AttributeTok{n =} \DecValTok{20}\NormalTok{)}
\end{Highlighting}
\end{Shaded}

The above produces the same as the following:

\begin{Shaded}
\begin{Highlighting}[]
\CommentTok{\# Using \%\textgreater{}\% to pass the results from the first}
\CommentTok{\# function to the second function}
\FunctionTok{as\_tibble}\NormalTok{(td) }\SpecialCharTok{\%\textgreater{}\%}
    \FunctionTok{print}\NormalTok{(}\AttributeTok{n =} \DecValTok{20}\NormalTok{)}
\end{Highlighting}
\end{Shaded}

\begin{verbatim}
# A tibble: 1,189 x 17
   Dep.Var  Stress   Category Morph.Type Before After     Speaker   YOB Sex   Education    Job     After.New Center.Age Age.Group Age_Sex  Phoneme Dep.Var.Full
   <chr>    <chr>    <chr>    <chr>      <fct>  <chr>     <chr>   <int> <chr> <chr>        <chr>   <fct>          <dbl> <fct>     <fct>    <fct>   <fct>       
 1 Realized Stressed Lexical  Mono       Stop   Consonant BOUF65   1965 F     Educated     White   Consonant      -4.45 Middle    Middle_F t       T           
 2 Deletion Stressed Lexical  Mono       Stop   Consonant CHIF55   1955 F     Educated     White   Consonant     -14.4  Middle    Middle_F t       Deletion    
 3 Deletion Stressed Lexical  Mono       Stop   Consonant CHIF55   1955 F     Educated     White   Consonant     -14.4  Middle    Middle_F t       Deletion    
 4 Deletion Stressed Lexical  Mono       Stop   Consonant CLAF52   1952 F     Educated     Service Consonant     -17.4  Middle    Middle_F t       Deletion    
 5 Realized Stressed Lexical  Mono       Stop   Consonant DONM53   1953 M     Educated     Service Consonant     -16.4  Middle    Middle_M t       T           
 6 Deletion Stressed Lexical  Mono       Stop   Consonant DONM58   1958 M     Not Educated Service Consonant     -11.4  Middle    Middle_M t       Deletion    
 7 Deletion Stressed Lexical  Mono       Stop   Consonant DOUF46   1946 F     Educated     Service Consonant     -23.4  Middle    Middle_F t       Deletion    
 8 Deletion Stressed Lexical  Mono       Stop   Consonant GARM42   1942 M     Not Educated Blue    Consonant     -27.4  Old       Old_M    t       Deletion    
 9 Deletion Stressed Lexical  Mono       Stop   Consonant GREM45   1945 M     Not Educated Blue    Consonant     -24.4  Middle    Middle_M t       Deletion    
10 Deletion Stressed Lexical  Mono       Stop   Consonant HOLF49   1949 F     Educated     Service Consonant     -20.4  Middle    Middle_F t       Deletion    
11 Deletion Stressed Lexical  Mono       Stop   Consonant HOLM52   1952 M     Not Educated Blue    Consonant     -17.4  Middle    Middle_M t       Deletion    
12 Deletion Stressed Lexical  Mono       Stop   Consonant INGM84   1984 M     Educated     Service Consonant      14.6  Young     Young_M  t       Deletion    
13 Deletion Stressed Lexical  Mono       Stop   Consonant INGM87   1987 M     Educated     Service Consonant      17.6  Young     Young_M  t       Deletion    
14 Deletion Stressed Lexical  Mono       Stop   Consonant KAYF29   1929 F     Not Educated Service Consonant     -40.4  Old       Old_F    t       Deletion    
15 Deletion Stressed Lexical  Mono       Stop   Consonant KAYM29   1929 M     Not Educated Blue    Consonant     -40.4  Old       Old_M    t       Deletion    
16 Realized Stressed Lexical  Mono       Stop   Consonant LATF53   1953 F     Educated     Service Consonant     -16.4  Middle    Middle_F t       T           
17 Realized Stressed Lexical  Mono       Stop   Consonant LEOF66   1966 F     Educated     White   Consonant      -3.45 Middle    Middle_F t       T           
18 Deletion Stressed Lexical  Mono       Stop   Consonant MOFM55   1955 M     Educated     White   Consonant     -14.4  Middle    Middle_M t       Deletion    
19 Deletion Stressed Lexical  Mono       Stop   Consonant NATF84   1984 F     Educated     Service Consonant      14.6  Young     Young_F  t       Deletion    
20 Deletion Stressed Lexical  Mono       Stop   Consonant NEIF49   1949 F     Educated     Service Consonant     -20.4  Middle    Middle_F t       Deletion    
# ... with 1,169 more rows
\end{verbatim}

\hypertarget{getting-a-glimpse}{%
\subsubsection{\texorpdfstring{Getting a
\texttt{glimpse()}}{Getting a glimpse()}}\label{getting-a-glimpse}}

Another useful addition to data exploration is the \texttt{glimpse()}
function from the \texttt{pilllar} package and re-exported by
\texttt{dplyr}. The \texttt{glipmpse()} function is like a cross between
\texttt{print()} (which shows the data) and \texttt{str()} (which shows
the structure of the data). I use \texttt{glimpse()} almost as
frequently as I use \texttt{summary()}. In fact, if you have very wide
data, i.e., with lots of columns, \texttt{glimpse()} may prove more
useful than \texttt{summary()} for getting a quick snapshot of your
data. \texttt{glimpse()} shows the number of rows, the number of
columns, the name of each column, its class, and however many values in
each column as will fit horizontally in the console.

\begin{Shaded}
\begin{Highlighting}[]
\FunctionTok{glimpse}\NormalTok{(td)}
\end{Highlighting}
\end{Shaded}

\begin{verbatim}
Rows: 1,189
Columns: 17
$ Dep.Var      <chr> "Realized", "Deletion", "Deletion", "Deletion", "Realized~
$ Stress       <chr> "Stressed", "Stressed", "Stressed", "Stressed", "Stressed~
$ Category     <chr> "Lexical", "Lexical", "Lexical", "Lexical", "Lexical", "L~
$ Morph.Type   <chr> "Mono", "Mono", "Mono", "Mono", "Mono", "Mono", "Mono", "~
$ Before       <fct> Stop, Stop, Stop, Stop, Stop, Stop, Stop, Stop, Stop, Sto~
$ After        <chr> "Consonant", "Consonant", "Consonant", "Consonant", "Cons~
$ Speaker      <chr> "BOUF65", "CHIF55", "CHIF55", "CLAF52", "DONM53", "DONM58~
$ YOB          <int> 1965, 1955, 1955, 1952, 1953, 1958, 1946, 1942, 1945, 194~
$ Sex          <chr> "F", "F", "F", "F", "M", "M", "F", "M", "M", "F", "M", "M~
$ Education    <chr> "Educated", "Educated", "Educated", "Educated", "Educated~
$ Job          <chr> "White", "White", "White", "Service", "Service", "Service~
$ After.New    <fct> Consonant, Consonant, Consonant, Consonant, Consonant, Co~
$ Center.Age   <dbl> -4.446594, -14.446594, -14.446594, -17.446594, -16.446594~
$ Age.Group    <fct> Middle, Middle, Middle, Middle, Middle, Middle, Middle, O~
$ Age_Sex      <fct> Middle_F, Middle_F, Middle_F, Middle_F, Middle_M, Middle_~
$ Phoneme      <fct> t, t, t, t, t, t, t, t, t, t, t, t, t, t, t, t, t, t, t, ~
$ Dep.Var.Full <fct> T, Deletion, Deletion, Deletion, T, Deletion, Deletion, D~
\end{verbatim}

\hypertarget{manipulating-data-with-dplyr}{%
\subsubsection{\texorpdfstring{Manipulating data with
\texttt{dplyr}}{Manipulating data with dplyr}}\label{manipulating-data-with-dplyr}}

The \texttt{dplyr} package is great for manipulating data in a data
frame/tibble. Some common things that \texttt{diplyr} can do include:

\begin{longtable}[]{@{}ll@{}}
\toprule()
Function & Description \\
\midrule()
\endhead
\texttt{mutate()} & add new variables or modify existing ones \\
\texttt{select()} & select variables \\
\texttt{filter()} & filter \\
\texttt{summarize()} & summarize/reduce \\
\texttt{arrange()} & sort \\
\texttt{group\_by()} & group \\
\texttt{rename()} & rename columns \\
\bottomrule()
\end{longtable}

Lets redo all our data manipulation of \texttt{td} but with
\texttt{dplyr} and its pipe \texttt{\%\textgreater{}\%} operator

\begin{Shaded}
\begin{Highlighting}[]
\CommentTok{\# Read in token file}
\NormalTok{td }\OtherTok{\textless{}{-}} \FunctionTok{read.delim}\NormalTok{(}\StringTok{"Data/deletiondata.txt"}\NormalTok{)}
\end{Highlighting}
\end{Shaded}

or\ldots{}

\begin{Shaded}
\begin{Highlighting}[]
\CommentTok{\# Read in token file}
\NormalTok{td }\OtherTok{\textless{}{-}} \FunctionTok{read.delim}\NormalTok{(}\StringTok{"https://www.dropbox.com/s/jxlfuogea3lx2pu/deletiondata.txt?dl=1"}\NormalTok{)}
\end{Highlighting}
\end{Shaded}

then\ldots{}

\begin{Shaded}
\begin{Highlighting}[]
\CommentTok{\# Subset data to remove previous \textquotesingle{}Vowel\textquotesingle{}}
\CommentTok{\# contexts: filter td to include everything that}
\CommentTok{\# is not \textquotesingle{}Vowel\textquotesingle{} in the column Before}
\NormalTok{td }\OtherTok{\textless{}{-}}\NormalTok{ td }\SpecialCharTok{\%\textgreater{}\%}
    \FunctionTok{filter}\NormalTok{(Before }\SpecialCharTok{!=} \StringTok{"Vowel"}\NormalTok{)}

\CommentTok{\# Re{-}code \textquotesingle{}H\textquotesingle{} to be \textquotesingle{}Consonant\textquotesingle{} in a new column:}
\CommentTok{\# create a new column called After.New that}
\CommentTok{\# equals a re{-}code of After in which H is}
\CommentTok{\# re{-}coded as Consonant}
\NormalTok{td }\OtherTok{\textless{}{-}}\NormalTok{ td }\SpecialCharTok{\%\textgreater{}\%}
    \FunctionTok{mutate}\NormalTok{(}\AttributeTok{After.New =} \FunctionTok{recode}\NormalTok{(After, }\AttributeTok{H =} \StringTok{"Consonant"}\NormalTok{))}

\CommentTok{\# Center Year of Birth: create a new column}
\CommentTok{\# called Center.Age equal to the YOB column but}
\CommentTok{\# scaled}
\NormalTok{td }\OtherTok{\textless{}{-}}\NormalTok{ td }\SpecialCharTok{\%\textgreater{}\%}
    \FunctionTok{mutate}\NormalTok{(}\AttributeTok{Center.Age =} \FunctionTok{as.numeric}\NormalTok{(}\FunctionTok{scale}\NormalTok{(YOB, }\AttributeTok{scale =} \ConstantTok{FALSE}\NormalTok{)))}

\CommentTok{\# Create Age.Group: cut YOB into discrete}
\CommentTok{\# categories.}
\NormalTok{td }\OtherTok{\textless{}{-}}\NormalTok{ td }\SpecialCharTok{\%\textgreater{}\%}
    \FunctionTok{mutate}\NormalTok{(}\AttributeTok{Age.Group =} \FunctionTok{cut}\NormalTok{(YOB, }\AttributeTok{breaks =} \FunctionTok{c}\NormalTok{(}\SpecialCharTok{{-}}\ConstantTok{Inf}\NormalTok{, }\DecValTok{1944}\NormalTok{,}
        \DecValTok{1979}\NormalTok{, }\ConstantTok{Inf}\NormalTok{), }\AttributeTok{labels =} \FunctionTok{c}\NormalTok{(}\StringTok{"Old"}\NormalTok{, }\StringTok{"Middle"}\NormalTok{, }\StringTok{"Young"}\NormalTok{)))}
\end{Highlighting}
\end{Shaded}

Before we continue, a note about the \texttt{cut()} function. The
\texttt{breaks=} option is a concatenated list of boundaries. It should
start and end with \texttt{-Inf} and \texttt{Inf} (negative and positive
infinity) as these will be the lower and upper bounds. The other values
are the boundaries or cut-off points. By default \texttt{cut()} has the
setting \texttt{right=TRUE}, which means the boundary values are
considered the last value in a group (e.g., rightmost value). Above,
this means \texttt{1944} will be the highest value in the \texttt{Old}
category and \texttt{1979} will the the highest value in the
\texttt{Middle} category. To reverse this you can add the option
\texttt{right=FALSE} in which case 1944 would be the lowest value in the
\texttt{Middle} category (e.g.~leftmost value) and 1979 would be the
lowest value in the \texttt{Young} category.

Let's continue.

\begin{Shaded}
\begin{Highlighting}[]
\CommentTok{\# Combine Age and Sex: use the unite() function}
\CommentTok{\# from the tidyr package, if remove=TRUE the}
\CommentTok{\# original Age.Group and Sex columns will be}
\CommentTok{\# deleted}
\NormalTok{td }\OtherTok{\textless{}{-}}\NormalTok{ td }\SpecialCharTok{\%\textgreater{}\%}
    \FunctionTok{unite}\NormalTok{(}\StringTok{"Age\_Sex"}\NormalTok{, }\FunctionTok{c}\NormalTok{(Age.Group, Sex), }\AttributeTok{sep =} \StringTok{"\_"}\NormalTok{,}
        \AttributeTok{remove =} \ConstantTok{FALSE}\NormalTok{)}

\CommentTok{\# Break Phoneme.Dep.Var into two columns: same as}
\CommentTok{\# before, but with td passed to mutate() by the}
\CommentTok{\# \%\textgreater{}\% operator}
\NormalTok{td }\OtherTok{\textless{}{-}}\NormalTok{ td }\SpecialCharTok{\%\textgreater{}\%}
    \FunctionTok{mutate}\NormalTok{(}\AttributeTok{Phoneme =} \FunctionTok{sub}\NormalTok{(}\StringTok{"\^{}(.)({-}{-}.*)$"}\NormalTok{, }\StringTok{"}\SpecialCharTok{\textbackslash{}\textbackslash{}}\StringTok{1"}\NormalTok{, Phoneme.Dep.Var),}
        \AttributeTok{Dep.Var.Full =} \FunctionTok{sub}\NormalTok{(}\StringTok{"\^{}(.{-}{-})(.*)$"}\NormalTok{, }\StringTok{"}\SpecialCharTok{\textbackslash{}\textbackslash{}}\StringTok{2"}\NormalTok{, Phoneme.Dep.Var),}
        \AttributeTok{Phoneme.Dep.Var =} \ConstantTok{NULL}\NormalTok{)}
\end{Highlighting}
\end{Shaded}

At this point we have done everything except partition the data and
re-center YOB in the partitioned data frames. You may ask, ``How is this
better?''. Well, the answer is that because all these modifications feed
into one another, we can actually include them all together in one
serialized operation. Behold!

All of the above code can be simplified as follows:

or\ldots{}

\begin{Shaded}
\begin{Highlighting}[]
\CommentTok{\# Read in token file}
\NormalTok{td }\OtherTok{\textless{}{-}} \FunctionTok{read.delim}\NormalTok{(}\StringTok{"https://www.dropbox.com/s/jxlfuogea3lx2pu/deletiondata.txt?dl=1"}\NormalTok{)}
\end{Highlighting}
\end{Shaded}

then\ldots{}

\begin{Shaded}
\begin{Highlighting}[]
\CommentTok{\# Subset data to remove previous \textquotesingle{}Vowel\textquotesingle{} contexts, }
\CommentTok{\# then modify several columns with mutate, }
\CommentTok{\# then convert any character column to a factor column}
\NormalTok{td }\OtherTok{\textless{}{-}}\NormalTok{ td }\SpecialCharTok{\%\textgreater{}\%}
      \FunctionTok{filter}\NormalTok{(Before }\SpecialCharTok{!=} \StringTok{"Vowel"}\NormalTok{)}\SpecialCharTok{\%\textgreater{}\%}
      \FunctionTok{mutate}\NormalTok{(}
        \AttributeTok{After.New =} \FunctionTok{recode}\NormalTok{(After, }\StringTok{"H"} \OtherTok{=} \StringTok{"Consonant"}\NormalTok{), }
        \AttributeTok{Center.Age =} \FunctionTok{as.numeric}\NormalTok{(}\FunctionTok{scale}\NormalTok{(YOB, }\AttributeTok{scale =} \ConstantTok{FALSE}\NormalTok{)), }
        \AttributeTok{Age.Group =} \FunctionTok{cut}\NormalTok{(YOB, }\AttributeTok{breaks =} \FunctionTok{c}\NormalTok{(}\SpecialCharTok{{-}}\ConstantTok{Inf}\NormalTok{, }\DecValTok{1944}\NormalTok{, }\DecValTok{1979}\NormalTok{, }\ConstantTok{Inf}\NormalTok{), }
                        \AttributeTok{labels =} \FunctionTok{c}\NormalTok{(}\StringTok{"Old"}\NormalTok{, }\StringTok{"Middle"}\NormalTok{, }\StringTok{"Young"}\NormalTok{)), }
        \AttributeTok{Phoneme =} \FunctionTok{sub}\NormalTok{(}\StringTok{"\^{}(.)({-}{-}.*)$"}\NormalTok{, }\StringTok{"}\SpecialCharTok{\textbackslash{}\textbackslash{}}\StringTok{1"}\NormalTok{, Phoneme.Dep.Var), }
        \AttributeTok{Dep.Var.Full =} \FunctionTok{sub}\NormalTok{(}\StringTok{"\^{}(.{-}{-})(.*)$"}\NormalTok{, }\StringTok{"}\SpecialCharTok{\textbackslash{}\textbackslash{}}\StringTok{2"}\NormalTok{, Phoneme.Dep.Var), }
        \AttributeTok{Phoneme.Dep.Var =} \ConstantTok{NULL}
\NormalTok{        )}\SpecialCharTok{\%\textgreater{}\%}
      \FunctionTok{mutate\_if}\NormalTok{(is.character, as.factor)}
\end{Highlighting}
\end{Shaded}

Now, doesn't the above look so much cleaner and easier to follow? You'll
notice that after some lines there is a \texttt{\#}. This an optional
way to signal the end of a line of code when your code is broken over
more than one line. Above, the \texttt{mutate()} function could have
been written in one single continuous line, but breaking it up over
multiple lines makes seeing each mutation much easier.

To partition the data we still need separate functions. Also, remember
to re-centre any continuous variables after partioning.

\begin{Shaded}
\begin{Highlighting}[]
\NormalTok{td.young }\OtherTok{\textless{}{-}}\NormalTok{ td }\SpecialCharTok{\%\textgreater{}\%}
    \FunctionTok{filter}\NormalTok{(Age.Group }\SpecialCharTok{==} \StringTok{"Young"}\NormalTok{) }\SpecialCharTok{\%\textgreater{}\%}
    \FunctionTok{mutate}\NormalTok{(}\AttributeTok{Center.Age =} \FunctionTok{as.numeric}\NormalTok{(}\FunctionTok{scale}\NormalTok{(YOB, }\AttributeTok{scale =} \ConstantTok{FALSE}\NormalTok{)))}

\NormalTok{td.middle }\OtherTok{\textless{}{-}}\NormalTok{ td }\SpecialCharTok{\%\textgreater{}\%}
    \FunctionTok{filter}\NormalTok{(Age.Group }\SpecialCharTok{==} \StringTok{"Middle"}\NormalTok{) }\SpecialCharTok{\%\textgreater{}\%}
    \FunctionTok{mutate}\NormalTok{(}\AttributeTok{Center.Age =} \FunctionTok{as.numeric}\NormalTok{(}\FunctionTok{scale}\NormalTok{(YOB, }\AttributeTok{scale =} \ConstantTok{FALSE}\NormalTok{)))}

\NormalTok{td.old }\OtherTok{\textless{}{-}}\NormalTok{ td }\SpecialCharTok{\%\textgreater{}\%}
    \FunctionTok{filter}\NormalTok{(Age.Group }\SpecialCharTok{==} \StringTok{"Old"}\NormalTok{) }\SpecialCharTok{\%\textgreater{}\%}
    \FunctionTok{mutate}\NormalTok{(}\AttributeTok{Center.Age =} \FunctionTok{as.numeric}\NormalTok{(}\FunctionTok{scale}\NormalTok{(YOB, }\AttributeTok{scale =} \ConstantTok{FALSE}\NormalTok{)))}
\end{Highlighting}
\end{Shaded}




\end{document}
