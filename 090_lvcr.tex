% Options for packages loaded elsewhere
\PassOptionsToPackage{unicode}{hyperref}
\PassOptionsToPackage{hyphens}{url}
\PassOptionsToPackage{dvipsnames,svgnames,x11names}{xcolor}
%
\documentclass[
  10pt,
  letterpaper]{article}

\usepackage{amsmath,amssymb}
\usepackage{lmodern}
\usepackage{iftex}
\ifPDFTeX
  \usepackage[T1]{fontenc}
  \usepackage[utf8]{inputenc}
  \usepackage{textcomp} % provide euro and other symbols
\else % if luatex or xetex
  \usepackage{unicode-math}
  \defaultfontfeatures{Scale=MatchLowercase}
  \defaultfontfeatures[\rmfamily]{Ligatures=TeX,Scale=1}
  \setmainfont[]{Charis SIL}
  \setmonofont[]{Monaco}
  \setmathfont[]{Monaco}
\fi
% Use upquote if available, for straight quotes in verbatim environments
\IfFileExists{upquote.sty}{\usepackage{upquote}}{}
\IfFileExists{microtype.sty}{% use microtype if available
  \usepackage[]{microtype}
  \UseMicrotypeSet[protrusion]{basicmath} % disable protrusion for tt fonts
}{}
\makeatletter
\@ifundefined{KOMAClassName}{% if non-KOMA class
  \IfFileExists{parskip.sty}{%
    \usepackage{parskip}
  }{% else
    \setlength{\parindent}{0pt}
    \setlength{\parskip}{6pt plus 2pt minus 1pt}}
}{% if KOMA class
  \KOMAoptions{parskip=half}}
\makeatother
\usepackage{xcolor}
\usepackage[margin = 1in]{geometry}
\setlength{\emergencystretch}{3em} % prevent overfull lines
\setcounter{secnumdepth}{-\maxdimen} % remove section numbering
% Make \paragraph and \subparagraph free-standing
\ifx\paragraph\undefined\else
  \let\oldparagraph\paragraph
  \renewcommand{\paragraph}[1]{\oldparagraph{#1}\mbox{}}
\fi
\ifx\subparagraph\undefined\else
  \let\oldsubparagraph\subparagraph
  \renewcommand{\subparagraph}[1]{\oldsubparagraph{#1}\mbox{}}
\fi

\usepackage{color}
\usepackage{fancyvrb}
\newcommand{\VerbBar}{|}
\newcommand{\VERB}{\Verb[commandchars=\\\{\}]}
\DefineVerbatimEnvironment{Highlighting}{Verbatim}{commandchars=\\\{\}}
% Add ',fontsize=\small' for more characters per line
\usepackage{framed}
\definecolor{shadecolor}{RGB}{241,243,245}
\newenvironment{Shaded}{\begin{snugshade}}{\end{snugshade}}
\newcommand{\AlertTok}[1]{\textcolor[rgb]{0.68,0.00,0.00}{#1}}
\newcommand{\AnnotationTok}[1]{\textcolor[rgb]{0.37,0.37,0.37}{#1}}
\newcommand{\AttributeTok}[1]{\textcolor[rgb]{0.40,0.45,0.13}{#1}}
\newcommand{\BaseNTok}[1]{\textcolor[rgb]{0.68,0.00,0.00}{#1}}
\newcommand{\BuiltInTok}[1]{\textcolor[rgb]{0.00,0.23,0.31}{#1}}
\newcommand{\CharTok}[1]{\textcolor[rgb]{0.13,0.47,0.30}{#1}}
\newcommand{\CommentTok}[1]{\textcolor[rgb]{0.37,0.37,0.37}{#1}}
\newcommand{\CommentVarTok}[1]{\textcolor[rgb]{0.37,0.37,0.37}{\textit{#1}}}
\newcommand{\ConstantTok}[1]{\textcolor[rgb]{0.56,0.35,0.01}{#1}}
\newcommand{\ControlFlowTok}[1]{\textcolor[rgb]{0.00,0.23,0.31}{#1}}
\newcommand{\DataTypeTok}[1]{\textcolor[rgb]{0.68,0.00,0.00}{#1}}
\newcommand{\DecValTok}[1]{\textcolor[rgb]{0.68,0.00,0.00}{#1}}
\newcommand{\DocumentationTok}[1]{\textcolor[rgb]{0.37,0.37,0.37}{\textit{#1}}}
\newcommand{\ErrorTok}[1]{\textcolor[rgb]{0.68,0.00,0.00}{#1}}
\newcommand{\ExtensionTok}[1]{\textcolor[rgb]{0.00,0.23,0.31}{#1}}
\newcommand{\FloatTok}[1]{\textcolor[rgb]{0.68,0.00,0.00}{#1}}
\newcommand{\FunctionTok}[1]{\textcolor[rgb]{0.28,0.35,0.67}{#1}}
\newcommand{\ImportTok}[1]{\textcolor[rgb]{0.00,0.46,0.62}{#1}}
\newcommand{\InformationTok}[1]{\textcolor[rgb]{0.37,0.37,0.37}{#1}}
\newcommand{\KeywordTok}[1]{\textcolor[rgb]{0.00,0.23,0.31}{#1}}
\newcommand{\NormalTok}[1]{\textcolor[rgb]{0.00,0.23,0.31}{#1}}
\newcommand{\OperatorTok}[1]{\textcolor[rgb]{0.37,0.37,0.37}{#1}}
\newcommand{\OtherTok}[1]{\textcolor[rgb]{0.00,0.23,0.31}{#1}}
\newcommand{\PreprocessorTok}[1]{\textcolor[rgb]{0.68,0.00,0.00}{#1}}
\newcommand{\RegionMarkerTok}[1]{\textcolor[rgb]{0.00,0.23,0.31}{#1}}
\newcommand{\SpecialCharTok}[1]{\textcolor[rgb]{0.37,0.37,0.37}{#1}}
\newcommand{\SpecialStringTok}[1]{\textcolor[rgb]{0.13,0.47,0.30}{#1}}
\newcommand{\StringTok}[1]{\textcolor[rgb]{0.13,0.47,0.30}{#1}}
\newcommand{\VariableTok}[1]{\textcolor[rgb]{0.07,0.07,0.07}{#1}}
\newcommand{\VerbatimStringTok}[1]{\textcolor[rgb]{0.13,0.47,0.30}{#1}}
\newcommand{\WarningTok}[1]{\textcolor[rgb]{0.37,0.37,0.37}{\textit{#1}}}

\providecommand{\tightlist}{%
  \setlength{\itemsep}{0pt}\setlength{\parskip}{0pt}}\usepackage{longtable,booktabs,array}
\usepackage{calc} % for calculating minipage widths
% Correct order of tables after \paragraph or \subparagraph
\usepackage{etoolbox}
\makeatletter
\patchcmd\longtable{\par}{\if@noskipsec\mbox{}\fi\par}{}{}
\makeatother
% Allow footnotes in longtable head/foot
\IfFileExists{footnotehyper.sty}{\usepackage{footnotehyper}}{\usepackage{footnote}}
\makesavenoteenv{longtable}
\usepackage{graphicx}
\makeatletter
\def\maxwidth{\ifdim\Gin@nat@width>\linewidth\linewidth\else\Gin@nat@width\fi}
\def\maxheight{\ifdim\Gin@nat@height>\textheight\textheight\else\Gin@nat@height\fi}
\makeatother
% Scale images if necessary, so that they will not overflow the page
% margins by default, and it is still possible to overwrite the defaults
% using explicit options in \includegraphics[width, height, ...]{}
\setkeys{Gin}{width=\maxwidth,height=\maxheight,keepaspectratio}
% Set default figure placement to htbp
\makeatletter
\def\fps@figure{htbp}
\makeatother
\newlength{\cslhangindent}
\setlength{\cslhangindent}{1.5em}
\newlength{\csllabelwidth}
\setlength{\csllabelwidth}{3em}
\newlength{\cslentryspacingunit} % times entry-spacing
\setlength{\cslentryspacingunit}{\parskip}
\newenvironment{CSLReferences}[2] % #1 hanging-ident, #2 entry spacing
 {% don't indent paragraphs
  \setlength{\parindent}{0pt}
  % turn on hanging indent if param 1 is 1
  \ifodd #1
  \let\oldpar\par
  \def\par{\hangindent=\cslhangindent\oldpar}
  \fi
  % set entry spacing
  \setlength{\parskip}{#2\cslentryspacingunit}
 }%
 {}
\usepackage{calc}
\newcommand{\CSLBlock}[1]{#1\hfill\break}
\newcommand{\CSLLeftMargin}[1]{\parbox[t]{\csllabelwidth}{#1}}
\newcommand{\CSLRightInline}[1]{\parbox[t]{\linewidth - \csllabelwidth}{#1}\break}
\newcommand{\CSLIndent}[1]{\hspace{\cslhangindent}#1}

\usepackage{tabularx}
\usepackage{threeparttable}
\usepackage{booktabs}
\usepackage{tipa}
\let\Oldtexttt\texttt
\renewcommand\texttt[1]{{\ttfamily\color{BrickRed}#1}}
\usepackage{authoraftertitle}
\usepackage{fancyhdr}
\pagestyle{fancy}
\rfoot{\copyright Matt Hunt Gardner}
\cfoot{\thepage}
\lhead{Doing LVC with \textit{R}: \MyTitle}
\rhead{}
\makeatletter
\@ifpackageloaded{tcolorbox}{}{\usepackage[many]{tcolorbox}}
\@ifpackageloaded{fontawesome5}{}{\usepackage{fontawesome5}}
\definecolor{quarto-callout-color}{HTML}{909090}
\definecolor{quarto-callout-note-color}{HTML}{0758E5}
\definecolor{quarto-callout-important-color}{HTML}{CC1914}
\definecolor{quarto-callout-warning-color}{HTML}{EB9113}
\definecolor{quarto-callout-tip-color}{HTML}{00A047}
\definecolor{quarto-callout-caution-color}{HTML}{FC5300}
\definecolor{quarto-callout-color-frame}{HTML}{acacac}
\definecolor{quarto-callout-note-color-frame}{HTML}{4582ec}
\definecolor{quarto-callout-important-color-frame}{HTML}{d9534f}
\definecolor{quarto-callout-warning-color-frame}{HTML}{f0ad4e}
\definecolor{quarto-callout-tip-color-frame}{HTML}{02b875}
\definecolor{quarto-callout-caution-color-frame}{HTML}{fd7e14}
\makeatother
\makeatletter
\makeatother
\makeatletter
\makeatother
\makeatletter
\@ifpackageloaded{caption}{}{\usepackage{caption}}
\AtBeginDocument{%
\ifdefined\contentsname
  \renewcommand*\contentsname{Table of contents}
\else
  \newcommand\contentsname{Table of contents}
\fi
\ifdefined\listfigurename
  \renewcommand*\listfigurename{List of Figures}
\else
  \newcommand\listfigurename{List of Figures}
\fi
\ifdefined\listtablename
  \renewcommand*\listtablename{List of Tables}
\else
  \newcommand\listtablename{List of Tables}
\fi
\ifdefined\figurename
  \renewcommand*\figurename{Figure}
\else
  \newcommand\figurename{Figure}
\fi
\ifdefined\tablename
  \renewcommand*\tablename{Table}
\else
  \newcommand\tablename{Table}
\fi
}
\@ifpackageloaded{float}{}{\usepackage{float}}
\floatstyle{ruled}
\@ifundefined{c@chapter}{\newfloat{codelisting}{h}{lop}}{\newfloat{codelisting}{h}{lop}[chapter]}
\floatname{codelisting}{Listing}
\newcommand*\listoflistings{\listof{codelisting}{List of Listings}}
\makeatother
\makeatletter
\@ifpackageloaded{caption}{}{\usepackage{caption}}
\@ifpackageloaded{subcaption}{}{\usepackage{subcaption}}
\makeatother
\makeatletter
\@ifpackageloaded{tcolorbox}{}{\usepackage[many]{tcolorbox}}
\makeatother
\makeatletter
\@ifundefined{shadecolor}{\definecolor{shadecolor}{rgb}{.97, .97, .97}}
\makeatother
\makeatletter
\makeatother
\ifLuaTeX
  \usepackage{selnolig}  % disable illegal ligatures
\fi
\IfFileExists{bookmark.sty}{\usepackage{bookmark}}{\usepackage{hyperref}}
\IfFileExists{xurl.sty}{\usepackage{xurl}}{} % add URL line breaks if available
\urlstyle{same} % disable monospaced font for URLs
% Make links footnotes instead of hotlinks:
\DeclareRobustCommand{\href}[2]{#2\footnote{\url{#1}}}
\hypersetup{
  pdftitle={Random Forests: The Basics},
  pdfauthor={Matt Hunt Gardner},
  colorlinks=true,
  linkcolor={blue},
  filecolor={Maroon},
  citecolor={Blue},
  urlcolor={Blue},
  pdfcreator={LaTeX via pandoc}}

\title{Random Forests: The Basics}
\usepackage{etoolbox}
\makeatletter
\providecommand{\subtitle}[1]{% add subtitle to \maketitle
  \apptocmd{\@title}{\par {\large #1 \par}}{}{}
}
\makeatother
\subtitle{from
\href{https://lingmethodshub.github.io/content/R/lvc_r/}{Doing LVC with
\emph{R}}}
\author{Matt Hunt Gardner}
\date{3/10/23}

\begin{document}
\maketitle
\ifdefined\Shaded\renewenvironment{Shaded}{\begin{tcolorbox}[interior hidden, borderline west={3pt}{0pt}{shadecolor}, frame hidden, sharp corners, breakable, boxrule=0pt, enhanced]}{\end{tcolorbox}}\fi

Another useful type of analysis available as part of the
\texttt{partykit} package is a random forest analysis. This algorithm
determines the most common classification tree (like the one above)
among a large collection of trees built on random subsets of a dataset.
Random forests are useful because they can rank the relative importance
of independent variables with respect to a dependent variable. They can
also simultaneously test variables that are (multi)collinear. Logistic
regression (discussed in
\href{https://lingmethodshub.github.io/content/R/lvc_r/110_lvcr.html}{Mixed-Efects
Logistic Regression Analysis}) assumes little or no collinearity among
independent predictor variables. Variables like lexical status
(\texttt{Category}) and morphological type (\texttt{Morph.Type}) are
highly correlated (i.e., semi-weak and past-tense verbs are all
lexical), thus should not be included as predictors in the same logistic
regression model.\footnote{See Tagliamonte (2012) and Tagliamonte and
  Baayen (2012) for a variationist-focused explanation of random forest
  analysis and its relationship to other analyses.}

Random forest analysis take a lot of computing power and time. I usually
leave my computer to run the analysis overnight. The bigger your dataset
and the more predictors you include, the longer the analysis will take.

The first step is to first set the ``seed'' before running your random
forest analysis.

\begin{Shaded}
\begin{Highlighting}[]
\CommentTok{\# Set Seed}
\FunctionTok{set.seed}\NormalTok{(}\DecValTok{123456}\NormalTok{)}
\end{Highlighting}
\end{Shaded}

The random forest algorithm finds the most common classification tree
from those built on random samplings of the data. Setting the seed for
\emph{R}'s random number generator means you can both generate random
samples of your data for creating these trees, and you can also make
your analysis reproducible. Think of each seed as a list of random
numbers. Above, the seed \texttt{123456} is used, but the seed you
select is arbitrary. The one time it is important to select a specific
seed is if you are trying to reproduce results. In that case, you must
use the same seed as the previous analysis. Keeping a record of your
seed setting in a script file and reporting your seed setting in a
manuscript is therefore very important.

You build your random forest formula just like you built the formula for
your \texttt{ctree()}, but with the function \texttt{cforest()}. The
formula here uses \texttt{Dep.Var} as the dependent variable and
\texttt{Stress}, \texttt{Category}, \texttt{Morph.Type},
\texttt{Before}, \texttt{After}, \texttt{Sex}, \texttt{Education},
\texttt{Job}, \texttt{After.New}, \texttt{Center.Age},
\texttt{Age.Group}, and \texttt{Phoneme} as predictor variables. The
data tested is set using \texttt{data=td}.

\begin{tcolorbox}[enhanced jigsaw, rightrule=.15mm, left=2mm, toprule=.15mm, breakable, colback=white, colframe=quarto-callout-tip-color-frame, leftrule=.75mm, bottomtitle=1mm, title=\textcolor{quarto-callout-tip-color}{\faLightbulb}\hspace{0.5em}{Get the data first}, colbacktitle=quarto-callout-tip-color!10!white, coltitle=black, arc=.35mm, titlerule=0mm, bottomrule=.15mm, toptitle=1mm, opacityback=0, opacitybacktitle=0.6]

If you don't have the \texttt{td} data loaded in \emph{R}, go back to
\href{https://lingmethodshub.github.io/content/R/lvc_r/050_lvcr.html}{Doing
it all again, but \texttt{tidy}} and run the code.

\end{tcolorbox}

The setting \texttt{ntree=128} specifies the number of trees to grow. I
have run analyses where I've specified 10,000 trees and it took almost
24 hours to compute on my 2015 MacBook Pro. As a rule, the more
data/independent variables you have, the more trees you should specify;
however, Oshiro, Perez, and Baranauskas (2012) point out, based on their
test with 29 different datasets, that after 128 trees there is no
significant improvement in the accuracy of the resulting random forest.
As a random forest is never my only analysis, I now specify 128 trees
for casual data exploration. I use at least 5,000 for any random forest
that I include in a published manuscript.

The \texttt{cforest()} function creates the random forest, but you still
need to determine which independent variables do the best job of
describing the data averaged over all of the trees created. To do this
you use the function \texttt{varimp()} on the result of
\texttt{cforest()} (here the object \texttt{td.cforest}). The setting
\texttt{conditional=TRUE} is important when independent variables are
collinear; it should always be used. In simple terms, this setting
ensures that, if two independent variables are collinear, only the
variable that does a better job at explaining the variation in the data
will be considered as an important descriptor and the other will be
ignored.

\begin{tcolorbox}[enhanced jigsaw, rightrule=.15mm, left=2mm, toprule=.15mm, breakable, colback=white, colframe=quarto-callout-note-color-frame, leftrule=.75mm, bottomtitle=1mm, title=\textcolor{quarto-callout-note-color}{\faInfo}\hspace{0.5em}{Note}, colbacktitle=quarto-callout-note-color!10!white, coltitle=black, arc=.35mm, titlerule=0mm, bottomrule=.15mm, toptitle=1mm, opacityback=0, opacitybacktitle=0.6]

You'll notice that I use the term descriptor for random forests and
predictor for regression analysis. This is because regression analysis
tests what independent variables do the best job of predicting whether a
given token will be \texttt{Deletion} or \texttt{Realization}. The
random forest analysis doesn't predict, instead it determines the most
useful way of dividing the data in order to best explain when
\texttt{Deletion} and \texttt{Realization} occurs.

\end{tcolorbox}

The \texttt{varimp()} function is the part of this analysis that takes a
long time and lots of computing power. Here is where the size of your
dataset, the number of trees grown, and the number of independent
variables all contribute to how much time and computing power is needed.

The result of \texttt{varimp()} is a list, rather than a dataframe. To
turn it into a dataframe we need to use the function \texttt{melt()}.
The setting \texttt{id.var=NULL} turns the names of each independent
variable into a value in a column called \texttt{variable}. If this
setting is not used \emph{R} will instead use the independent variable
names as row names, which is not useful for graphing.

Given that \texttt{varimp()} often takes a long time, it is a good idea
to save the results as a text file. That way you don't have to re-run
the analysis if you want to refer to it at a later date. As above, doing
so will automatically melt the data, but it won't include column names.
If you read this file back into \texttt{R} you need to assign the column
names using the function
\texttt{names(td.cforest.results)\textless{}-c("variable",\ "value")}.

\begin{Shaded}
\begin{Highlighting}[]
\CommentTok{\# Calculate a Random Forest testing Stress,}
\CommentTok{\# Category, Morph.Type, Before, After, Sex,}
\CommentTok{\# Education, Job, After.New, Center.Age,}
\CommentTok{\# Age.Group , and Phoneme}
\FunctionTok{library}\NormalTok{(partykit)}
\NormalTok{td.cforest }\OtherTok{\textless{}{-}} \FunctionTok{cforest}\NormalTok{(Dep.Var }\SpecialCharTok{\textasciitilde{}}\NormalTok{ Stress }\SpecialCharTok{+}\NormalTok{ Category }\SpecialCharTok{+}
\NormalTok{    Morph.Type }\SpecialCharTok{+}\NormalTok{ Before }\SpecialCharTok{+}\NormalTok{ After }\SpecialCharTok{+}\NormalTok{ Sex }\SpecialCharTok{+}\NormalTok{ Education }\SpecialCharTok{+}
\NormalTok{    Job }\SpecialCharTok{+}\NormalTok{ After.New }\SpecialCharTok{+}\NormalTok{ Center.Age }\SpecialCharTok{+}\NormalTok{ Age.Group }\SpecialCharTok{+}\NormalTok{ Phoneme,}
    \AttributeTok{data =}\NormalTok{ td, }\AttributeTok{ntree =} \DecValTok{128}\NormalTok{)}
\NormalTok{td.cforest.varimp }\OtherTok{\textless{}{-}} \FunctionTok{varimp}\NormalTok{(td.cforest, }\AttributeTok{conditional =} \ConstantTok{TRUE}\NormalTok{)}

\FunctionTok{library}\NormalTok{(reshape2)}
\NormalTok{td.cforest.results }\OtherTok{\textless{}{-}} \FunctionTok{melt}\NormalTok{(}\FunctionTok{data.frame}\NormalTok{(}\FunctionTok{as.list}\NormalTok{(td.cforest.varimp)),}
    \AttributeTok{id.vars =} \ConstantTok{NULL}\NormalTok{)}

\CommentTok{\# Write Random Forest to File}
\FunctionTok{write.table}\NormalTok{(td.cforest.results, }\AttributeTok{file =} \StringTok{"Data/tdCForestVarimp.txt"}\NormalTok{,}
    \AttributeTok{quote =} \ConstantTok{FALSE}\NormalTok{, }\AttributeTok{row.names =} \ConstantTok{FALSE}\NormalTok{)}
\end{Highlighting}
\end{Shaded}

\begin{Shaded}
\begin{Highlighting}[]
\CommentTok{\# Read Random Forest from File}
\NormalTok{td.cforest.results }\OtherTok{\textless{}{-}} \FunctionTok{read.table}\NormalTok{(}\AttributeTok{file =} \StringTok{"Data/tdCForestVarimp.txt"}\NormalTok{,}
    \AttributeTok{header =} \ConstantTok{TRUE}\NormalTok{, }\AttributeTok{row.names =} \ConstantTok{NULL}\NormalTok{)}
\FunctionTok{print}\NormalTok{(td.cforest.results)}
\end{Highlighting}
\end{Shaded}

\begin{verbatim}
     variable         value
1      Stress  0.0092881683
2    Category  0.0010024089
3  Morph.Type  0.0428405420
4      Before  0.0891103455
5       After  0.0225015590
6         Sex -0.0046620498
7   Education  0.0142773676
8         Job  0.0190235499
9   After.New  0.0108922165
10 Center.Age  0.0286954756
11  Age.Group  0.0005741748
12    Phoneme -0.0052887491
\end{verbatim}

Now that you have the variable importance of each independent variable
in a dataframe you can compare the values to determine which variables
do the best job at explaining the variation. You could do this by just
looking at the values --- higher values are more useful, lower values
are less useful --- however, the more common way to compare the
variables is by graphing them.

Older manuals suggest using the \emph{R} base graphics function
\texttt{dotplot()} to visualize this data. I prefer using
\texttt{ggplot()} because it is so much more customizable. As in
previous chapters, I won't list all the different ways you can customize
a \texttt{ggplot()}, but I will show you how to make and save a nice
looking random forest graph.

\begin{Shaded}
\begin{Highlighting}[]
\FunctionTok{library}\NormalTok{(ggplot2)}
\CommentTok{\# Reorder variable importance by highest to}
\CommentTok{\# lowest value in column value}
\NormalTok{td.cforest.results}\SpecialCharTok{$}\NormalTok{variable }\OtherTok{\textless{}{-}} \FunctionTok{reorder}\NormalTok{(td.cforest.results}\SpecialCharTok{$}\NormalTok{variable,}
\NormalTok{    td.cforest.results}\SpecialCharTok{$}\NormalTok{value)}
\end{Highlighting}
\end{Shaded}

First you make sure \texttt{ggplot2} is loaded using the
\texttt{library()} function. Next, you need to order the descriptor
variable names in the \texttt{td.cforest.results} column
\texttt{variable} based on the values of the column \texttt{value}. To
do this you use the \texttt{reorder()} function, which takes two
objects: first, the column you want to sort (here
\texttt{td.cforest.results\$variable}); and second, the column you want
to sort by (here \texttt{td.cforest.results\$value}). You assign this
reordered dataframe back as the original dataframe using the assignment
operator \texttt{\textless{}-}.

Alternatively, you could to this in a \texttt{tidy} way using
\texttt{arrange()}

\begin{Shaded}
\begin{Highlighting}[]
\CommentTok{\# Reorder variable importance by highest to}
\CommentTok{\# lowest value in column value, but tidy}
\NormalTok{td.cforest.results }\OtherTok{\textless{}{-}}\NormalTok{ td.cforest.results }\SpecialCharTok{\%\textgreater{}\%}
    \FunctionTok{arrange}\NormalTok{(variable, value)}
\end{Highlighting}
\end{Shaded}

Next you need to find the absolute value of the lowest negative-scoring
descriptor variable. You'll see why in a second. You create a new object
\texttt{td.cforest.min} and, using the assignment operator
\texttt{\textless{}-}, make it equal to the absolute value (using
\texttt{abs()}) of the lowest negative-scoring variable (using
\texttt{min()}). Can can also do this in a \texttt{tidy} way using
\texttt{\%\textgreater{}\%}, though it's not much quicker to type:

\begin{Shaded}
\begin{Highlighting}[]
\CommentTok{\# Get absolute minimum value of column value}
\NormalTok{td.cforest.min }\OtherTok{\textless{}{-}} \FunctionTok{abs}\NormalTok{(}\FunctionTok{min}\NormalTok{(td.cforest.results}\SpecialCharTok{$}\NormalTok{value))}

\CommentTok{\# Get absolute minimum value of column value, but}
\CommentTok{\# tidy}
\NormalTok{td.cforest.min }\OtherTok{\textless{}{-}}\NormalTok{ td.cforest.results}\SpecialCharTok{$}\NormalTok{value }\SpecialCharTok{\%\textgreater{}\%}
    \FunctionTok{min}\NormalTok{() }\SpecialCharTok{\%\textgreater{}\%}
    \FunctionTok{abs}\NormalTok{()}
\end{Highlighting}
\end{Shaded}

Now you have everything you need to visualize the variable importance of
the variables in the random forest. You create a new object
\texttt{dotplot}, which will be the graph created by \texttt{ggplot()}.
The function \texttt{ggplot()} has a special syntax. Generally you first
create the plot using \texttt{ggplot()}; this is the function where you
specify what values are going to serve as the \texttt{x} and \texttt{y}
axes of the graph and also what data will be used. Specifying the
\texttt{x} and \texttt{y} axes values is done inside the aesthetics
(\texttt{aes()}) of the \texttt{gplot()}. These aesthetics are carried
forward to all the other components you add to the plot. If you stop
just at the \texttt{ggplot()} function, then \texttt{dotplot} would only
be the plotting area, plus an \texttt{x} and \texttt{y} axes.

The next step is to add some data points to your plot. You do this by
adding what are called ``geom's'' ({[}'gi.owmz{]}). You want to create a
dotplot, so you use \texttt{geom\_point()}, which will add the
corresponding \texttt{x} and \texttt{y} values (inherited from
\texttt{aes()}) as points. Use \texttt{+} to add the geom to the already
created \texttt{ggplot()}. The geom \texttt{geom\_point()} can be on a
separate line, but the \texttt{+} must be on the same line as the object
being added to.

The next step is to name the \texttt{x}-axis by adding \texttt{xlab()}
with the label in quotation marks. You can remove the \texttt{y}-axis
label by specifying the label as \texttt{NULL}. Much of the
customization of \texttt{ggplot()}'s look is done through the
\texttt{theme()} function. I'm not going to delve too deep into
customization; instead, I'm going to suggest one of the built-in themes:
\texttt{theme\_classic()}, which gives a very bare-bones graph (my
personal preference). There are other built-in themes like
\texttt{theme\_linedraw()}; \texttt{theme\_bw()};
\texttt{theme\_dark()}; etc. \emph{Google} is your best friend when you
want to customize the look of your \texttt{ggplot2} visualizations.

The final element you add to the dotplot is a vertical line where
\texttt{x\ =\ td.cforest.min}. This is the absolute value of the lowest
negative-scoring variable you calculated earlier. The geom
\texttt{geom\_vline()} adds the vertical line. Its first argument is
where you specify the \texttt{xintercept}, i.e., where you want the
vertical line to intercept the \texttt{x}-axis. The second argument is
the \texttt{linetype}; here you specify it as \texttt{2}. There are
different line types in \emph{R}; \texttt{0} is ``blank'', \texttt{1} is
``solid'', \texttt{2} is ``dashed'', \texttt{3} is ``dotted'',
\texttt{4} is ``dotdash'', \texttt{5} is ``longdash'', and \texttt{6} is
``twodash''. You can either use the line type number or name, but the
name must be in quotation marks. The \texttt{alpha} value is a number
between \texttt{0} and \texttt{1} that indicates how transparent to make
the line. Here \texttt{0.5} (50\% transparent) is used.

If you type \texttt{dotplot} in the \emph{R} console and hit execute,
you'll see what your graph looks like.

\begin{Shaded}
\begin{Highlighting}[]
\CommentTok{\# Create object of Random Forest as a dotplot}
\CommentTok{\# using ggplot2}
\FunctionTok{library}\NormalTok{(ggplot2)}
\NormalTok{dotplot }\OtherTok{\textless{}{-}} \FunctionTok{ggplot}\NormalTok{(td.cforest.results, }\FunctionTok{aes}\NormalTok{(}\AttributeTok{x =}\NormalTok{ value,}
    \AttributeTok{y =}\NormalTok{ variable)) }\SpecialCharTok{+} \FunctionTok{geom\_point}\NormalTok{() }\SpecialCharTok{+} \FunctionTok{xlab}\NormalTok{(}\StringTok{"Variable Importance"}\NormalTok{) }\SpecialCharTok{+}
    \FunctionTok{ylab}\NormalTok{(}\ConstantTok{NULL}\NormalTok{) }\SpecialCharTok{+} \FunctionTok{theme\_classic}\NormalTok{() }\SpecialCharTok{+} \FunctionTok{geom\_vline}\NormalTok{(}\AttributeTok{xintercept =}\NormalTok{ td.cforest.min,}
    \AttributeTok{linetype =} \DecValTok{2}\NormalTok{, }\AttributeTok{alpha =} \FloatTok{0.5}\NormalTok{)}
\CommentTok{\# Save dotplot}
\FunctionTok{ggsave}\NormalTok{(}\StringTok{"Data/tdCForest.png"}\NormalTok{, dotplot, }\AttributeTok{width =} \DecValTok{4}\NormalTok{, }\AttributeTok{height =} \DecValTok{4}\NormalTok{,}
    \AttributeTok{units =} \StringTok{"in"}\NormalTok{, }\AttributeTok{dpi =} \DecValTok{300}\NormalTok{)}
\end{Highlighting}
\end{Shaded}

\includegraphics{090_lvcr_files/figure-pdf/unnamed-chunk-9-1.pdf}

To save the graph use the function \texttt{ggsave()}. The first argument
is the save destination. You can specify which type of file you want to
save your image as by using different file extensions: .png, .pdf, etc.
I generally only use PNG or PDF files. The second argument is the name
of the object to save (here \texttt{dotplot}). Next you specify the
\texttt{width} and \texttt{height} of the image you want to create and
the \texttt{units} of the values you specify. You don't need to do this,
but I like to so that I know exactly how big my image will be. I find
this helpful when later inserting images in my manuscripts. It is also a
good way to ensure consistency in size from graph to graph. For PNG you
can also specify the \texttt{dpi} (dots per inch/pixels per inch), which
will be the resolution of the final image. The higher the \texttt{dpi},
the bigger the file, but also the clearer the image will be if you
decide later to zoom in on it or enlarge it. Though, if you need a
bigger image it's better to just specify the exact size using the height
and width arguments, rather than enlarging a smaller image. For graphs
with little to no text \texttt{300} dpi is sufficient. If there is a lot
of text, you may want to bump it up to \texttt{600} dpi. For certain
publishers there may be a specific dpi needed.

In the dotplot above, the descriptor variables are ordered top to bottom
from most important to least important. Variables can be considered
informative if their variable importance is above the absolute value of
the lowest negative-scoring variable. The rationale for this rule of
thumb, according to Strobl, Malley, and Tutz (2009, 342), is that
importance of irrelevant variables varies randomly around zero. In other
words, everything to the right of the dashed line is a useful in
describing \texttt{Deletion} in the data. The further to the right of
the dashed line, the more useful.

The variables with the highest variable importance are \texttt{Before}
and \texttt{Morph.Type}, followed by \texttt{Center.Age},
\texttt{After}, \texttt{Job}, \texttt{Education}, \texttt{After.New} and
\texttt{Stress}. The variables \texttt{Category}, \texttt{Age.Group},
\texttt{Sex}, and \texttt{Phoneme} are all to the left of the dashed
line, so you can consider them as not useful in describing the
variation.

You'll notice that many of the independent variables included are
collinear --- in other words, the value of one can predict (all or some
of) the values of the other. As I noted above, \texttt{Category} and
\texttt{Morph.Type} are collinear because all \texttt{Semi-Weak} and
\texttt{Past} tokens are also \texttt{Lexical}. Likewise
\texttt{Age.Group} and \texttt{Center.Age} are completely collinear; the
value of \texttt{Center.Age} can exactly predict the value of
\texttt{Age.Group}. Values for both age measures can also do a good job
of predicting both \texttt{Education} and \texttt{Job} in the Cape
Breton data. Younger speakers are generally \texttt{Educated} and have
\texttt{White} (collar) or \texttt{Service} jobs. All \texttt{Students}
in the variable \texttt{Education} are also \texttt{Students} for the
variable \texttt{Job}, and all are also \texttt{Young}. It would violate
the assumptions of the model to include all these predictors at the same
time in the same regression (see
\href{https://lingmethodshub.github.io/content/R/lvc_r/100_lvcr.html}{next
chapter}); however, these collinear variables can be included in a
random forest. One of the greatest uses of a random forest analysis is
when you have multiple ways of categorizing the same phenomenon (like
social class, which can be operationalized by either \texttt{Job} or
\texttt{Education}, or age which can be operationalized as a continuous
variable like \texttt{Center.Age}, or a discrete variable like
\texttt{Age.Group}) and you want to decide which categorization to use.

In the dotplot the variable \texttt{After} is higher ranked than
\texttt{After.New}. This would be good justification for choosing
\texttt{After} --- which includes a separate level for (t, d) before /h/
--- instead of \texttt{After.New} in further analyses. Likewise,
\texttt{Job} is describes the variation better than \texttt{Education};
the random forest therefore points to \texttt{Job} being a more useful
reflection of social class stratification than \texttt{Education} for
the (t, d) variable. The only caveat is that when selecting between
independent variables that are collinear, you must also take into
account the hypotheses that you aim to test. In other words, selection
of variables should be also be independently motivated.

A very important thing to remember about random forest analysis is that
they are truly random, so the results can vary from run to run depending
on which seed has been set. For this reason it's a good idea to verify
your findings by re-running your random forest analysis using different
seeds and perhaps different numbers of trees. Below is the code for
generating a random forest from seed 654321 and 8,000 randomly grown
trees (I also used \texttt{theme\_linedraw()} instead of
\texttt{theme\_classic()}). You'll see that the variable importance
rankings are nearly identical. \texttt{Before} is far-and-away the most
informative descriptor, followed by \texttt{Center.Age} and
\texttt{Morph.Type}, then \texttt{Job}, \texttt{After}, and
\texttt{Education}, and finally marginally-informative \texttt{Stress}
and \texttt{After.New}. This gives me confidence that the results from
the previous random forest are relatively accurate. It also provides
justification for choosing \texttt{Job} over \texttt{Education} as a
metric of social status, \texttt{Center.Age} over \texttt{Age.Group} as
a measure of apparent time, and \texttt{Morph.Type} over
\texttt{Category} (two highly-correlated descriptors). \texttt{Sex} and
\texttt{Phoneme} also appear to be poor descriptors of the variation in
the data, so leaving them out of regression analysis (in order to
simplify modelling), could also be justified.

\begin{Shaded}
\begin{Highlighting}[]
\FunctionTok{set.seed}\NormalTok{(}\DecValTok{654321}\NormalTok{)}
\FunctionTok{library}\NormalTok{(partykit)}
\NormalTok{td.cforest2 }\OtherTok{\textless{}{-}} \FunctionTok{cforest}\NormalTok{(Dep.Var }\SpecialCharTok{\textasciitilde{}}\NormalTok{ Stress }\SpecialCharTok{+}\NormalTok{ Category }\SpecialCharTok{+}
\NormalTok{    Morph.Type }\SpecialCharTok{+}\NormalTok{ Before }\SpecialCharTok{+}\NormalTok{ After }\SpecialCharTok{+}\NormalTok{ Sex }\SpecialCharTok{+}\NormalTok{ Education }\SpecialCharTok{+}
\NormalTok{    Job }\SpecialCharTok{+}\NormalTok{ After.New }\SpecialCharTok{+}\NormalTok{ Center.Age }\SpecialCharTok{+}\NormalTok{ Age.Group }\SpecialCharTok{+}\NormalTok{ Phoneme,}
    \AttributeTok{data =}\NormalTok{ td, }\AttributeTok{ntree =} \DecValTok{8000}\NormalTok{)}
\NormalTok{td.cforest2.varimp }\OtherTok{\textless{}{-}} \FunctionTok{varimp}\NormalTok{(td.cforest2, }\AttributeTok{conditional =} \ConstantTok{TRUE}\NormalTok{)}
\FunctionTok{library}\NormalTok{(reshape2)}
\NormalTok{td.cforest2.results }\OtherTok{\textless{}{-}} \FunctionTok{melt}\NormalTok{(}\FunctionTok{data.frame}\NormalTok{(}\FunctionTok{as.list}\NormalTok{(td.cforest2.varimp)),}
    \AttributeTok{id.vars =} \ConstantTok{NULL}\NormalTok{)}
\end{Highlighting}
\end{Shaded}

\begin{Shaded}
\begin{Highlighting}[]
\NormalTok{td.cforest2.results}\SpecialCharTok{$}\NormalTok{variable }\OtherTok{\textless{}{-}} \FunctionTok{reorder}\NormalTok{(td.cforest2.results}\SpecialCharTok{$}\NormalTok{variable,}
\NormalTok{    td.cforest2.results}\SpecialCharTok{$}\NormalTok{value)}
\NormalTok{td.cforest2.min }\OtherTok{\textless{}{-}} \FunctionTok{abs}\NormalTok{(}\FunctionTok{min}\NormalTok{(td.cforest2.results}\SpecialCharTok{$}\NormalTok{value))}
\FunctionTok{ggplot}\NormalTok{(td.cforest2.results, }\FunctionTok{aes}\NormalTok{(}\AttributeTok{x =}\NormalTok{ value, }\AttributeTok{y =}\NormalTok{ variable)) }\SpecialCharTok{+}
    \FunctionTok{geom\_point}\NormalTok{() }\SpecialCharTok{+} \FunctionTok{xlab}\NormalTok{(}\StringTok{"Variable Importance"}\NormalTok{) }\SpecialCharTok{+} \FunctionTok{ylab}\NormalTok{(}\ConstantTok{NULL}\NormalTok{) }\SpecialCharTok{+}
    \FunctionTok{theme\_linedraw}\NormalTok{() }\SpecialCharTok{+} \FunctionTok{geom\_vline}\NormalTok{(}\AttributeTok{xintercept =}\NormalTok{ td.cforest.min,}
    \AttributeTok{linetype =} \DecValTok{2}\NormalTok{, }\AttributeTok{alpha =} \FloatTok{0.5}\NormalTok{)}
\end{Highlighting}
\end{Shaded}

\begin{figure}[H]

{\centering \includegraphics{090_lvcr_files/figure-pdf/unnamed-chunk-12-1.pdf}

}

\end{figure}

Another important thing to remember when using random forest analyses is
that the values along the \texttt{x}-axis are only relative rankings for
a specific random forest. The absolute values shouldn't be interpreted
or compared across different studies Strobl, Malley, and Tutz (2009,
336). What you can compare across studies is the relative ranking of
variables along the \texttt{y}-axis and which variables are on each side
of the dashed line. Further, random forests do not take into account the
weirdness of sociolinguisic data, which generally includes an
inconsistent number of tokens from each speaker. Mixed-effects
regression \emph{can} do this. So, random forests should always be used
in conjunction with other forms of data modelling like mixed-effects
regression.

\hypertarget{references}{%
\subsubsection{References}\label{references}}

\hypertarget{refs}{}
\begin{CSLReferences}{1}{0}
\leavevmode\vadjust pre{\hypertarget{ref-Oshiro2012}{}}%
Oshiro, Thais Mayumi, Pedro Santoro Perez, and José Augusto Baranauskas.
2012. {``How Many Trees in a Random Forest?''} In \emph{Machine Learning
and Data Mining in Pattern Recognition}, edited by Petra Perner,
154--68. Berlin: Springer.

\leavevmode\vadjust pre{\hypertarget{ref-Strobl2009}{}}%
Strobl, Carolin, James Malley, and Gerhard Tutz. 2009. {``An
Introduction to Recursive Partitioning: Rationale, Application, and
Characteristics of Classification and Regression Trees, Bagging, and
Random Forests.''} \emph{Psychological Methods} 14 (4): 323--48.

\leavevmode\vadjust pre{\hypertarget{ref-Tagliamonte2012}{}}%
Tagliamonte, Sali A. 2012. \emph{Variationist Sociolinguistics: Change,
Observation, Interpretation}. {M}alden, MA: {W}iley-{B}lackwell.

\leavevmode\vadjust pre{\hypertarget{ref-Tagliamonte2012b}{}}%
Tagliamonte, Sali A., and R. Harald Baayen. 2012. {``{M}odels, Forests,
and Trees of {Y}ork {E}nglish: \emph{Was/Were} Variation as a Case Study
for Statistical Practice.''} \emph{{L}anguage {V}ariation and {C}hange}
24 (2): 135--78.

\end{CSLReferences}



\end{document}
